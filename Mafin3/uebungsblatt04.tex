\documentclass{article}
\usepackage{amsmath}
\usepackage{amssymb}

\title{Mathe 3 Übungsblatt 4}
\author{Pascal Diller, Timo Rieke}
\date{}

\begin{document}
\maketitle

\section*{Aufgabe 1}
\subsection*{(i)}

\[\sum_{k=0}^{\infty} \frac{e^{-k}}{2k!}x^{k+2} = x^2 \cdot \sum_{k=0}^{\infty} \frac{e^{-k}}{2k!}x^k\]
$x^2$ beeinflusst den Konvergenzradius nicht \\
Sei $c_k = \frac{e^{-k}}{2k!}$, dann:
\[\sum_{k=0}^{\infty}c_k x^k\]
Anwenden des Quotientenkriterium um Grenzwert $L = \lim_{k \to \infty}\left|\frac{c_{k+1}}{c_k}\right|$ zu berechnen.
\[L = \lim_{k \to \infty}\left|\frac{e^{-(k+1)}}{2(k+1)!} \cdot \frac{2k!}{e^{-k}}\right| = \lim_{k \to \infty}\left|\frac{e^{-k}\cdot e^{-1}}{e^{-k}} \cdot \frac{2k!}{2(k+1)\cdot k!}\right|\]
\[= \lim_{k \to \infty}\left|e^{-1} \cdot \frac{1}{k+1}\right|= \lim_{k \to \infty}\left|\frac{1}{e(k+1)}\right| = 0\]
Der Konvergenzradius $R$ ist $R = \frac{1}{L}$. Da $L = 0$, ist $R = \infty$

\subsection*{(ii)}
Der Entwicklugspunkt ist $x_0 = 2$ \\
Bestimmen vom Konvergenzradius $R$ fuer den Koeffizienten $c_k = (-1)^k \cdot \frac{k}{3^k}$ mit Quotientkriterium:
\[L = \lim_{k\to\infty}\left|\frac{c_{k+1}}{c_k}\right| = \lim_{k\to\infty}\left|\frac{(-1)^{k+1} \cdot (k+1)}{3^{k+1}} \cdot \frac{3^k}{(-1)^k \cdot k}\right| = \lim_{k\to\infty}\left|(-1) \cdot \frac{k+1}{k} \cdot \frac{3^k}{3^k \cdot 3}\right|\]
\[= \lim_{k\to\infty}\left(\frac{k+1}{k} \cdot \frac{1}{3}\right) = \lim_{k\to\infty}\left((1 + \frac{1}{k}) \cdot \frac{1}{3}\right)= \left((1+ 0) \cdot \frac{1}{3}\right) = \frac{1}{3}\]
Der Konvergenzradius ist $R = \frac{1}{L} = \frac{1}{1/3} = 3$ \\ 
\newline
Die Potenzreihe konvergiert fuer $|x-2| < 3$, was dem offenen Intervall $(2-3, \: 2+3) = (-1,5)$ entspricht.
\subsubsection*{1. Randpunkt}
$x = -1$
\[\sum_{k=1}^{\infty}(-1)^k \cdot \frac{k}{3^k} \cdot (-3)^k = \sum_{k=1}^{\infty}(-1)^k\cdot \frac{k}{3^k} \cdot (-1)^k3^k\]
\[= \sum_{k=1}^{\infty}(-1)^k (-1)^k \cdot k \cdot \frac{3^k}{3^k} = \sum_{k=1}^{\infty}(-1)^{2k} \cdot k = \sum_{k=1}^{\infty}k\]
Diese Reihe ist divergent, da ihre Glieder kein Nullfolge bilden.

\subsubsection*{2. Randpunkt}
$x = 5$
\[\sum_{k=1}^{\infty}(-1)^k \cdot \frac{k}{3^k} \cdot 3^k = \sum_{k=1}^{\infty}(-1)^k\cdot k \cdot \frac{3^k}{3^k}\]
\[= \sum_{k=1}^{\infty}(-1)^k k\]
Die Reihe $= \sum_{k=1}^{\infty}(-1)^k k = -1 + 2 - 3 + 4 - \ldots$ ist divergent, da ihre Glieder kein Nullfolge bilden.\\
\newline
Also konvergiert die Potenzreihe fuer $x \in (-1, 5)$ und divergiert an beiden Randpunkten.

\section*{Aufgabe 2}
\subsection*{($f_1$)}

Untersuchung des Verhaltens von $f_1(x)$ für $x \to 0$. Dazu nutzen wir die Taylor-Entwicklung der Kosinusfunktion um den Entwicklungspunkt $x_0 = 0$:
\[\cos x = 1 - \frac{x^2}{2!} + \frac{x^4}{4!} - \dots = 1 - \frac{x^2}{2} + O(x^4)\]
Für den Zähler von $f_1(x)$ ergibt sich:
\[1 - \cos x = 1 - \left( 1 - \frac{x^2}{2} + O(x^4) \right) = \frac{x^2}{2} - O(x^4)\]
Da $x^2 = |x|^2$ und $O(-x^4) = O(x^4)$ gilt:
\[1 - \cos x = \frac{1}{2}|x|^2 + O(|x|^4)\]
Einsetzen der Funktion $f_1(x)$:
\[f_1(x) = \frac{\frac{1}{2}|x|^2 + O(|x|^4)}{|x|^{1/2}} = \frac{1}{2}|x|^{2 - 1/2} + O(|x|^{4 - 1/2}) = \frac{1}{2}|x|^{3/2} + O(|x|^{7/2})\]
Das dominante Verhalten von $f_1(x)$ nahe $x=0$ ist also proportional zu $|x|^{3/2}$. \\
\newline
Analyse des Grenzwertes $\lim_{x\to 0} \frac{f_1(x)}{|x|^s}$:
\[\lim_{x\to 0} \frac{f_1(x)}{|x|^s} = \lim_{x\to 0} \frac{\frac{1}{2}|x|^{3/2} + O(|x|^{7/2})}{|x|^s} = \lim_{x\to 0} \left( \frac{1}{2}|x|^{3/2-s} + O(|x|^{7/2-s}) \right)\]

\textbf{(i) $f_1(x) = o(|x|^s)$ (für $x \to 0$)}
Diese Bedingung ist erfüllt, wenn der Grenzwert $0$ ist. Dies ist der Fall, wenn der Exponent des dominanten Terms $\frac{1}{2}|x|^{3/2-s}$ positiv ist:
\[\frac{3}{2} - s > 0 \implies s < \frac{3}{2}\]

\textbf{(ii) $f_1(x) = O(|x|^s)$ (für $x \to 0$)}
Diese Bedingung ist erfüllt, wenn der Grenzwert endlich (beschränkt) ist. Dies ist der Fall, wenn der Exponent des dominanten Terms nicht-negativ ist:
\[\frac{3}{2} - s \ge 0 \implies s \le \frac{3}{2}\]
(Für den Fall $s = 3/2$ ist der Grenzwert $\frac{1}{2}$, was endlich ist).

\textbf{Ergebnis für $f_1$:}
\begin{itemize}
    \item (i) $f_1(x) = o(|x|^s)$ gilt für $s \in (-\infty, 3/2)$.
    \item (ii) $f_1(x) = O(|x|^s)$ gilt für $s \in (-\infty, 3/2]$.
\end{itemize}

\subsection*{($f_2$)}

Untersuchen von $f_2(x)$ für $x \to 0$. Da $x^2 \to 0^+$ (für $x \to 0$), geht $1/x^2 \to \infty$ und somit der Exponent $-1/x^2 \to -\infty$. Die Funktion $f_2(x)$ geht also "extrem schnell" gegen $0$.

Analyse des Grenzwert $L = \lim_{x\to 0} \frac{f_2(x)}{|x|^s} $ für ein beliebiges $s \in \mathbb{R}$:
\[L = \lim_{x\to 0} \frac{e^{-1/x^2}}{|x|^s}\]
Substituieren $y = \frac{1}{|x|}$. Wenn $x \to 0$, geht $y \to \infty$.
Weiterhin gilt $1/x^2 = (1/|x|)^2 = y^2$ und $|x|^s = (1/y)^s = y^{-s}$.
Der Grenzwert lässt sich umschreiben zu:
\[L = \lim_{y\to \infty} \frac{e^{-y^2}}{y^{-s}} = \lim_{y\to \infty} \frac{y^s}{e^{y^2}}\]
Die Exponentialfunktion $e^{y^2}$ im Nenner wächst wesentlich schneller als jede Potenz $y^s$ im Zähler, unabhängig davon, wie groß $s$ gewählt wird. \\
Somit ist der Grenzwert für jedes $s \in \mathbb{R}$ gleich Null:
\[L = 0 \quad \text{für alle } s \in \mathbb{R}\]

\textbf{(i) $f_2(x) = o(|x|^s)$ (für $x \to 0$)}
Wir benötigen $\lim_{x\to 0} \frac{f_2(x)}{|x|^s} = 0$. Da $L=0$ für alle $s \in \mathbb{R}$, ist diese Bedingung für alle $s \in \mathbb{R}$ erfüllt.

\textbf{(ii) $f_2(x) = O(|x|^s)$ (für $x \to 0$)}
Wir benötigen $\limsup_{x\to 0} \left| \frac{f_2(x)}{|x|^s} \right| < \infty$. Da der Grenzwert $L=0$ (und $0$ eine endliche Zahl ist), ist diese Bedingung ebenfalls für alle $s \in \mathbb{R}$ erfüllt. \\

\textbf{Ergebnis für $f_2$:}
\begin{itemize}
    \item (i) $f_2(x) = o(|x|^s)$ gilt für alle $s \in \mathbb{R}$.
    \item (ii) $f_2(x) = O(|x|^s)$ gilt für alle $s \in \mathbb{R}$.
\end{itemize}

\end{document}
