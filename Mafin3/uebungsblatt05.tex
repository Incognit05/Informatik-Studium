\documentclass{article}
\usepackage{amsmath}
\usepackage{amssymb}

\title{Mathe 3 Übungsblatt 5}
\author{Pascal Diller, Timo Rieke}
\date{}

\begin{document}
\maketitle

\section*{Aufgabe 1}
\subsection*{(i)}

\subsection*{(ii)}
$f(0)=1$ \\
$f'(x)=x-logf(x)$ \\
$f'(0)=0-logf(0)=0-log1=0$ \\
$f''(x)=1-\frac{1}{f(x)}*f'(x)=1-\frac{f'(x)}{f(x)}$ \\
$f''(0)=1-\frac{0}{1}=1$ \\
$f'''(x)=-\frac{f''(x)*f(x)-f'(x)*f'(x)}{(f(x))^2}$\\
$f'''(0)=-\frac{f''(0)*f(0)-f'(0)*f'(0)}{(f(0))^2}= -\frac{1*1-0*0}{1^2} =-1$
\[T_3^0f(x)=\sum_{k=0}^3\frac{f^{(k)}(0)}{k!}(x-0)^k=\sum_{k=0}^3\frac{f^{(k)}(0)}{k!}(x)^k\]
\[=\frac{f(0)}{0!}(x)^0+\frac{f'(0)}{1!}(x)^1+\frac{f''(0)}{2!}(x)^2+\frac{f'''(0)}{3!}(x)^3\]
\[=\frac{1}{1}x^0+\frac{0}{1!}x^1+\frac{1}{2!}x^2+\frac{-1}{3!}x^3\]
\[=1+ \frac{1}{2}x^2-\frac{1}{6}x^3\]

\section*{Aufgabe 2}
$x_0=0$, $f(x)=sinh(x)$ \\
$f'(x)=cosh(x)$ \\
$f''(x)=sinh(x)$ \\
\newline
$f^{(2k)}= sinh(x)\rightarrow f^{(2k)}(0)=0$ \\
$f^{(2k+1)}= cosh(x)\rightarrow f^{(2k+1)}(0)=1$ \\
Da bei geraden Ableitungen 0 herauskommt, fallen diese weg.
\[T^0f(x)= \sum_{k=0}^\infty\frac{1}{(2k+1)!}(x)^{2k+1}\]
\[=\frac{1}{(2*0+1!)}x^{2*0+1}+\frac{1}{(2*1+1!)}x^{2*1+1}+\frac{1}{(2*2+1!)}x^{2*2+1}+...\]
\[=x+\frac{x^3}{3!}+\frac{x^5}{5!}+...\]
\newline
\[R^0_{m+1}f(x)=f^{(m+1)}(\xi)\frac{(x-x_0)^{m+1}}{(m+1)!}\]
\[|f^{(m+1}(\xi)|\leq cosh(|x|)\]
Somit gilt:
\[|R^0_{m+1}f(x)|=|f^{(m+1)}(\xi)\frac{(x-0)^{m+1}}{(m+1)!}| \leq cosh(|x|)\frac{(|x-0|)^{m+1}}{(m+1)!}\]
Grenzwert:
\[\lim_{m\to\infty}(cosh(|x|)\frac{(|x-0|)^{m+1}}{(m+1)!})\]
\[=cosh(|x|)\lim_{m\to\infty}(\frac{(|x-0|)^{m+1}}{(m+1)!})\]
Da $(m+1)!$ schneller wächst als $|x-0|^{m+1}$, ist: 
\[\lim_{m\to\infty}(\frac{(|x-0|)^{m+1}}{(m+1)!})=0\]
Somit ist ebenfalls:
\[\lim_{m\to\infty} R_m^0f(x)=0\]

\section*{Aufgabe 3}
\subsection*{(i)}
Solange $|-x^2| < 1$ kann man $q = -x^2$ substituieren und die Geometrische Reihe verwenden.
\[f(x) = \frac{1}{1-(-x^2)} = \sum_{k=0}^{\infty}(-x^2)^k\]
\[= \sum_{k=0}^{\infty}(-1)^k x^{2k} = 1 - x^2 + x^4 - x^6 + \dots\]
Also ist diese alterniernde Reihe die Taylorreihe um $x_0 = 0$ mit $|-x^2|<1$, da sie durch Substitution aus der Geometrischen Reihe hervorgekommen ist.
\subsection*{(ii)}
\[g'(x) = \frac{1}{1+x^2} = f(x)\]
\[\implies g(x) = \int f(t) dt = \int \left( \sum_{k=0}^{\infty}(-1)^k t^{2k} \right)dt\]
\[= \sum_{k=0}^{\infty}(-1)^k\frac{x^{2k+1}}{2k+1} + C\]
Entwicklungspunkt $x_0 = 0$ einsetzen um C zu bestimmen
\[g(0) = \arctan(0) = 0\]
\[0 = \sum_{k=0}^{\infty}(-1)^k\frac{0^{2k+1}}{2k+1} + C \implies C = 0\]
Also ist die Taylorreihe:
\[\sum_{k=0}^{\infty}(-1)^k\frac{x^{2k+1}}{2k+1} = x - \frac{x^3}{3} + \frac{x^5}{5} - \frac{x^7}{7} + \dots\]

\section*{Aufgabe 4}
Wir untersuchen die Folge $(a_n)_{n\in\mathbb{N}} = (x_n, y_n, z_n)$ komponentenweise auf Konvergenz. \\
\newline
\textbf{1. Komponente ($x_n$)}
\[x_n = \frac{\sqrt[n]{n}+3^n+n^3}{10+n^9+9^n}\]
Da der Term $9^n$ im Nenner am stärksten wächst, erweitern wir den Bruch mit $\frac{1}{9^n}$:
\[x_n = \frac{\frac{\sqrt[n]{n}}{9^n} + (\frac{3}{9})^n + \frac{n^3}{9^n}}{\frac{10}{9^n} + \frac{n^9}{9^n} + 1}\]
Bekanntlich gilt $\lim_{n\to\infty} \sqrt[n]{n} = 1$ und Exponentialfunktionen wachsen schneller als Polynome. Daher gehen alle Zählerterme und die ersten beiden Nennerterme gegen 0.
\[\implies \lim_{n\to\infty} x_n = \frac{0+0+0}{0+0+1} = 0\]
\textbf{2. Komponente ($y_n$)} \\
\newline
Hier liegt der Fall $\infty - \infty$ vor. Wir erweitern mit dem konjugierten Ausdruck:
\[y_n = \sqrt{n^2+n+1}-\sqrt{n^2+1} = \frac{(\sqrt{n^2+n+1}-\sqrt{n^2+1})(\sqrt{n^2+n+1}+\sqrt{n^2+1})}{\sqrt{n^2+n+1}+\sqrt{n^2+1}}\]
\[= \frac{(n^2+n+1) - (n^2+1)}{\sqrt{n^2+n+1}+\sqrt{n^2+1}} = \frac{n}{\sqrt{n^2(1+\frac{1}{n}+\frac{1}{n^2})} + \sqrt{n^2(1+\frac{1}{n^2})}}\]
\[= \frac{n}{n\left(\sqrt{1+\frac{1}{n}+\frac{1}{n^2}} + \sqrt{1+\frac{1}{n^2}}\right)} = \frac{1}{\sqrt{1+\frac{1}{n}+\frac{1}{n^2}} + \sqrt{1+\frac{1}{n^2}}}\]
Für $n \to \infty$ gehen die Terme $\frac{1}{n}, \frac{1}{n^2} \to 0$.
\[\implies \lim_{n\to\infty} y_n = \frac{1}{\sqrt{1}+\sqrt{1}} = \frac{1}{2}\]
\textbf{3. Komponente ($z_n$)} \\
\newline
\[z_n = \frac{1}{3}\log(n^3)-\log(n+2) = \log((n^3)^{1/3}) - \log(n+2)\]
\[= \log(n) - \log(n+2) = \log\left(\frac{n}{n+2}\right)\]
\[= \log\left(\frac{n}{n(1+\frac{2}{n})}\right) = \log\left(\frac{1}{1+\frac{2}{n}}\right)\]
Da der Logarithmus stetig ist und $\frac{2}{n} \to 0$:
\[\lim_{n\to\infty} z_n = \log(1) = 0\]
\newline
$\implies$ Die Folge konvergiert gegen den Vektor:
\[\lim_{n\to\infty} (x_n, y_n, z_n) = \left(0, \frac{1}{2}, 0\right)\]

\end{document}
