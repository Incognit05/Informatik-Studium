\documentclass{article}
\usepackage{amsmath}
\usepackage{amssymb}

\title{Mathe 3 Übungsblatt 3}
\author{Pascal Diller, Timo Rieke}
\date{}

\begin{document}
\maketitle

\section*{Aufgabe 2}
\[\int_0^4 f(x) dx = \int_0^1 (5x^4 + 3x^2)dx + \int_1^2 (\frac{1}{x})dx + \int_2^3 (\frac{3}{2}\sqrt{x-2})dx + \int_3^4(\frac{2x+1}{x^2 + x - 11})dx\]
\\
$\int_0^1 (5x^4 + 3x^2)dx$:
\[ \int_0^1 (5x^4 + 3x^2)dx = \bigl[x^5 + x^3\bigr]_0^1 = (1^5 + 1^3) - (0^5 + 0^3) = 2\]
$\int_1^2 (\frac{1}{x})dx$:
\[\int_1^2 (\frac{1}{x})dx = \bigl[\ln|x|\bigr]_1^2 = \ln(2)-\ln(1) = \ln(2)-0 = \ln(2)\]
\[\]
$\int_2^3 (\frac{3}{2}\sqrt{x-2})dx$: \\
\newline
$u = x - 2, \: du = dx$ \\
Grenzen: $x=2 \implies u=0, \: x=3 \implies u=1$
\[\int_0^1(\frac{3}{2} \sqrt{u}) du = \int_0^1 \frac{3}{2}u^{1/2}du = \bigl[\frac{3}{2}\cdot \frac{u^{3/2}}{3/2}\bigr]_0^1=[u^{3/2}]_0^1 = 1^{3/2}-0^{3/2} = 1\]
$\int_3^4(\frac{2x+1}{x^2 + x - 11})dx$: \\
\newline
$u = x^2 + x - 11, \: du = (2x+1)dx$ \\
Grenzen: \\
$x = 3 \implies u = 3^2 + 3 - 11 = 1$ \\
$x = 4 \implies u = 4^2 + 4 - 11 = 9$
\[\int_1^9 \frac{1}{9} du = [\ln|u|]_1^9 = \ln(9)-\ln(1) = \ln(9)\]
\newline
\[\int_0^4 f(x)dx = 2 + \ln(2) + 1 \ln(9) = 3 + \ln(2\cdot9) = 3 + \ln(18)\]

\section*{Aufgabe 3}
\[\sum_{k=0}^{\infty}\frac{1}{2^k}(x - 3)^k = \sum_{k=0}^{\infty}\frac{(x - 3)^k}{2^k} = \sum_{k=0}^{\infty}\left(\frac{x - 3}{2}\right)^k\]
Das ist die Geometrische Reihe mit Quotienten $r = \frac{x-3}{2}$. Sie konvergiert genau dann, wenn der Betrag des Quotienten kleiner als 1 ist.
\[|r| < 1 \implies \left|\frac{x-3}{2}\right| < 1\]
\[|x-3| < 2\]
\[-2 < x-3 < 2\]
\[1 < x  < 5\]
Also ist das groesst moegliche Intervall $(1,5)$, in dem die Reihe konvergiert.\\
\newline
Berechnung des Reihenwertes: \\
Einsetzen von $r$ in die Summenformel fuer eine konvergente geometrische Reihe:
\[S(x) = \frac{1}{1-(\frac{x-3}{2})} = \frac{1}{\frac{2}{2} - \frac{x-3}{2}} = \frac{1}{\frac{2-(x-3)}{2}} = \frac{1}{\frac{2-x+3}{2}} = \frac{1}{\frac{5-x}{2}} = \frac{2}{5-x}\]

\section*{Aufgabe 4}
\subsection*{(i)}
Zu zeigen: $\int_{-a}^a f(x)dx = 0$ \\
$f$ ist ungerade, $\implies f(-x) = -f(x)$
\[\int_{-a}^a f(x)dx = \int_{-a}^0 f(x)dx + \int_0^a f(x)dx\]
Substitution mit $t = -x, \: dt=-dx$ fuer $\int_{-a}^0 f(x)dx$
\[\int_{-a}^0 f(x)dx = \int_a^0 f(-t)(-dt) = \int_0^a f(-t)dt\]
\[=^{\text{ungerade Funktion}} \int_0^a -f(t)dt = - \int_0^a f(t)dt = - \int_0^a f(x)dx\]
\[\int_{-a}^a f(x)dx = -\int_0^a f(t)dt + \int_0^a f(x)dx = 0\]
\subsection*{(ii)}
Zu zeigen: $\int_{-a}^a g(x)dx + 2 \int_0^a g(x) dx$ \\
$g$ ist gerade $\implies g(-x) = g(x)$
\[\int_{-a}^a g(x)dx = \int_{-a}^0 g(x)dx + \int_0^a g(x)dx\]
Substitution im ersten Integral mit $t = -x$
\[\int_{-a}^0 g(x)dx = \int_a^0 g(-t)(-dt) = \int_0^a g(-t) dt\]
\[=^{\text{gerade Funktion}} \int_0^a g(t)dt = \int_0^a g(x)dx\]
\[\int_{-a}^a g(x)dx = \int_0^a g(x)dx + \int_0^a g(x)dx = 2 \int_0^a g(x)dx\]

\end{document}
