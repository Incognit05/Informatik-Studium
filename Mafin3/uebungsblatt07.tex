\documentclass{article}
\usepackage{amsmath}
\usepackage{amssymb}

\title{Mathe 3 Übungsblatt 7}
\author{Pascal Diller, Timo Rieke}
\date{}

\begin{document}
\maketitle

\section*{Aufgabe 1}
\subsection*{(i)}
\[\partial_x f(0,y) = \lim_{h \to 0} \frac{f(0 + h,y) - f(0,y)}{h}\]
\[\partial_x f(0,y) = \lim_{h \to 0} \frac{ f(h,y) - \frac{2 \cdot 0 \cdot y^3 - 2 \cdot 0^3 \cdot y}{0^2 + y^2}{} }{h}\]
\[\partial_x f(0,y) = \lim_{h \to 0} \frac{ f(h,y) - 0 }{h}\]
\[\partial_x f(0,y) = \lim_{h \to 0} \frac{ \frac{2hy^3 -2h^3y}{h^2+y^2} - 0 }{h}\]
\[\partial_x f(0,y) = \lim_{h \to 0} \frac{2hy^3 - 2h^3 y}{h(h^2 + y^2)}\]
\[\partial_x f(0,y) = \lim_{h \to 0} \frac{2hy^3 - 2h^3 y}{h(h^2 + y^2)}\]
\[\partial_x f(0,y) = \lim_{h \to 0} \frac{2y^3 - 2h^2 y}{h^2 + y^2}\]
\[\partial_x f(0,y) = \frac{2y^3}{y^2} = 2y\]
\newline
\[\partial_y f(x,0) = \lim_{h \to 0} \frac{f(x ,0 + h) - f(x,0)}{h}\]
\[\partial_y f(x,0) = \lim_{h \to 0} \frac{ \frac{2xh^3 - 2x^3h}{x^2 + h^2} }{h}\]
\[\partial_y f(x,0) = \lim_{h \to 0} \frac{ 2xh^3 - 2x^3h }{h(x^2 + h^2)}\]
\[\partial_y f(x,0) = \lim_{h \to 0} \frac{ 2xh^2 - 2x^3 }{x^2 + h^2}\]
\[\partial_y f(x,0) = \frac{-2x^3}{x^2} = -2x \]

\subsection*{(ii)}
Fuer $(x,y) \neq (0,0)$:
\[\partial_x f(x,y) = \frac{(2y^3 - 6x^2y)(x^2 + y^2) - (2xy^3 - 2x^3y)(2x)}{(x^2+y^2)^2}\]
\[\partial_x f(x,y) = \frac{2x^2y^3 + 2y^5 - 6x^4y - 6x^2y^3 - 4x^2y^3 + 4x^4y}{(x^2+y^2)^2}\]
\[\partial_x f(x,y) = \frac{y^5 - 4x^2y^3 - x^4y}{(x^2+y^2)^2} \cdot C\]
Fuer $(x,y) = (0,0)$: \\
\newline
Aus Teil(i): $\partial_x f(0,0) = 2\cdot 0 = 0$
Wir betrachten $\partial_x f(x,y)$ und nutzen die Polarkoordinaten $x = r \cos \phi, y = r \sin \phi$. \\
Der Zaehler von $f$ ist ein Polynom vom Grad 4, der Nenner Grad 2. Nach der Ableitung hat der Zaehler Grad 5 und der Nenner Grad 4.
\[\partial_x f \approx \frac{r^5}{r^4} = r\]
Da der Term linear in $r$ gegen 0 geht fuer $r \to 0$, ist die partielle Ableitung im Ursprung stetig. Dasselbe gilt analog fuer $\partial_y f$. \\
\newline
$\implies$ Da die partiellen Ableitungen ueberall existieren und stetig sind, ist $f \in \mathcal{C}^1 (\mathbb{R}^2)$

\subsection*{(iii)}
Aus $(i)$: $\partial_x f(0,y) = 2y$, $\partial_y f(x,0) = -2x$ \\
\newline
$\partial_y(\partial_x f)(0,0)$ ist die Ableitung von $\partial_x f$ nach $y$ an $(0,0)$. \\
Wir betrachten die Funktion $\partial_x f$ auf der y-Achse, also $\partial_x f(0,y) =2y$
\[\partial_y(\partial_x f)(0,0) = \frac{d}{dy}(2y)\Big|_{y=0} = 2\]
$\partial_x(\partial_y f)(0,0)$ ist die Ableitung von $\partial_y f$ nach $x$ an $(0,0)$. \\
Wir betrachten die Funktion $\partial_y f$ auf der x-Achse, also $\partial_y f(x,0) =-2x$
\[\partial_x(\partial_y f)(0,0) = \frac{d}{dx}(-2x)\Big|_{x=0} = -2\]

\subsection*{(iv)}
Nach dem Satz von Schwarz gilt fuer jede Funktion $f \in \mathcal{C}^2(\mathbb{R}^2)$, dass die gemischten partiellen Ableitungen vertauschbar sein muessen.
\[\partial_x \partial_yf(x,y) = \partial_y \partial_x f (x,y)\]
In $(iii)$ wurde jedoch gezeigt, dass im Punkt $(0,0)$ gilt:
\[\partial_y\partial_x f(0,0) = 2 \neq -2 = \partial_x\partial_yf(0,0)\]
Also gilt der Satz nicht. $\implies f \notin \mathcal{C}^2 (\mathbb{R}^2)$

\section*{Aufgabe 4}
Die Bogenlaenge einer Kurve $\gamma$ ist definiert durch das Integral ueber die Norm ihrer Ableitung:
\[L(\gamma) = \int_a^b ||\gamma'(t)|| dt\]
Bestimmung von $\gamma'(t)$: \\
\newline
Gegeben ist $\gamma(t) = (r(t)\cos t, r(t)\sin t)$. \\
\[x(t) = r(t)\cos t \implies x'(t) = r'(t)\cos t - r(t)\sin t\]
\[y(t) = r(t)\sin t \implies y'(t) = r'(t)\sin t + r(t)\cos t\]
\[\gamma'(t) = \begin{pmatrix} r'(t)\cos t - r(t)\sin t \\ r'(t)\sin t + r(t)\cos t \end{pmatrix}\]
Berechnung der Norm $||\gamma'(t)||$: \\
\newline
\[||\gamma'(t)||^2 = (x'(t))^2 + (y'(t))^2\]
\[||\gamma'(t)||^2 = (r'(t)\cos t - r(t)\sin t)^2 + (r'(t)\sin t + r(t)\cos t)^2\]
\[||\gamma'(t)||^2 = ((r')^2\cos^2 t - 2r'r\cos t\sin t + r^2\sin^2 t) + ((r')^2\sin^2 t + 2r'r\sin t\cos t + r^2\cos^2 t)\]
\[||\gamma'(t)||^2 = (r')^2\cos^2 t + r^2\sin^2 t + (r')^2\sin^2 t + r^2\cos^2 t\]
\[||\gamma'(t)||^2 = (r'(t))^2(\cos^2 t + \sin^2 t) + (r(t))^2(\sin^2 t + \cos^2 t)\]
\[||\gamma'(t)||^2 = (r'(t))^2 + (r(t))^2\]
\[||\gamma'(t)|| = \sqrt{(r(t))^2 + (r'(t))^2}\]
Einsetzen in die Bogenlaengenformel: \\
\newline
\[L(\gamma) = \int_a^b ||\gamma'(t)|| dt\]
\[L(\gamma) = \int_a^b \sqrt{(r(t))^2 + (r'(t))^2} dt\]

\end{document}
