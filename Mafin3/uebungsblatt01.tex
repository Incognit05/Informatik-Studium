\documentclass{article}
\usepackage{amsmath}
\usepackage{amssymb}

\title{Mathe 3 Übungsblatt 1}
\author{Pascal Diller, Timo Rieke}
\date{}

\begin{document}
\maketitle
\section*{Aufgabe 3}
\subsection*{(i)}
Seien $a<b$, $J:=[a,b]$ und $f \in C^0(J)$ mit $f \geq 0$. Sei $I \subset J$ ein Intervall. Dann gilt:
\[\int_I f(x) dx \leq \int_J f(x)dx\]
Beweis: \\
Sei $a<x<b, I:= [a,x]$ \\
Somit ist \(\int_If(x)dx + \int_x^b f(x)dx = \int_J f(x)dx \) \\
Da \(f\geq 0 \) ist \( \int f \geq 0\) \\
Sei $c \geq 0$ \\
Somit ist \(\int_I{f(x) dx} + c = \int_Jf(x)dx\) \\
Also ist \(\int_I f(x) dx \leq \int_J f(x)dx\)

\subsection*{(ii)}
Zu zeigen: \(\int_{-1}^{1} cosh(x) dx \geq 2\)

\[cosh(x):= \frac{1}{2}(e^x+e^{-x})\]
Ebenfalls wissen wir das $cosh(0)=1$, streng monton wachsend in $[0,\infty)$, sowie streng monton fallend in $(-\infty,0]$. \\
Außerdem ist cosh gerade, d.h. es gilt cosh(-x) = cosh(x) für $x\in \mathbb{R}$
\[cosh(1) = \frac{1}{2}(e+e^{-1}) \approx 1,54\]
Der kleinte Wert der Funktion ist bei $x=0$. Daraus folgt dass $cosh(x)\geq1$ für alle $x \in \mathbb{R}$. \\
Da $cosh(x) \geq 1$ folgt $\int_{-1}^1 cosh(x) dx \geq 1*(1-(-1)) =2$

\section*{Aufgabe 4}
\subsection*{(i)}
\[\int_0^1{(x^2+ax) dx} = \frac{4}{3}\]
\[[\frac{1}{3}x^3+\frac{1}{2}ax^2]_0^1 = \frac{4}{3}\]
\[\frac{1}{3}1^3+\frac{1}{2}a1^2 - (\frac{1}{3}0^3+\frac{1}{2}a0^2) = \frac{4}{3}\]
\[\frac{1}{3}+\frac{1}{2}a - 0 = \frac{4}{3}\]
\[\frac{1}{2}a  = 1 \] 
\[a=2\]

\subsection*{(ii)}
\[\int_0^\beta{x^2 dx}=9\]
\[[\frac{1}{3}x^3]_0^\beta = 9\]
\[\frac{1}{3}\beta^3-(\frac{1}{3}0^3)=9\]
\[\frac{1}{3}\beta^3=9\]
\[\beta^3=27\]
\[\beta=3\]

\subsection*{(iii)}
\[\int_3^0{(\gamma x^2-2x-2) dx}=-12\]
\[[(\frac{1}{3}\gamma x^3-\frac{1}{2}*2x^2-2x)]_3^0 = -12\]
\[(\frac{1}{3}\gamma 0^3-\frac{1}{2}*2*0^2-2*0)-(\frac{1}{3}\gamma 3^3-\frac{1}{2}*2*3^2-2*3=-12)\]
\[-(\frac{1}{3}\gamma 3^3-\frac{1}{2}*2*3^2-2*3=-12)\]
\[-(9\gamma -9-6)=-12\]
\[-9\gamma +15=-12\]
\[-9\gamma=-27\]
\[\gamma=3\]

\subsection*{(iv)}
\[\mu:=\frac{1}{4}\int_0^4{x^2-4x+1 dx}\]
\[=\frac{1}{4}[\frac{1}{3}x^3-2x^2+x]_0^4\]
\[=\frac{1}{4}(\frac{1}{3}*4^3-2*4^2+4-(\frac{1}{3}*0^3-2*0^2+0))\]
\[=\frac{1}{4}(\frac{64}{3}-32+4-0)\]
\[=\frac{1}{4}(\frac{64}{3}-28)\]
\[=\frac{64}{12}-7\]
\[= -\frac{5}{3}\]

\end{document}
