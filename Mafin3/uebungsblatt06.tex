\documentclass{article}
\usepackage{amsmath}
\usepackage{amssymb}

\title{Mathe 3 Übungsblatt 6}
\author{Pascal Diller, Timo Rieke}
\date{}

\begin{document}
\maketitle

\section*{Aufgabe 1}
\subsection*{(i)}
Sei $(x_k)_{k \in \mathbb{N}}$ eine beliebige Folge in $D \backslash \{x_0\}$ mit $\lim_{k \to \infty} x_k = x_0$. \\
Laut Vorraussetzung gilt fuer alle Folgeglieder: $g(x_k) \leq f(x_k) \leq h(x_k)$. \\
Da $\lim_{x \to x_0} g(x) = c$ und $\lim_{x \to x_0} h(x) = c$ folgt nach 11.2.1:
\[\lim_{k \to \infty} g(x_k) = c \text{ und } \lim_{k\to\infty}h(x_k) = c\]
Da die Folge $f(x_k)$ zwischen den beiden nach $c$ konvergierenden Folgen $f,h$ liegt, muss auch gelten:
\[\lim_{k \to \infty} f(x_k) = c\]
Da die Folge $(x_k)$ beliebig gewaehlt war, gilt dies auch fuer alle solche Folgen, was nach 11.2.1 bedeuted:
\[\lim_{x\to x_0}f(x) = c\]

\subsection*{(ii)}

\subsubsection*{(a)}
Fuer den Nenner gilt: $|x| + |y| \geq \sqrt{x^2 + y^2}$, da $(|x| + |y|)^2 = x^2 + 2|xy| + y^2 \geq x^2 + y^2$ \\
Fuer $(x,y) \neq (0,0)$ folgt: 
\[0 \leq \frac{x^2 + y^2}{|x| + |y|} \leq \frac{x^2 + y^2}{\sqrt{x^2 + y^2}} = \sqrt{x^2 + y^2}\]
Grenzuebergang von $(x,y) \to (0,0)$: \\
linke Seite konstant 0. \\
rechte Seite $\sqrt{x^2 + y^2}$ strebt gegen 0. \\
Dann gilt nach dem Sandwichkriterium: 
\[\lim_{(x,y) \to (0,0)} \frac{x^2 + y^2}{|x| + |y|} = 0\]

\subsubsection*{(b)}
Untersuchen von verschiedenen Pfaden um zu ueberpruefen, ob ein Grenzwert existiert: \\
\newline
1. Pfad: Entlang der Geraden $y = -x$ (fuer $x \neq 0$)
\[f(x, -x) = \frac{e^{x-x} - x - (-x) - 1}{ x^2 + (-x)^2 } = \frac{e^0 - 0 - 1}{2x^2} = 0\]
$\implies$ Grenzwert hier ist 0 \\
\newline
2. Pfad: Entlang der Geraden $y = x$ (fuer $x > 0$)
\[f(x,x) = \frac{e^{2x} - 2x - 1}{2x^2}\]
Die Taylor-Entwicklung fuer $e^t$ um $t = 0$ mit $t = 2x$.
\[e^{2x} \approx 1 + 2x + \frac{1}{2}(2x)^2 = 1 + 2x + 2x^2\]
einsetzen:
\[\lim_{x \to 0} \frac{(1 + 2x + 2x^2) - 2x - 1}{2x^2} = \lim_{x \to 0} \frac{2x^2}{2x^2} = 1\]
$\implies$ Grenzwert hier ist 1. \\
\newline
Da die Grenzwerte auf den verschiedenen Pfaden unterschiedlich sind, existiert der Grenzwert nicht.

\section*{Aufgabe 2}

\subsection*{(i)}
Fuer $(x,y) \neq (0,0)$ ist $f$ stetig, da es sich um Polynome handelt und der Nenner $\neq 0$ ist. \\
\newline
Fuer $(x,y) = (0,0)$: Untersuchung mit Pfaden \\
\newline
1. Pfad: $y = 0$
\[ \lim_{x \to 0} \frac{0 - x^2}{x^2 + 0} = \lim_{x \to 0} (-1) = -1 \] 
2. Pfad: $x = 0$
\[ \lim_{y \to 0} \frac{y^2 - 0}{0 + y^2} = \lim_{y \to 0} (1) = 1 \] 
Da $1 != -1$ existiert der Grenzwert nicht. \\
\newline
$\implies f$ ist in $(0,0)$ nicht stetig. \\
$\implies f$ ist in $\mathbb{R}^2 \backslash (0,0)$ stetig. \\

\subsection*{(ii)}
Fuer $(x,y) \in \mathbb{R}^2 \backslash (0,0)$: $g$ ist stetig, da der Sinus stetig ist und $f$ dort ebenfalls stetig ist. \\
\newline
Fuer $(x,y)= (0,0)$: zu zeigen: $\lim_{(x,y) \to (0,0)} g(x,y) = 0$ \\
Untersuchen den Betrag von $g(x,y)$
\[ |g(x,y)| = |\sin(x+y)| \cdot \left| \frac{y^2 - x^2}{x^2 + y^2} \right| \]
Nach der Dreiecksungleichung gilt: $|y^2 - x^2| \leq x^2 + y^2$. Daraus folgt:
\[ \left| \frac{y^2 - x^2}{x^2 + y^2} \right| \leq \frac{x^2 + y^2}{x^2 + y^2} = 1 \]
$\implies $ der Term aus $f(x,y)$ ist also beschraenkt. \\
\newline
Da der Sinus stetig ist, gilt:
\[ \lim_{(x,y) \to (0,0)} \sin(x+y) = \sin(0) = 0\]
Aus dem Produkt der Nullfolge $(\sin)$ und der beschraenkten Function $(f)$ folgt:
\[\lim_{(x,y) \to (0,0)} g(x,y) = 0 \]
Da der Grenzwert $0 = g(0,0)$ ist, ist $g$ im Ursprung stetig. \\
\newline
$\implies$ $g$ ist in $\mathbb{R}^2$ stetig.

\section*{Aufgabe 3}
\[f(x, y) := \frac{e^{x+y} - x - y - 1}{x^2 + y^2}\]
\[\frac{\partial f}{\partial x}(0,0) = \lim_{h \to 0} \frac{f(0+h, 0) - f(0,0)}{h}\]
\[f(h, 0) = \frac{e^{h+0} - h - 0 - 1}{h^2 + 0^2} = \frac{e^h - h - 1}{h^2}\]
sei $c:=f(0,0)$
\[\frac{\partial f}{\partial x}(0,0) = \lim_{h \to 0} \frac{\frac{e^h - h - 1}{h^2} - c}{h} = \lim_{h \to 0} \frac{e^h - h - 1 - c \cdot h^2}{h^3}\]
Taylorentwicklung der Expontentialfunktion um $x_0=0$
\[e^h = 1 + h + \frac{h^2}{2} + \frac{h^3}{6}\]
Somit ist der Zähler:
\[(1 + h + \frac{h^2}{2} + \frac{h^3}{6} ) - h - 1 - c h^2\]
\[= (\frac{1}{2} - c)h^2 + \frac{h^3}{6}\]
\[\frac{1}{2} - c = 0 \implies c = \frac{1}{2}\]
Somit ist:
\[f(0,0) := \frac{1}{2}\]
und:
\[\frac{\partial f}{\partial x}(0,0) = \lim_{h \to 0} \frac{\frac{1}{6}h^3}{h^3} = \lim_{h \to 0} (\frac{1}{6}) = \frac{1}{6}\]

\section*{Aufgabe 4}
\subsection*{(i)}
\[f(x):=(3x+5y)^4\]
\[\frac{\partial f}{\partial x}(x,y)= 4*(3x+5y)^3*3=12(3x+5y)^3\]
\[\frac{\partial f}{\partial y}(x,y)= 4*(3x+5y)^3*5=20(3x+5y)^3\]
\newline
\[\frac{\partial'' f}{\partial x}(x,y)=12*3(3x+5y)^3*3=108(3x+5y)^3\]
\[\frac{\partial'' f}{\partial y}(x,y)=20*3(3x+5y)^3*5=300(3x+5y)^3\]
\[\frac{\partial'' f}{\partial x\partial y}(x,y)=\frac{\partial}{\partial y} (12(3x+5y)^2)=12*3(3x+5y)^2*5=180(3x+5y)^2\]

\subsection*{(ii)}
\[g(x,y,z):=3xe^{xyz}\]
\[\frac{\partial g}{\partial x}(x,y,z) = 3 \cdot e^{xyz} + 3x \cdot (yz \cdot e^{xyz}) = 3e^{xyz}(1 + xyz)\]
\[\frac{\partial g}{\partial y}(x,y,z) = 3x \cdot (xz \cdot e^{xyz}) = 3x^2z e^{xyz}\]
\[\frac{\partial g}{\partial z}(x,y,z) = 3x \cdot (xy \cdot e^{xyz}) = 3x^2y e^{xyz}\]
\newline

\[\frac{\partial^2 g}{\partial x^2} = 3(yz)e^{xyz} + [3yz \cdot e^{xyz} + 3xyz(yz)e^{xyz}] = 3yze^{xyz}(2 + xyz)\]
\[\frac{\partial^2 g}{\partial y \partial x} = \frac{\partial}{\partial y}(3x^2z e^{xyz}) = 3x^2z(xz)e^{xyz} = 3x^3z^2 e^{xyz}\]

\[ \frac{\partial^2 g}{\partial x \partial y} = 6xz e^{xyz} + 3x^2z(yz)e^{xyz} = 3xz e^{xyz}(2 + xyz)\]
\[ \frac{\partial^2 g}{\partial y^2} = 3x^2z \cdot (xz)e^{xyz} = 3x^3z^2 e^{xyz}\]
\[ \frac{\partial^2 g}{\partial z \partial y} = 3x^2 e^{xyz} + 3x^2z(xy)e^{xyz} = 3x^2 e^{xyz}(1 + xyz)\]

\[ \frac{\partial^2 g}{\partial x \partial z} = 6xy e^{xyz} + 3x^2y(yz)e^{xyz} = 3xy e^{xyz}(2 + xyz)\]
\[\frac{\partial^2 g}{\partial y \partial z} = 3x^2 e^{xyz} + 3x^2y(xz)e^{xyz} = 3x^2 e^{xyz}(1 + xyz)\]
\[ \frac{\partial^2 g}{\partial z^2} = 3x^2y \cdot (xy)e^{xyz} = 3x^3y^2 e^{xyz}\]

\end{document}
