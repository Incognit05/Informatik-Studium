\documentclass{article}
\usepackage{amsmath}
\usepackage{amssymb}

\title{Mathe 3 Übungsblatt 1}
\author{Pascal Diller, Timo Rieke}
\date{}

\begin{document}
\maketitle

\section*{Aufgabe 1}
\subsection*{(i)}

\[\int^{2}_{-1} \left( \frac{3}{2} x^2 + 2x - 5\right)dx = \bigl[ \frac{1}{2}x^3 + x^2 - 5x \bigr]^2_{-1}\]
\[ = (\frac{1}{2} \cdot 2^3 + 2^2 - 5 \cdot 2) - (\frac{1}{2} \cdot (-1)^3 + (-1)^2 - 5 \cdot (-1) ) = (4+4-10)-(-0.5 +1 + 5) = - 7.5\]

\subsection*{(ii)}
\[ \int^{100}_{-100} \sinh (x) dx= \bigl[ \cosh (x) \bigr]^{100}_{-100} = \cosh(100)-\cosh(-100) \]
\[ \text{da } \cosh(100) = \cosh(-100) \text{ , }\int^{100}_{-100} \sinh (x) dx= 0\]

\subsection*{(iii)}
\[\int^1_{-1}(|x| + 1) dx = \int^0_{-1}(-x + 1) dx + \int^1_0 (x + 1) dx\]
\[= \bigl[ -\frac{1}{2}x^2 + x \bigr]^0_{-1} + \bigl[\frac{1}{2}x^2 + x\bigr]^1_0\]
\[= \left(\left( -\frac{1}{2}0^2 + 0 \right) - \left( -\frac{1}{2}(-1)^2 + (-1) \right) \right)  + \left(\left( -\frac{1}{2}1^2 + 1 \right) - \left( -\frac{1}{2}0^2 + 0 \right) \right)\]
\[= \frac{3}{2} + \frac{3}{2} = 3\]

\subsection*{(iv)}
\[\int_0^{\ln 2} \tanh(x) dx = \int_0^{\ln 2} \frac{\sinh(x)}{\cosh(x)} dx\]
Sei $u = \cosh(x)$, dann ist $\frac{du}{dx} = \sinh(x) $%\Longleftrightarrow du = \sinh(x)dx$
\[\int_0^{\ln 2} \frac{\frac{du}{dx}}{u}dx = \int_0^{\ln 2} \frac{du}{dx} \frac{dx}{u} =  \int_0^{\ln 2} \frac{du}{u} = \int_0^{\ln 2} \frac{1}{u}du\]
Die Stammfunktion von $\frac{1}{u}$ ist $\ln(u) \implies \ln|\cosh(x)|$ \\
Da $\cosh(x) = \frac{e^x + e^{-x}}{2} \geq 1$ kann man schreiben: $\ln(\cosh(x))$
\[\implies \int_0^{\ln 2} \tanh(x) dx = \bigl[ \ln(\cosh(x)) \bigr]_0^{\ln(2)} = \ln(\cosh(\ln(2))) - \ln(\cosh(0))\]
\[= \ln(\frac{e^{\ln(2)} + e^{-\ln(2)}}{2}) - \ln (\frac{e^0 + e^0}{2}) = \ln(\frac{2 + \frac{1}{2}}{2}) - \ln(\frac{1+1}{2})\]
\[= \ln(\frac{5}{4}) - \ln(1) = \ln(\frac{5}{4}) - 0 = \ln(\frac{5}{4})\]
\[\implies \int_0^{\ln 2} \tanh(x)dx = \ln(\frac{5}{4})\]

\section*{Aufgabe 4}
\subsection*{(i)}
Nach dem Mittelwertsatz gilt fuer $J = [a,b]$:
\[\frac{1}{|J|} \cdot \int_{J} f(x)dx =  f(c) \Longleftrightarrow \int_J f(x) dx = f(c) \cdot |J| \]
Es gilt: $\int_J f(x) dx = 0$
\[0 = f(c) \cdot |J| = f(c) \cdot (b - a)\]
Da $b > a$, muss $|J| > 0$, also muss $f(c) = 0$ um die Gleichung zu erfuellen.  \\
Also gibt es eine Zahl $c \in [a, b]$ mit $f(c) = 0$

\subsection*{(ii)}
Beweis durch Widerspruch: \\ 
Angenommen $g(x)$ hat keine Nullstelle in $[a,b]$ \\
Da $g$ keine stetig ohne Nullstelle ist, muessen alle $g(x)$ fuer $x \in [a,b]$ entweder $> 0$ oder $< 0$ sein. \\
\newline
Fuer $g(x) > 0$: \\
Da $f(x) \geq \epsilon > 0$, also $f(x) > 0$, ist $f(x) \cdot g(x) > 0$ \\
Das Integral einer stetigen Function $h$ mit $h(x) > 0$ ueber $[a,b]$, ist ebenfalls ueber das Intervall positiv. \\
Das ist ein Widersprucht zur Angabe, dass $\int_{a}^{b}f(x)g(x)dx = 0$ \\
\newline
Fuer $g(x) < 0$: \\
Da $f(x) > 0$ ist $f(x) \cdot g(x) < 0$
Das Integral einer stetigen Funktion, die ueberall negativ ist, ist ebenfalls negativ. \\
Das ist auch ein Widerspruch zur Angabe. \\
\newline
$\implies g(x)$ muss mindestens eine Nullstelle $c \in [a,b]$ haben. 

\end{document}
