\documentclass{article}
\usepackage[utf8]{inputenc}
\usepackage{amsmath, amssymb, amsthm, todonotes}
\title{Mathe Übung 11}
\author{Pascal Diller, Timo Rieke}
\begin{document}
\setcounter{secnumdepth}{0}
\maketitle

\section*{Aufgabe 1}
\subsection*{(i)}
\[|I_n| = \left(1 + \frac{1}{n^2}\right) - \left(1- \left(\frac{1}{2}\right)^n\right) = \frac{1}{n^2} + \left(\frac{1}{2}\right)^n\]
\[\lim_{n\to\infty} \frac{1}{n^2} = 0, \lim_{n\to\infty} \left(\frac{1}{2}\right)^n = 0 \implies |I_n| \xrightarrow{n\to\infty} 0\]
\[I_{n+1} = \left[ 1-\left(\frac{1}{2}\right)^{n+1}, 1 + \frac{1}{(n+1)^2} \right]\]
Da $\left(\frac{1}{2}\right)^{n+1} < \left(\frac{1}{2}\right)^n$ und $\frac{1}{(n+1)^2} < \frac{1}{n^2}$, gilt: $I_{n+1} \subset I_n$ für jedes $n$. \\
Somit ist gezeigt, dass $I_n$ eine Intervallverschachtelung ist.
\subsection*{(ii)}
\[|J_n| = \frac{3n + 5}{n} - \left(3 - \frac{1}{n}\right) = \frac{3n}{n} + \frac{5}{n} - 3 + \frac{1}{n} = \frac{6}{n}\]
Da $\frac{6}{n} \to 0$ für $n \to \infty$, folgt $|J_n| \to 0$
\[J_{n+1} = \left[ 3 - \frac{1}{n+1}, \frac{3(n+1) + 5}{n+1} \right]\]
$3 - \frac{1}{n+1} < 3 - \frac{1}{n}$, da $\frac{1}{n+1} < \frac{1}{n}$ und $\frac{3(n+1) + 5}{n+1} < \frac{3n+5}{n}$ da der Bruch mit größerem $n$ kleiner wird. Somit folgt: $J_{n+1} \subset J_n$. \\
\newline
Somit ist gezeigt, dass $J_n$ eine Intervallverschachtelung ist.

\section*{Aufgabe 2}
\subsection*{(i)}
Sei $a_n = \left( 1 - \frac{1}{n} \right)^n$
\[\ln\left( a_n \right) = n \ln\left(1 - \frac{1}{n}\right)\]
Aus der Taylor-Reihe für $\ln(1-x)$ gilt: $\ln(1-x) \approx -x$. Also: $\ln\left(1 - \frac{1}{n}\right) \approx - \frac{1}{n}$
\[\ln(a_n) = n \ln\left(1-\frac{1}{n}\right) \approx n \left(-\frac{1}{n}\right) = -1\]
\[a_n = e^{\ln(a_n)} \approx e^{-1} = \frac{1}{e}\]
Also: $\lim_{n\to\infty} \left(1 - \frac{1}{n}\right)^n = \frac{1}{e}$
\subsection*{(ii)}
\subsubsection*{(a)}
\[a_n = \left(1 + \frac{1}{n}\right)^{2n} = \left[ \left(1 + \frac{1}{n}\right)^n \right]^2\]
Es gilt: $\lim_{n\to\infty} \left(1+ \frac{1}{n}\right)^n = e$
\[\lim_{n\to\infty}a_n = e^2\]
\subsubsection*{(b)}
Sei $b_n = \left(1 + \frac{1}{n}\right)^{n^2}$
\[\ln(b_n) = n^2 \ln \left(1+\frac{1}{n}\right)\]
Es gilt: $\ln(1+x) \approx x$
\[\ln(b_n) \approx n^2 \frac{1}{n} = n\]
\[b_n = e^{\ln(b_n)} \approx e^n\]
Also: $b_n \to \infty \implies$ die Folge divergiert.
\subsubsection*{(c)}
\[\frac{(n+1)!}{n!} \frac{1}{(n+2)^n} = (n+1) \frac{1}{(n+2)^n} = \frac{n+1}{(n+2)^n}\]
Bei großem $n$ dominiert der Nenner über den Zähler, also strebt der Ausdruck gegen 0.

\section*{Aufgabe 3}

\subsection*{(i)}

    \[a_n = \frac{n - (-1)^n}{4 + (-1)^n n}\]
    Die Häufungspunkte der Folge \((a_n)_{n \in \mathbb{N}}\) sind: \(\frac{1}{4}\) und \(\frac{1}{5}\).

    \[b_n = n^2\frac{ n!}{(n+1)!}\]
    Die Häufungspunkte der Folge \((b_n)_{n \in \mathbb{N}}\) sind: \(1\).

\subsection*{(ii)}

Die Folge \((a_n)_{n \in \mathbb{N}}\) divergiert, da sie zwei unterschiedliche Häufungspunkte besitzt, nämlich \(\frac{1}{4}\) und \(\frac{1}{5}\). Da eine konvergente Folge nur einen einzigen Häufungspunkt haben kann, muss \((a_n)_{n \in \mathbb{N}}\) divergent sein.

\subsection*{(iii)}


    \[(w_n)_{n \in \mathbb{N}} = (0, 2, 4, 6, 8, \dots)\]  
    \((w_n)_{n \in \mathbb{N}}\) ist eine Teilfolge von \((c_n)_{n \in \mathbb{N}}\).

    \[(x_n)_{n \in \mathbb{N}} = (1, 3, 5, 7, 9, \dots)\]  
    \((x_n)_{n \in \mathbb{N}}\) ist eine Teilfolge von \((c_n)_{n \in \mathbb{N}}\).

    \[(y_n)_{n \in \mathbb{N}} = (n^2)_{n \in \mathbb{N}}\]  
    \((y_n)_{n \in \mathbb{N}}\) ist keine Teilfolge von \((c_n)_{n \in \mathbb{N}}\).

    \[(z_n)_{n \in \mathbb{N}} = (1, 3, 2, 5, 7, 6, 9, 11, 10, \dots)\]  
    \((z_n)_{n \in \mathbb{N}}\) ist keine Teilfolge von \((c_n)_{n \in \mathbb{N}}\).


\section*{Aufgabe 4}

Sei \((a_n)_{n \in \mathbb{N}}\) eine monoton wachsende und nach oben beschränkte Folge. Das bedeutet:
\begin{center}
    \(a_n \leq a_{n+1}\) für alle \(n \in \mathbb{N}\). \\
    es existiert ein \(M \in \mathbb{R}\) mit \(a_n \leq M\) für alle \(n \in \mathbb{N}\).
\end{center}

Nach dem Bolzano-Weierstraß-Satz besitzt jede beschränkte Folge eine konvergente Teilfolge. Da \((a_n)_{n \in \mathbb{N}}\) monoton wachsend ist, muss jede Teilfolge ebenfalls monoton wachsend sein und denselben Grenzwert haben.\\

Sei \(b_n\) eine konvergente Teilfolge von \((a_n)\) mit Grenzwert \(L\). Da \((a_n)\) monoton wachsend ist und \(a_n \leq M\), folgt, dass \(a_n \to L\) für \(n \to \infty\). \\

Daher konvergiert die gesamte Folge \((a_n)\) gegen \(L\).

\end{document}


\end{document}
