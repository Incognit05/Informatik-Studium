\documentclass{article}
\usepackage[utf8]{inputenc}
\usepackage{amsmath, amssymb, amsthm, todonotes}

\title{Übung 10}
\author{Pascal Diller, Timo Rieke}

\begin{document}

\maketitle

\section{Aufgabe 1}
\[(a_n)_{n\in \mathbb{N}}=(\frac{1-n^2}{-1-n})_{n \in \mathbb{N}} = \frac{-(n^2-1)}{-(n+1)}= \frac{(n^2-1)}{(n+1)}\]
\[= \frac{(n-1)(n+1)}{(n+1)} = n-1\]
Die Folge ist nach unten beschränkt durch $-1$ aber nach oben unbeschränkt.
Außerdem ist die Folge $a_n = n-1$ für $n \in \mathbb{N}$ streng monton wachsend.

\section{Aufgabe 2}
\[b_n = 6-(\frac{6+n^2}{n})\]
Vereinfachen der Folge:
\[b_n = 6-(\frac{6}{n}+\frac{n^2}{n}) = 6- (\frac{6}{n}+n) = 6-\frac{6}{n}-n\]

\subsection{(i)}
\begin{center}
    $\frac{6}{n}$ geht gegen 0, wenn $n \xrightarrow{} \infty$ \\
    $-n$ geht gegen $-\infty$, wenn $n \xrightarrow{} \infty$ 
\end{center}
Somit konvergiert $b_n$ gegen $-\infty$, was bedeuted das die Folge nach unten unbeschränkt ist und nach oben mit 6 beschränkt ist.

\subsection{(ii)}
\[B_n= b_{n+1}-b_n\]

Berechnen von $B_n$:
\[B_n = b_{n+1}-b_2 = (6-\frac{6}{n+1}-(n+1))-(6-\frac{6}{n}-n)\]
\[= -\frac{6}{n+1}+\frac{6}{n}-1 = \frac{6n-6(n+1)}{n(n+1)}-1= \frac{6n-6n-6}{n(n+1)}-1\]
\[=\frac{-6}{n(n+1)}-1\]
Da $\frac{-6}{n(n+1)}-1$ für alle $n \in \mathbb{N}$ negativ ist, ist $B_n<0$.

\subsection{(iii)}
Da $B_n <0$ für alle $n \geq 1$ ist die Folge streng monton fallend ab $n=1$.
Somit ist der minimale Wert für $m$ und $l$: $m=1$, $l=1$

\section{Aufgabe 3}
\[(c_n)_{n \in \mathbb{N}}=\frac{A}{2n^3-15n}+B\]


\end{document}
