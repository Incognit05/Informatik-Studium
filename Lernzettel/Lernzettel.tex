\documentclass[12pt]{article}

\usepackage[utf8]{inputenc}
\usepackage{amsmath, amssymb, amsthm}
\usepackage{hyperref}
\usepackage{graphicx}
\usepackage{tikz}
\usepackage{multirow}
\usepackage{enumitem}
\usepackage{cancel}
\usepackage[table,x11names]{xcolor}
\usetikzlibrary{arrows}

\title{Lernzettel}
\author{Pascal Diller}

\setcounter{secnumdepth}{0}

\begin{document}

\maketitle
\newpage

\tableofcontents

\newpage
\section{Logik}
\begin{itemize}
    \item ''$\land$'': \textbf{Und}
    \item ''$\lor$'': \textbf{Oder}
    \item ''$\lnot$'': \textbf{Nicht} (Verneinung)
    \item $A \implies B$: $A \textbf{ impliziert} B$
    \item $A \impliedby B$: $A \text{ wird durch } B \textbf{ impliziert}$
    \item $A \Longleftrightarrow  B$: $A \text{ ist äquivalent zu } B$
    \subitem{Es gilt: $A \implies B$ und $A \impliedby B$}
    \item $\forall$: \textbf{Für alle}
    \item $\exists$: \textbf{Es existiert (mindestens) ein}
\end{itemize}

\section{Mengen}
Eine \textbf{Menge} ist eine Zusammenfassung von (mathematischen) Objekten. \\
Die Objekte in einer Menge werden als \textbf{Elemente} bezeichnet.
 
\begin{itemize}
    \item $x \in M$: x in/Element M
    \item $x \notin M$: x nicht in/Element M
\end{itemize}
\vspace{1ex}
Defintion einer Menge:
\begin{itemize}
    \item Aufzählung:
    \begin{itemize}
        \item[] $M_1 = \{0, 1, 2, 3, 5, 8, -1\}; \hspace{2ex} M_2 = \{1, 2, 3, 4, 5, \dots\}$
        \item[] Es kommt \textbf{nicht} auf die \textbf{Reihenfolge} und \textbf{nicht} auf \\ \textbf{Verdopplungen} an:
        $\{1, 3, 2, 3\} = \{3, 2, 1\} = \{1, 2, 3\}$
    \end{itemize}
    \item Beschreibung:
    \[M_3 = \{x \in \mathbb{R} : x \geq -1 \land x \leq 1\} = [-1, 1]\]
\end{itemize}
\vspace{1ex}
Menge $B$ ist eine \textbf{Teilmenge} von Menge $A$, wenn für jedes $x \in A$ auch $x \in B$ gilt.
\begin{itemize}
    \item $A \subset B$ (''$A$ ist eine Teilmenge von $B$'')
    \item $A \supset B$ (''$B$ ist eine Teilmenge von $A$'')
\end{itemize}
\vspace{1ex}
Mengenoperationen:
\begin{itemize}
    \item \textbf{Vereinigung} der Mengen $A$ und $B$
    \subitem{$A \cup B = \{x : x \in A \lor x \in B\}$ (''A vereinigt B'')}
    \item \textbf{Durchschnitt} der Mengen $A$ und $B$
    \subitem{$A \cap B = \{x : x \in A \land x \in B\}$} (''A geschnitten B'')
    \item \textbf{Differenzmenge} der Mengen $A$ und $B$
    \subitem{$A \setminus B = \{x : x \in A \land x \notin B\}$} (''A ohne B'')
\end{itemize}
\vspace{1ex}
\textbf{Kartesisches Produkt:} \\
sei $n \in \mathbb{N}$ und seien $X_1, \dots, X_n$ Mengen, dann ist
\[X_1 \times \dots \times X_n = \{(x_1, \dots, x_n) : x_i \in X_i, \text{ für } i = 1,\dots,n\}\]
die Menge der n-\textbf{Tupel} mit $i$-ter Koordinate $x_i$ in $X_i$ für $i = 1,\dots,n$. \\
\newline
\textbf{Potenzmenge:} \\
Die Menge aller Teilmengen einer Menge $X$ heißt Potenzmenge von $X$ und wird mit $\mathcal{P(X)}$ bezeichnet:
\[\mathcal{P}(X) = \{Y: Y \subset X\}\]
Es gilt immer: $\emptyset \in \mathcal{P}(X)$ und $X \in \mathcal{P}(X)$.\\
Beispiel: Sei $X = \{1, 2, 3\}$. Dann ist \[\mathcal{P}(X) = \{\emptyset, \{1\}, \{2\}, \{3\}, \{1, 2\}, \{1, 3\}, \{2. 3\}, \{1, 2, 3\}\}\]
Sei $P$ eine Menge bestehend aus Mengen. Dann steht
\[\bigcup_{Y \in P} Y = \{y: \text{ es gibt } Y \in P \text{ so dass } y \in Y\}\]
für die (möglicherweise unendliche) Vereinigung aller Mengen in $P$. \\
\newline
Partitionen: \\
Sei $X$ eine Menge. Eine Partition von $X$ ist eine Teilmenge $P \in \mathcal{P}(X) \setminus \{\emptyset\}$ sodass
\begin{itemize}
    \item für alle $Y,Z \in P$ mit $Y \neq Z, Y \cap Z = \emptyset$ ($Y$ und $Z$ sind disjunkt).
    \item $\bigcup_{Y \in P} Y = X$. 
\end{itemize}
Definierte Mengen:
\begin{itemize}
    \item Leere Menge: $\emptyset = \{\}$
    \item Natürliche Zahlen: $\mathbb{N} = \{1, 2, 3, 4, 5, 6, 7, 8, 9, 10, \dots\}$ ($0 \notin \mathbb{N}$)
    \item Ganze Zahlen: $\mathbb{Z} = \{0, 1, -1, 2, -2, 3, -3, 4, -4, \dots\}$
    \item Rationale Zahlen: $\mathbb{Q} = \left\{\frac{p}{q} : p,q \in \mathbb{Z}, q \neq 0\right\}$
    \item Relle Zahlen: $\mathbb{R}$, Menge aller \textbf{rellen Zahlen}, die man \textbf{nicht abzählen} kann
\end{itemize}
Es gilt: $\mathbb{N} \in \mathbb{Z} \in \mathbb{Q} \in \mathbb{R}$

\section{Boolische Algebra}
Als eine \textbf{Boolische Algebra} bezeichnet man eine Menge $V = \{a,b,c,\dots\}$, auf der zwei zweistellige Operationen $\oplus$ und $\otimes$ derart definiert sind, dass durch ihre Anwendung auf Elemente aus $V$ wieder Elemente aus V enstehen (Abgeschlossenheit).\\
\newline
\textbf{Abgeschlossenheit:} für alle $a,b \in V$ gilt: \[a \otimes b \in V \]\[ a \oplus b \in V\]
\newpage
Zudem müssen die vier \textbf{Huntingtonischen Axiome} gelten:
\begin{itemize}
    \item H1: \textbf{Kommutativgesetz}
        \[a \otimes b = b \otimes a\]
        \[a \oplus b = b \oplus a\]
    \item H2: \textbf{Distributivgesetz}
        \[a \otimes (b \oplus c) = (a \otimes b) \oplus (a \otimes c)\]
        \[a \oplus (b \otimes c) = (a \oplus b) \otimes (a \oplus c)\]
    \item H3: \textbf{Neutrale Elemente} \\ Es existieren zwei Elemente $e,n \in V$, so dass gilt: \\
        \begin{align*}
            a \otimes e &= a \quad &(\text{$e$ wird \textbf{Einselelement} genannt})\\
            a \oplus n &= a \quad &(\text{$n$ wird \textbf{Nullelement} genannt})
        \end{align*}
    \item H4: \textbf{Inverse Elemente} \\ Für jedes $a \in V$ existiert ein Element $a^{-1} \in V$, so dass gilt: \\
        \[a \otimes a^{-1} = n\]
        \[a \oplus a^{-1} = e\]
\end{itemize}
\subsection{Schaltalgebra}
Die Schaltalgebra $(\{0,1\}, \land, \lor)$ ist eine spezielle boolische Algebra. 0 und 1 können als die logischen Werte wahr und falsch interpretieren. \\
Es gelten die vier Huntingtonischen Axiome: \\
\newline
\begin{tabular}{l l l}
    (H1) & Kommutativgesetz & $a \lor b = b \lor a$ \\
    & & $a \land b = b \land a$ \\
    (H2) & Distributivgesetz & $a \land (b \lor c) = (a \land b) \lor (a \land c)$ \\
    & & $a \lor (b \land c) = (a \lor b) \land (a \lor c)$ \\
    (H3) & Neutrale Elemente & $a \land 1 = a$ \\
    & & $a \lor 0 = a$ \\
    (H4) & Invere Elemente & $a \land \neg a = 0$ \\
    & & $a \lor \neg a = 1$ \\
\end{tabular}
\subsection{Boolische Funktionen}
Eine Funktion $f: \{0,1\}^n \rightarrow \{0,1\}$ wird als boolische Funktion bezeichnet.
\subsection{Boolischer Ausdruck}
Sei $V = \{x_1, x_2, \dots, x_n\}$ eine Menge boolischer Variablen. Dann ist die Menge der boolischen Ausdrücke wie folgt definiert: \\
\begin{itemize}
    \item $0,1,x_i$ sind boolische Ausdrücke.
    \item Ist $\Phi$ ein boolischer Ausdruck, dann ist auch $\neg \Phi$ ein boolischer Ausdruck.
    \item Wenn $\Phi$ und $\Psi$ boolische Ausdrücke sind, dann sind auch $\Phi \land \Psi$ und $\Phi \lor \Psi$ boolische Ausdrücke.
    \item Ist $\Phi$ ein boolischer Ausdruck, dann ist auch $(\Phi)$ ein boolischer Ausdruck.
\end{itemize}

\section{Relationen}
Eine \textbf{(binäre) Relation} zwischen zwei Mengen $X$ und $Y$ ist eine Teilmenge
\[\text{R} \subset X \times Y\]
Im Falle $X = Y$ sprechen wir von einer Relation auf X. \\
$x \in X$ steht in Relation zu $y \in Y$ genau dann wenn $(x, y) \in \text{R}$. \\
Auch geschrieben: $x\:\text{R}\:y$ oder $x \sim_{\text{R}} y$ für $(x, y) \in \text{R}$ und $x\:\cancel{\text{R}}\:y$ oder $x \cancel{\sim}_{\text{R}} y$ für $(x,y) \notin \text{R}$. \\ \newline
Seien $X, Y$ und $Z$ Mengen und R $\subset X \times Y$, S $\subset Y \times X$ Relationen.
\begin{itemize}
    \item Die zu R \textbf{inverse Relation} ist \[\text{R}^{-1} = \{(y,x) \in Y \times X : (x, y \in \text{R})\}\]
    \item Die Verkettung von R und S ist \[\text{S} \circ \text{R} = \{(x,z) \in X \times Z : \text{es gibt } y \in Y \text{ mit } (x,y) \in \text{R und } (y,z) \in \text{S}\}\]
\end{itemize}
Eine binäre Relation R auf der Menger $X$ heißt:
\begin{itemize}
    \item \textbf{relfexiv}, wenn $x\:\text{R}\:x$ für alle $x \in X$.
    \item \textbf{symmetrisch}, wenn für alle $x, y \in X$ aus $x\:\text{R}\:y$ stets $y\:\text{R}\:x$ folgt.
    \item \textbf{antisymmetrische}, wenn für alle $x, y \in X$ aus $x\:\text{R}\:y$ und $y\:\text{R}\:x$ stets $x = y$ folgt.
    \item \textbf{asymmetrisch}, wenn für alle $x, y \in X$ aus $x\:\text{R}\:y$ stets $y\:\cancel{R}\:x$ folgt.
    \item \textbf{transitiv}, wenn für alle $x, y, z \in X$ aus $x\:\text{R}\:y$ und $y\:\text{R}\:z$ stets $x\:\text{R}\:z$ folgt.
\end{itemize}
\subsection{Äquivalenzrelationen}
Sei $X$ eine nicht leere Menge. Eine Relation R auf $X$ die relfexiv, symmetrisch und transitiv ist, heißt \textbf{Äquivalenzrelationen}. Für $x \in X$ nennt man die Menge
\[[x]\sim_{\text{R}} = \{y \in X : x\:\text{R}\:y\}\]
die \textbf{Äquivalenzklasse} von $x$. Man nennt $x$ und jedes andere Element aus $[x]\sim_{\text{R}}$ einen \textbf{Vertreter} oder \textbf{Repräsentanten} dieser Äquivalenzklasse. \\
\newline
Sei $X$ eine Menge und $\sim$ eine Äquivalenzrelation auf $X$. Ein \textbf{Vertretersystem} ist eine Teilmenge von $X$, die für jede Äquivalenzklasse genau ein Element enthält.
\subsection{Ordnungsrelationen}
Sei $X$ eine Menge. Eine \textbf{Ordnung} auf $X$ ist eine reflexive, antisymmetrische und transitive Relation. Eine \textbf{strikte Ordnung} auf X ist eine asymmetrisch und transitive Relation. Wir nennen eine (strikte) Ordnung $\preceq$ \textbf{total}, wenn je zwei Elemente vergleichbar sind: 
\[\text{für alle } x,y \in X \text{ gilt } x \preceq y \text{ oder } y \preceq x\]
Ansonsten nennen wir sie \textbf{partiell}.
\subsection{Hüllen}
Sei R eine Relation auf der Menge $X$. Wir definieren:
\begin{itemize}
    \item Für $n \in \mathbb{N}_0$ \[R^n = \begin{cases}
        \text{I}_X & n = 0 \\
        \text{R} \circ \text{R}^{n-1} & n \geq 1 \\
    \end{cases}\] Es gilt, dass R$^1 = $ R
    \item Die \textbf{transitive Hülle} von R ist \[\text{R}_{\text{trans}} = \bigcup_{n\in\mathbb{N}} \text{R}^n\]
    \item Die \textbf{reflexive Hülle} von R ist \[\text{R}_{\text{refl}} = \text{R} \cup \text{I}_X\]
    \item Die \textbf{symmetrische Hülle} von R ist \[\text{R}_{\text{sym}} = \text{R} \cup \text{R}^{-1}\]
\end{itemize}
\newpage
\section{Abbildungen}
Eine \textbf{Abbildung} $f: X \to Y$ besteht aus:
\begin{itemize}
    \item einer Menge X, der \textbf{Definitionsbereich} von $f$;
    \item einer Menge Y, der \textbf{Wertebereich} von $f$;
    \item einer \textbf{Vorschrift}, die jedem $x \in X$ eindeutig ein $y \in Y$ zuordnet.
\end{itemize}
Notation: $f: X \to Y, x \mapsto f(x)$ \\ 
\newline
\hbox{Seien $X, Y$ Mengen, $f: X \to Y$ eine Abbildung und $x \in X, y \in Y$ sodass $f(x)=y$.} \\
Dann ist $y$ das \textbf{Bild} von $x$ und $x$ ein \textbf{Urbild} von $y$.\\
Für eine Teilmenge $X_0 \subset X$ ist
\[f(X_0) := \{y \in Y: \text{ es gibt } x \in X_0 \text{, sodass } f(x)=y\} \subset Y\]
das \textbf{Bild} von $X_0$ und für eine Teilmenge $Y_0 \subset Y$ ist
\[f^{-1}(Y_0) := \{x \in X : f(x) \in Y_0\} \subset X\]
das \textbf{Urbild} von $Y_0$. \\ 
\newline
Seien $X$ und $Y$ Mengen und $f: X \to Y$ eine Abbildung.
\begin{itemize}
    \item[] $f$ ist \textbf{injektiv} falls aus $x_1,x_2 \in X$ mit $f(x_1) = f(x_2)$ stets $x_1 = x_2$ folgt.
        \subitem{''zu jedem y höchstens 1 x-Wert''}
    \item[] $f$ ist \textbf{surjektiv} falls es für jedes $y \in Y$, ein $x \in X$ existiert so dass $f(x) = y$.
        \subitem{''zu jedem y mindestens 1 x-Wert''}
    \item[] $f$ ist \textbf{bijektiv} falls $f$ injektiv und surjektiv ist.
\end{itemize}
Seien $X, Y, Z$ Mengen und $f: X \to Y$ und $g: Y \to Z$ Abbildungen.\\
Die \textbf{Komposition} oder \textbf{Verknüpfung} von $f$ und $g$ ist die\\
Abbildung $g \circ f: X \to Z$, definiert durch $(g \circ f)(x)=g(f(x))$.

\section{Zahlensysteme}
\subsection{Binärsystem}
\begin{itemize}[leftmargin=*]
    \item[] Eine Binärzahl $b$ mit $n + 1$ Stellen hat die Form $b_n\dots b_1 b_2$ mit $b_i \in \{0,1\}$.
    \item[] Sie entspricht der Dezimalzahl $d$ mit $d = b_n \cdot 2^n + \cdots + b_1 \cdot 2^1 + b_0 \cdot 2^0$ 
    \item[] Beispiel: $1101_2 = 1 \cdot 2^3 + ^\cdot 2^2 + 0 \cdot 2^1 + 1 \cdot 2^0 = 13_{10}$
\end{itemize}
\subsubsection{Carry-Flag}
Wenn bei einer Addition oder Subtraktion ein \textbf{Übertrag in der höchsten Stelle} auftritt, wird die Carry-Flag gesetzt.
Dieser kann von nachfolgenden Befehlen aufgerufen werden.
\subsubsection{Zweierkomplement}
\begin{itemize}[leftmargin=*]
    \item[] Um \textbf{negative Zahlen} darzustellen wird der entsprechende Wert des \textbf{höchsten Bits negiert}.
    \item[] Beispiel bei 4 Bit: $1011_{2c} = 1 \cdot (-2^3) + 0 \cdot 2^2 + 1 \cdot 2^1 + 1 \cdot 2^0 = -5$
    \item[] Um von einer positiven ganzen Zahl zur negativen Zahl (oder umgekehrt) gleichen Betrags zu gelangen werden \textbf{alle Bits invertiert} und \textbf{1 zum Ergebnis addiert}.
\end{itemize}

\subsection{Hexadezimalsystem}
\begin{itemize}[leftmargin=*]
    \item[] Eine Hexadezimalzahl $h$ mit $n + 1$ Stellen hat die Form $h_n \dots h_1 h_0$ mit $h_i \in \{0, 1, 2, 3, 4, 5, 6, 7, 8, 9, A(\widehat{=}10), B(\widehat{=}11), C(\widehat{=}12), D(\widehat{=}13), E(\widehat{=}14), F(\widehat{=}15)\}$.
    \item[] Sie entspricht der Dezimalzahl $d$ mit $d = h_n \cdot 16^n + h_1 \cdot 16^1 + h_0 \cdot 16^0$.
    \item[] Beispiel: $\mathtt{5F}_{16} = 5 \cdot 16^1 + 15 \cdot 16^0 = 95_{10}$
    \item[] 4 Binärziffern lassen sich zu einer Hexadezimalzahl zusammenfassen:\\$\underbrace{1101}_{13_{10}=D_{16}} \underbrace{0011_2}_{3_{16}} = \mathtt{D3_{16}}$
\end{itemize}
\subsection{Oktalsystem}
\begin{itemize}[leftmargin=*]
    \item[] Eine Oktalzahl $o$ mit $n + 1$ Stellen hat die Form $o_n \dots o_1 o_0$ mit $o_i \in \{0, 1, 2, 3, 4, 5, 6, 7\}$
    \item[] Sie entspricht der Dezimalzahl $d$ mit $d = o_n \cdot 8^n + \cdots + o_1 \cdot 8^1 + o_0 \cdot 8^0$.
    \item[] Beispiel: $36_8 = 3 \cdot 8^1 + 6 \cdot 8^0 = 30_{10}$
    \item[] 3 Binärziffern lassen sich zu einer Oktalzahl zusammenfassen:\\$\underbrace{11}_{3_8} \underbrace{010}_{2_8} \underbrace{011_2}_{3_8} = 323_8$
\end{itemize}
\subsection{Festkommazahlen}
Eine Festkommazahl besteht aus einer \textbf{festen Anzahl von Ziffern vor und nach dem Komma}.\\ \newline
\begin{tabular}{c|c|c|c|c|c|c|c|c}
    $2^3$ & $2^2$ & $2^1$ & $2^0$ &  & $2^{-1}$ & $2^{-2}$ & $2^{-3}$ & $2^{-4}$ \\ \hline
    1 & 1 & 0 & 1 & . & 0 & 1 & 0 &1
\end{tabular}
\subsection{Gleitkommazahlen: IEEE 754}
3 Formate:
\begin{itemize}
    \item Single Precision: 32 Bit
    \item Double Precision: 64 Bit
    \item Extended Precision: 80 Bit
\end{itemize}
Basiert auf der wissenschaftlichen Notation.
\subsubsection{Aufbau}
\begin{tabular}{@{}l l}
    Single Precision: &
    \begin{tabular}{|c|c|c|}
        \hline
        1 Bit Vorzeichen & 8 Bit Exponent & 23 Bit normalisierte Mantisse\\
        \hline
    \end{tabular} \\
    Double Precision: &
    \begin{tabular}{|c|c|c|}
        \hline
        1 Bit Vorzeichen & 11 Bit Exponent & 52 Bit normalisierte Mantisse\\
        \hline
    \end{tabular}
\end{tabular} \\
\newline
Vorzeichen: $0 = +$; $1 = -$ \\
\newline
Exponent: wird gespeichert, indem man den festen Biaswert (127:SP, 1023:DP) addiert. \\
\newline
Die Mantisse beginnt mit einem ''Hidden Bit'' (immer 1).
\subsubsection{Dezimal zu IEEE 754}
Beispiel: -62.058
\begin{enumerate}
    \item Vorzeichen Bit bestimmen
    \subitem{Vorzeichen Bit = 1}
    \item Zu pur Binär umwandeln
    \subitem{$62.058_{10} = 111110.10010100_2$}
    \item Normalisieren für Mantisse und Exponent (ohne Bias)
    \subitem{$111110.10010100_2 = 1.1111010010100_2 \cdot 2^5$}
    \item Exponent mit Bias bestimmen
    \subitem{$5 + 127 = 132_{10} = 10000100_2$}
    \item Führende 1 der Mantisse abschneiden
    \subitem{$1.1111010010100_2 \rightarrow 1111010010100_2$}
    \item Zusammenfügen
    \subitem{$-62.058_{10} = \underbrace{1}_{\substack{\text{Vorzeichen}\\ \text{Bit}}} \underbrace{10000100}_{\text{Exponent}} \underbrace{1111010010100}_{\text{Mantisse}}$}
\end{enumerate}
\subsubsection{IEEE 754 zu Dezimal}
Beispiel: $01000010011010100000000000000000$
\begin{enumerate}
    \item Vorzeichen bestimmen
    \subitem{Vorzeichen: $+$}
    \item Exponent bestimmen (Bias muss abgezogen werden)
    \subitem{$10000100_2 - 127_{10} = 132_{10} - 127_{10} = 5_{10}$}
    \item Mantisse bestimmen
    \subitem{
        \begin{tabular}{|c|c|c|c|c|c|}
            $\frac{1}{2}$ & $\frac{1}{4}$ & $\frac{1}{8}$ & $\frac{1}{16}$ & $\frac{1}{32}$ & $\frac{1}{64}$ \\ \hline
            1 & 1 & 0 & 1 & 0 & 1
        \end{tabular}
    }
    \subitem{$\frac{1}{2} + \frac{1}{4} + \frac{1}{16} + \frac{1}{64} = \frac{53}{64} = 0.828125$}
    \item 1 zur Mantisse addieren (Hidden Bit) und Vorzeichen einrechnen
    \subitem{1.828125}
    \item Ergebnis berechnen
    \subitem{$1.828125 \cdot 2^5 = 58.5_{10}$}
\end{enumerate}
\section{Fehlererkennung}
\subsection{Redundanzen}
Eine Einheite von $n$ Datenbits und $k$ Redundanzbits nennt man \textbf{Codewort}.\\
Die \textbf{Länge} eines Codeworts ist insgesammt $n+k$.\\
Die Menge aller gültigen Codewörter nennt man \textbf{Code}.
\subsection{Hamming-Distanz}
Die Hamming Distanz zweier Codewörter ist gegeben als die Anzahl der Bitpositionen, in denen sie sich unterscheiden. \\
\indent Beispiel: $11110000$ und $11001100 \implies$ Hamming-Distanz beträgt 4 \\ \newline 
Die Hamming Distanz eines Codes ist die kleinste Hamming-Distanz zweier Codewörter \\
\indent Beispiel: $\{1100, 0011, 1111\} \implies$ Hamming-Distanz beträgt 2 \\ \newline
$c$-Bit Fehler können erkannt werden, wenn die Hamming-Distanz $c+1$ beträgt.
$c$-Bit Fehler können korrigiert werden, wenn die Hamming-Distanz $2c+1$ beträgt.
\subsection{Parität}
Durch Hinzufügen eines \textbf{Paritätsbits} wird ein Code mit Hamming-Distanz 2 erzeugt.
\begin{itemize}
    \item[] Das Paritätsbit wird gesetzt sodass die Gesamtzahl der 1en...
    \item[] ... gerade ist
    \subitem{$\underbrace{00100101}_{Datenbits} \underbrace{1}_{\text{Paritätsbit}}$}
    \item[] ... ungerade ist
    \subitem{$\underbrace{00100101}_{Datenbits} \underbrace{0}_{\text{Paritätsbit}}$}
\end{itemize}
\subsection{Zweidimensionale Parität}
Die zweidimensionale Parität konstruiert einen Code mit Hamming-Distanz 4. \\
Dabei werden $n$ Wörter zu je $n$ Bits in einer $n \times n$-Matrix untereinandergeschrieben und über jede Zeile und jede Spalte je ein Paritätsbit berechnet. \\
\newline
Bei einem 1-Bit-Fehler stimmen die Paritätsbits genau einer Zeile und Spalte nicht.\\
Dann ist die Position des Fehlers klar und er kann korrigiert werden. \\ \newline
\newcolumntype{a}{>{\columncolor{lightgray}}c}
\newcolumntype{b}{>{\columncolor{white}}c}
\begin{tabular}{l l r r}
    fehlerfrei: & \begin{tabular}{c c c c|c}
        0 & 1 & 1 & 0 & 0 \\
        1 & 0 & 0 & 0 & 1 \\
        0 & 1 & 0 & 0 & 1 \\
        0 & 1 & 1 & 1 & 1 \\ \hline
        1 & 1 & 0 & 1 & 1 \\
    \end{tabular} &
    1-Bit-Fehler: & \begin{tabular}{b b a b|b}
        0 & 1 & 1 & 0 & 0 \\
        \cellcolor{lightgray}1 & \cellcolor{lightgray}0 & \textcolor{red}{1} & \cellcolor{lightgray}0 & \cellcolor{lightgray}1\\
        0 & 1 & 0 & 0 & 1 \\
        0 & 1 & 1 & 1 & 1 \\ \hline
        1 & 1 & 0 & 1 & 1 \\
    \end{tabular} 
\end{tabular} \\
Das Bit ganz unten rechts wird zur Paritätsberechnung der Paritätszeile und -spalte genutzt.
\subsection{Hamming-Code}
Ein Hamming-Code mit $n$ \textbf{Redundanzbits} hat maximal $2^n - 1$ Bits und maximal $2^n -1 - n$ Datenbits (mit $n \in \mathbb{N}$)\\
Die Bits des Codewortes werden, beginnend bei 1, durchnummeriert. \\
Das $i$-te \textbf{Prüfbit}(auch Redundanzbit) steht im Codewort an Position $2^i$ ($\implies 1,2,4,8,\dots$) \\
Beispiel (gerade Parität): \begin{tabular}{c|c c}
    Position & Bits des Codewortes & \\ \hline
    \cellcolor{lightgray}$1_{10}$ & \cellcolor{lightgray}0 & Prüfbit \\
    \cellcolor{lightgray}$2_{10}$ & \cellcolor{lightgray}1 & Prüfbit \\
    $3_{10}$ & 0 & \\
    \cellcolor{lightgray}$4_{10}$ & \cellcolor{lightgray}0 & Prüfbit \\
    $5_{10}$ & 1 & \\
    $6_{10}$ & 0 & \\
    $7_{10}$ & 1 & \\
\end{tabular} \\ Gespeichertes Datenwort: $0101$ \\
\subsubsection{Berechnung der Prüfbits}
Jedes Prüfbit ist ein Paritätsbit über eine eindeutige Menge von Bits. \\
Das $i$-te Prüfbit an Position $2^i$ wird über alle Stellen aus dem Codewort berechnet, für die in der Binärdarstellung für $2^i$ das Bit auf der jeweiligen Position auf 1 gesetzt ist. \\
Beispiel: 0. Prüfbit an Stelle $2^2 = 4_{10} = 100_2 \implies$ jedes Bit aus dem Codewort, in dessen Binärdarstellung der Position das Bit auf Position $2^2$ gesetzt ist, wird zur Berechnung des Prüfbits verwendet. \\ 

\newpage
\section{Summenzeichen und Produktzeichen}
\subsection{Summenzeichen}
Seien $m, n \in \mathbb{Z}$ mit $m \leq n$. Die Summen der Zahlen $a_m, a_{m+1}, \dots, a_n$ wird folgendermaßen bezeichnet:
\[\sum_{i=m}^{n}a_i = a_m + a_{m+1} + \ldots + a_n\]
Dabei gilt: $i \: \widehat{=} \textbf{ Summationsindex}; \quad m/n \: \widehat{=} \textbf{ untere/obere Summationsgrenze}$.
Rechenregeln:
\[\sum_{i=m}^{n}c \cdot a_i = c \cdot \sum_{i=m}^{n}a_i\]
\[\sum_{i=m}^{n}(a_i + b_1) = \sum_{i=m}^{n}a_i + \sum_{i=m}^{n}b_i\]
\subsection{Produktzeichen}
Seien $m, n \in \mathbb{Z}$ mit $m \leq n$. Das Produkt der Zahlen $a_m, a_{m+1},\ldots, a_n$ wird folgendermaßen bezeichnet:
\[\prod_{i=m}^{n}a_i = a_m \cdot a_{m+1} \cdot \ldots \cdot a_n\]
Dabei gilt: $i \: \widehat{=} \textbf{ Laufindex}; \quad m/n \: \widehat{=} \textbf{ untere/obere Grenze}$.

\newpage
\section{Rechenregeln}
\subsection{Bruchregeln}
\begin{center}
    \scalebox{1.3}{
        \begin{tabular}{c c}
            $\frac{a}{b} = \frac{a \cdot c}{b \cdot c}$ & $\frac{a}{b} + \frac{c}{b} = \frac{a + c}{b}$ \vspace{1ex}\\
            $\frac{a}{b} \cdot \frac{c}{d} = \frac{ac}{bd}$ & $\frac{a}{b} : \frac{c}{d} = \frac{a}{b} \cdot \frac{d}{c} = \frac{ad}{bc}$
        \end{tabular}
    }
\end{center}

\subsection{Potenzgesetze}
\begin{center}
\scalebox{1.1}{
    \begin{tabular}{c c}
        $a^n \cdot a^m = a^{n+m}$ & $a^n \cdot b^n = (a \cdot b)^n$ \vspace{1ex}\\ 
        ${\left(a^n\right)}^m = {\left(a^m\right)}^n = a^{n \cdot m}$ & $a^{-n} = \frac{1}{a^n}, a \neq 0$ \vspace{1ex}\\
        $a^0 = 1, a \in \mathbb{R}$
    \end{tabular}
}
\end{center}

\subsection{Wurzelgesetze}
\begin{center}
    \scalebox{1.1}{
    \begin{tabular}{c c}
        $\sqrt[n]{a^n}= a$ &  $\left(\sqrt[n]{a}\right)^n = a$ \vspace{1ex}\\
        $\sqrt[n]{a \cdot b} = \sqrt[n]{a} \cdot \sqrt[n]{b}$ & $\sqrt[n]{\frac{a}{b}} = \frac{\sqrt[n]{a}}{\sqrt[n]{b}}, \hspace{1ex} b \neq 0$ \vspace{1ex}\\
        $a^\frac{1}{n} = \sqrt[n]{a}$ & $\hspace{1ex} a^{-\frac{1}{n}} = \frac{1}{\sqrt[n]{a}}, a > 0$
    \end{tabular}
    }
\end{center}

\subsection{Logarithmengesetze}
\begin{center}
    \scalebox{1.1}{
        \begin{tabular}{c c}
            $\log 1 = 0$ & $\hspace{1ex} \log e = 1$ \vspace{1ex} \\
            $a^x = b \Leftrightarrow x = \log_a(b)$ & $\log(a^x) = x \log a$ \vspace{1ex} \\
            $\log(x \cdot y) = \log x + \log y$ & $\log \left(\frac{x}{y}\right) = \log x - \log y$ \\
        \end{tabular}
        }
\end{center}

\newpage
\section{Trigonometrie}
\subsection{Bogenmaß}
Der Bogenmaß ist die Länge des Kreisbogens des Einheitskreises und gibt den Betrag des Winkels an. Der Umfang des Einheitskreises beträgt $2\pi$.
\begin{center}
    \scalebox{1.15}{
    \begin{tabular}{c||c|c|c|c|c|c|c|c}
        Bogenmaß & 0 & $\frac{\pi}{6}$ & $\frac{\pi}{4}$ & $\frac{\pi}{3}$ & $\frac{\pi}{2}$ & $\pi$ & $\frac{3\pi}{2}$ & $2\pi$\\
        \hline
        Gradmaß & 0° & 30° & 45° & 60° & 90° & 180° & 270° & 360°\\
    \end{tabular}
    }
\end{center}
Umwandlung von Winkel $\alpha$ von Gradmaß zu Bogenmaß: Bogenmaß $= \alpha \frac{\pi}{\text{180°}}$
Umwandlung von Winkel $\alpha$ von Bogenmaß zu Gradmaß: Gradmaß $= \alpha \frac{\text{180°}}{\pi}$

\end{document}