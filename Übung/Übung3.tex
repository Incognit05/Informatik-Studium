\documentclass{article}
\usepackage[utf8]{inputenc}
\usepackage{amsmath, amssymb, amsthm}

\title{Übung 3}
\author{Pascal Diller, Timo Rieke}

\setcounter{secnumdepth}{0}

\begin{document}

\maketitle

\section{1}
\subsection{(i)}
Für $m$ gerade: \\
\[m = 2k \text{ für } k \in \mathbb{N}\]
\[f(m) = \frac{2k}{2} = k\]
$\implies$ $k$ kann alle natürlichen Zahlen annehmen. Somit sind die Werte bei $f(m)$ für $m$ gerade alle in $\mathbb{N}$. \\
\newline
Für $m$ gerade: \\
\[m = 2k + 1 \text{ für } k \in \mathbb{N}\]
\[f(m) = \frac{(2k + 1) - 1}{2} = \frac{2k}{2} = k\]
$\implies k$ kann wieder $\mathbb{N}$ annehmen. \\
\newline
$\implies$ Da $f(m)$ alle natürlichen Zahlen abbildet ist $f$ surjektiv auf $\mathbb{N}$, jedoch nicht auf $\mathbb{Z}$, da keine negativen Zahlen erreicht werden.
\subsection{(ii)}


\section{2}
\subsection{(i)}
\[A: \forall x \in \mathbb{N}, x > 1 \land (d|x \to d = 1 \lor d = x)\]
\subsection{(ii)}
$\neg A$: Es gibt mindestens eine Natürliche Zahl, die keine Primzahl ist.
\[\neg A: \exists x \in \mathbb{N}, d|x \to d \neq 1 \land d \neq x\]
Die Aussage $A$ ist falsch, da es auch Natürliche Zahlen gibt, die nicht nur 1 und sich selber als Teiler haben.

\section{3}
\subsection{(ii)}
\[\left( ((P \to R) \to (R \to Q)) \to (P \to Q) \right) \mapsto 0\]

\end{document}