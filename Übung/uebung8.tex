\documentclass{article}
\usepackage[utf8]{inputenc}
\usepackage{amsmath, amssymb, amsthm}
\usepackage{graphicx} % Required for inserting images

\title{Mathe Übungsblatt 8}
\author{Pascal Diller, Timo Rieke}
\date{October 2024}

\setcounter{secnumdepth}{0}

\begin{document}
\maketitle

\section{Aufgabe 1}
\subsection{(i)}
Linke Seite:
\[
\begin{bmatrix}
-1 & 2 & 0 \\
5 & 3 & -2 \\
0 & 4 & -2 \\
\end{bmatrix} \cdot 
\begin{bmatrix}
    3 & 0 & 0\\
    0 & 3 & 0 \\
    0 & 0 &3\\
\end{bmatrix} = 
\begin{bmatrix}
    -1*3+5*0+0*0 & 2*3+3*0+4*0 & 0*3-2*0-2*0\\
    -1*0+5*3+0*0 & 2*0+3*3+4*0 & 0*0-2*3-2*0\\
    -1*0+5*0+0*3 & 2*0+3*0+4*3 & 0*0-2*0-2*3\\
\end{bmatrix}\]  \[ =
\begin{bmatrix}
    -3 & 6 & 0\\
    15 & 9 & -6\\
    0 & 12 & -6\\
\end{bmatrix}
\]
Rechte Seite:
\[
\begin{bmatrix}
    3 & 0 & 0\\
    0 & 3 & 0 \\
    0 & 0 &3\\
\end{bmatrix} \cdot 
\begin{bmatrix}
-1 & 2 & 0 \\
5 & 3 & -2 \\
0 & 4 & -2 \\
\end{bmatrix}\]  \[ = 
\begin{bmatrix}
3*(-1) + 0*5 +0*0& 0*(-1) + 3*5 +0*0 & 0*(-1) + 0*5 +3*0\\
3*5+0*3+0+(-2)&0*5+3*3+0+(-2)& 0*5+0*3+3+(-2) \\
3*0+0*4+0*(-2)& 0*0+3*4+0*(-2) & 0*0+0*4+3*(-2)\\
\end{bmatrix} =
\begin{bmatrix}
    -3 & 6 & 0\\
    15 & 9 & -6\\
    0 & 12 & -6\\
\end{bmatrix}
\]

\subsection{(ii)}
$A \cdot \lambda$:
\[
\begin{bmatrix}
    a_{1,1} & a_{1,2} & a_{1,3}\\
    a_{2,1} & a_{2,2} & a_{2,3} \\
    a_{3,1} & a_{3,2} &a_{3,3}\\
\end{bmatrix} \cdot 
\begin{bmatrix}
\lambda & 0 & 0 \\
0 & \lambda & 0 \\
0 & 0 & \lambda \\
\end{bmatrix}\]  \[ = 
\begin{bmatrix}
    a_{1,1}*\lambda + a_{1,2}*0 + a_{1,3}*0 & a_{2,1}*\lambda + a_{2,2}*0 + a_{2,3}*0 & a_{3,1}*\lambda + a_{3,2}*0 + a_{3,3}*0\\
    a_{1,1}*0 + a_{1,2}*\lambda + a_{1,3}*0& a_{2,1}*0 + a_{2,2}*\lambda + a_{2,3}*0& a_{3,1}*0 + a_{3,2}*\lambda + a_{3,3}*0\\
    a_{1,1}*0 + a_{1,2}*0 + a_{1,3}*\lambda& a_{2,1}*0 + a_{2,2}*0 + a_{2,3}*\lambda & a_{3,1}*0 + a_{3,2}*0 + a_{3,3}*\lambda\\
\end{bmatrix}\]\[ =
\begin{bmatrix}
 a_{1,1}*\lambda& a_{2,1}*\lambda& a_{3,1}*\lambda\\
a_{1,2}*\lambda& a_{2,2}*\lambda& a_{3,2}*\lambda\\
 a_{1,3}*\lambda&a_{2,3}*\lambda & a_{3,3}*\lambda\\
\end{bmatrix}
\]

$\lambda \cdot A$:
\[
\begin{bmatrix}
\lambda & 0 & 0 \\
0 & \lambda & 0 \\
0 & 0 & \lambda \\
\end{bmatrix}
 \cdot 
\begin{bmatrix}
    a_{1,1} & a_{1,2} & a_{1,3}\\
    a_{2,1} & a_{2,2} & a_{2,3} \\
    a_{3,1} & a_{3,2} &a_{3,3}\\
\end{bmatrix}\]  \[ = 
\begin{bmatrix}
    \lambda*a_{1,1}+0*a_{1,2}+0*a_{1,3}&0*a_{1,1}+\lambda*a_{1,2}+0*a_{1,3} &0*a_{1,1}+0*a_{1,2}+\lambda*a_{1,3} \\
    \lambda*a_{2,1}+0*a_{2,2}+0*a_{2,3}&0*a_{2,1}+\lambda*a_{2,2}+0*a_{2,3} &0*a_{2,1}+0*a_{2,2}+\lambda*a_{2,3} \\
    \lambda*a_{3,1}+0*a_{3,2}+0*a_{3,3}&0*a_{3,1}+\lambda*a_{3,2}+0*a_{3,3} &0*a_{3,1}+0*a_{3,2}+\lambda*a_{3,3} \\
\end{bmatrix}\]\[ =
\begin{bmatrix}
    \lambda*a_{1,1}&\lambda*a_{1,2} &\lambda*a_{1,3} \\
    \lambda*a_{2,1}&\lambda*a_{2,2}&\lambda*a_{2,3} \\
    \lambda*a_{3,1}&\lambda*a_{3,2}&\lambda*a_{3,3} \\
\end{bmatrix} =
\begin{bmatrix}
 a_{1,1}*\lambda& a_{2,1}*\lambda& a_{3,1}*\lambda\\
a_{1,2}*\lambda& a_{2,2}*\lambda& a_{3,2}*\lambda\\
 a_{1,3}*\lambda&a_{2,3}*\lambda & a_{3,3}*\lambda\\
\end{bmatrix}
\]

\subsection{(iii)}

\section{Aufgabe 2}
\subsection{(i)}
\[T(1,2,3,4)=(2\cdot1+4\cdot3, -2\cdot2-4\cdot4) =(14,-20)\]
$T^{-1}(\{(2,4)\})$:
\[T(x_1,x_2,x_3,x_4)=(2,4)\]
Es ergeben sich die beiden Gleichungen:
\[2x_1+4x_3=2, -2x_2-4x_4=4\]
Das Gleichungssystem hat unendlich viele Lösungen, deswegen gibt es frei wählbare Variablen. Für $x_3$ und $x_4$ frei wählbar:
\[2x_1+4x_3=2 \ \ \ |:2\] 
\[x_1+2x_3=1\ \ \ |-2x_3\]
\[x_1=1-2x_3\]

\[-2x_2-4x_4=4\ \ \ |:(-2)\]
\[x_2+2x_4 = -2\ \ \ |-2x_4\]
\[x_2= -2-2x_4\]
Somit ergibt sich:
\[T^{-1}(\{(2,4)\})=\{(1-2x_3,-2-2x_4,x_3,x_3)|x_3,x_4\in \mathbb{R}\}\]

\subsection{(ii)}
\(T(u+v)=T(u)+T(v)\):\\
Sei $u=(u_1,u_2,u_3,u_4)$ und $v=(v_1,v_2,v_3,v_4)$
\[T(u+v)=(2(u_1+v_1)+4(u_3+v_3),-2(u_2+v_2)-4(u_4+v_4)) \]
\[=((2u_1 + 2v_1) + (4u_3 + 4v_3), (-2u_2 - 2v_2) + (-4u_4 - 4v_4)).\]
\[=( (2u_1 + 4u_3) + (2v_1 + 4v_3), (-2u_2 - 4u_4) + (-2v_2 - 4v_4)=T(u)+T(v)\]
\newline
\(T(au)=aT(u)\): \\
Sei \(a \in \mathbb{R}\) und \(u=(u_1,u_2,u_3,u_4)\)
\[T(au)=(2(au_1)+4(au_3),-2(au_2)-4(au_4))\]
\[=a(2u_1+4u_3,-2u_2-4u_4) = aT(u)\]
\newline
Da $T(u+v)=T(u)+T(v)$ und $T(au)=aT(u)$ gilt, ist $T$ linear.

\subsection{(iii)}
Gegeben ist die Abbildung $T: \mathbb{R}^4 \to \mathbb{R}^2, (x_1, x_2, x_3, x_4) \mapsto (2x_1 + 4x_3, -2x_2 - 4x_4)$.
Gesucht ist die $2 \times 4$-Matrix $A$ sodass $T_A = T$. \\
Es gilt: \[A = \left(\begin{matrix}
    T_A(e_1) & T_A(e_2) & T_A(e_3) & T_A(e_4)
\end{matrix}\right)\]
Da $T = T_A$:
\begin{align*}
    T_A(e_1) &= T(1, 0, 0 ,0) = (2, 0) \\
    T_A(e_2) &= T(0, 1, 0 ,0) = (0, -2) \\
    T_A(e_3) &= T(0, 0, 1 ,0) = (4, 0) \\
    T_A(e_4) &= T(0, 0, 0 ,1) = (0, -4) \\
\end{align*}
\[A = \left(\begin{matrix}
    2 & 0 & 4  & 0 \\
    0 & -2 & 0 & -4 \\
\end{matrix}\right)\]

\section{Aufgabe 3}
\subsection{(i)}
\subsubsection{(a)}
$\text{ref}_G^2$ ist eine lineare Abbildung, da $\text{ref}_G$ eine lineare Abbildung ist, und die Verkettung zweier linearer Abbildungen ebenfalls eine lineare Abbildung ist.
\subsubsection{(b)}
\[\text{ref}_G = \frac{1}{x^2 + y^2} \left(\begin{matrix}
    x^2-y^2 & 2xy \\
    2xy & y^2-x^2 \\
\end{matrix}\right)\]
\[\text{ref}_G^2 = \frac{1}{x^2 + y^2} \left(\begin{matrix}
    x^2-y^2 & 2xy \\
    2xy & y^2-x^2 \\
\end{matrix}\right) \frac{1}{x^2+y^2} \left(\begin{matrix}
    x^2-y^2 & 2xy \\
    2xy & y^2-x^2 \\
\end{matrix}\right)\]
\[= \frac{1}{x^4 + 2x^2y^2+y^4}\left(\begin{matrix}
    x^4-2x^2y^2+y^4+4x^2y^2 & 2x^3y-2xy^3+2xy^3-2x^3y \\
    2x^3y-2xy^3+2xy^3-2x^3y & 4x^2y^2+y^4-2x^2y^2+x^4 \\
\end{matrix}\right)\]
\[= \frac{1}{x^4 + 2x^2y^2+y^4}\left(\begin{matrix}
    x^4+2x^2y^2+y^4 & 0 \\
    0 & x^4+2x^2y^2+y^4 \\
\end{matrix}\right)\]
\[= \left(\begin{matrix}
    1 & 0 \\
    0 & 1 \\
\end{matrix}\right)\]
Somit ist gezeigt, dass $\text{ref}_G^2 = \text{Id}_{\mathbb{R}^2}$
\subsection{(ii)}
\[R_\alpha \cdot R_\beta = \left(\begin{matrix}
    \cos(\alpha) & -\sin(\alpha) \\
    \sin(\alpha) & \cos(\alpha) \\
\end{matrix}\right) \cdot \left(\begin{matrix}
    \cos(\beta) & -\sin(\beta) \\
    \sin(\beta) & \cos(\beta) \\
\end{matrix}\right)\]
\[= \left(\begin{matrix}
    \cos(\alpha)\cos(\beta)-\sin(\alpha)\sin(\beta) & -\cos(\alpha)\sin(\beta)-\sin(\alpha)\cos(\beta) \\
    \sin(\alpha)\cos(\beta)+\cos(\alpha)\sin(\beta) & -\sin(\alpha)\sin(\beta)+\cos(\alpha)\cos(\beta) \\
\end{matrix}\right)\]
\[= \left(\begin{matrix}
    \cos(\beta)\cos(\alpha)-\sin(\beta)\sin(\alpha) & -\cos(\beta)\sin(\alpha)-\sin(\beta)\cos(\alpha) \\
    \sin(\beta)\cos(\alpha)+\cos(\beta)\sin(\alpha) & -\sin(\beta)\sin(\alpha)+\cos(\beta)\cos(\alpha) \\
\end{matrix}\right)\]
\[=\left(\begin{matrix}
    \cos(\beta) & -\sin(\beta) \\
    \sin(\beta) & \cos(\beta) \\
\end{matrix}\right) \cdot \left(\begin{matrix}
    \cos(\alpha) & -\sin(\alpha) \\
    \sin(\alpha) & \cos(\alpha) \\
\end{matrix}\right) = R_\beta = R_\alpha\]
\[= \left(\begin{matrix}
    \cos(\alpha)\cos(\beta)-\sin(\alpha)\sin(\beta) & -(\sin(\alpha)\cos(\beta)+\cos(\alpha)\sin(\beta)) \\
    \sin(\alpha)\cos(\beta)+\cos(\alpha)\sin(\beta) & \cos(\alpha)\cos(\beta)-\sin(\alpha)\sin(\beta) \\
\end{matrix}\right)\]
\[=\left(\begin{matrix}
    \cos(\alpha + \beta) & -\sin(\alpha + \beta) \\
    \sin(\alpha + \beta) & \cos(\alpha + \beta) \\
\end{matrix}\right) = R_{\alpha + \beta}\]
\end{document}
