\documentclass{article}
\usepackage[utf8]{inputenc}
\usepackage{amsmath, amssymb, amsthm}

\title{Übung 6}
\author{Pascal Diller, Timo Rieke}

\setcounter{secnumdepth}{0}

\begin{document}

\maketitle

\section{Aufgabe 1}
Zu zeigen: $cos(2x) = 1 - 2 sin^2 x$ 
\begin{center}
    $cos(2x) = cos(x+x)$ \\
    Additionstheorem: $cos(x+x) = cos x * cosx - sinx* sinx$ \\
    $cosx*cosx-sinx*sinx = cos^2x - sin^2x$ \\
    Pythagoras: $cos^2x+sin^2x=1 \Longleftrightarrow cos^2x = 1-sin^2x$ \\
    Einsetzen von $cos^2x = 1-sin^2x$ in die Formel $cos^2x - sin^2x$:\\
    $(1-sin^2x) - sin^2x = 1- 2sin^2x$\\
    $\Longrightarrow cos(2x) = 1-2sin^2x$ 
\end{center}

\section{Aufgabe 2}
\subsection{(i)}
Seien $f$ und $g$ zwei ungerade Funktionen. Somit gilt: $f(-x) = -f(x)$ und $g(-x) = -g(x)$. Zu zeigen: $(f \cdot g)(-x) = (f \cdot g)(x)$
\[(f \cdot g)(-x) = f(-x) \cdot g(-x) = -f(x) \cdot (-g(x)) = f(x) \cdot g(x) = (f \cdot g)(x)\]
Somit ist gezeigt, dass das Produkt zweier ungeraden Funktionen gerade ist.

\subsection{(ii)}
Sei $f$ eine gerade Funktion ($f(-x)=f(x)$) und $g$ eine ungerade Funktion ($g(-x)=-g(x)$).
Zu zeigen: $(f \cdot g)(-x) = -(f \cdot g)(x)$
\[(f \cdot g)(-x) = f(-x) \cdot g(-x) = f(x) \cdot (-g(x)) = -(f(x) \cdot g(x)) = -(f \cdot g)(x)\]
Somit ist gezeigt, dass das Produkt einer gerade und einer ungeraden Funktion ungerade ist.

\subsection{(iii)}
Seien $f$ und $g$ zwei gerade Funktionen. Zu zeigen: $(f + g)(-x) = (f + g)(x)$
\[(f + g)(-x) = f(-x) + g(-x) = f(x) + g(x) = (f+g)(x)\]
Somit ist gezeigt, dass die Summe zweier geraden Funktionen auch gerade ist.

\subsection{(iv)}
Sei $\lambda \in \mathbb{N}$ und $f$ eine gerade Funktion. Zu zeigen: $\lambda f(-x) = \lambda f(x)$
\[(\lambda f)(-x) = \lambda f (-x) = \lambda f (x)\]
Somit ist gezeigt, dass für $\lambda$ und die gerade Funktion $f$ das Produkt aus $\lambda f$ gerade ist.

\subsection{(v)}
Zu zeigen: \[f_n: \mathbb{R} \to \mathbb{R}, x \to x^n \begin{cases}
    \text{gerade wenn } n \text{ gerade} \\
    \text{ungerade wenn } n \text{ ungerade}
\end{cases}\]
I.A. \\
Sei $n = 0$. Da $f_0(-x) = (-x)^0 = f_0(x) = x^0 = 1$ ist $f$ gerade. \\
\newline
Sei $n = 1$. Da $f_1(-x) = (-x)^1 = -f_1(x) = -x^1 = -x$ ist $f$ ungerade.

\section{Aufgabe 3}
\subsection{(i) Bestimmen der Nullstellen von $P(x)=x^4-8x^2-9$}
Substitution: $z=x^2$ 
\begin{center}
    $x^4-8x^2-9=z^2-8z-9$
\end{center}
PQ-Formel: 
\begin{center}
    $z_{1,2}=\frac{8}{2} \pm \sqrt{(\frac{-8}{2})^2 + 9} = 4 \pm \sqrt{25} = 4 \pm 5$ \\
    $z_1=4+5=9$ \\
    $z_2=4-5=-1$
\end{center}
Rücksubstitution:
\begin{center}
    $z_1$: $\sqrt{9} = -3 \lor 3$ \\
    $z_2$: $\sqrt{-1} \Longrightarrow$ keine Lösung \\
    $\Longrightarrow$ Nullstellen bei $x_1 = -3$ und $x_2 = 3$
\end{center}

\subsection{(ii) Nullstellen und Grad der Funktion $Q(x)$}
\begin{center}
    $Q(x)=(x-5)^2(x+2)(x^2-4)(3x^2+2)$ 
\end{center}
Nullstellen:  \\
    \begin{enumerate}
        \item $(x-5)^2$: Nullstelle $x=5$ mit Ordnung 2 
        \item $(x+2)$: Nullstelle $x=-2$ mit Ordnung 1 
        \item $(x^2-4) = (x-2)(x+2)$: Nullstellen $x=2$ und $x=-2$, da $x=-2$ erneut vorkommt: \\
         $x=2$ mit Ordnung 1 und $x=-2$ mit Ordnung 2 
        \item $(3x^2+2)$:  $3x^2+2=0 \Longleftrightarrow x^2=-\frac{2}{3} \Longleftrightarrow$ Keine Nullstelle
    \end{enumerate}
    $\Longrightarrow$ Nullstellen: $x=-2$ mit Ordnung 2, $x=2$ mit Ordnung 1 und $x=5$ mit Ordnung 2 \\
\newline 
Grad von $Q(x)$: \\
$2+1+2+2=7$
$\Longrightarrow$ Der Grad von $Q(x)$ entspricht 7.
\end{document}
