\documentclass{article}
\usepackage[utf8]{inputenc}
\usepackage{amsmath, amssymb, amsthm}

\title{Übung 6}
\author{Pascal Diller, Timo Rieke}

\setcounter{secnumdepth}{0}

\begin{document}

\maketitle

\section{Aufgabe 2}
\subsection{(i)}
Seien $f$ und $g$ zwei ungerade Funktionen. Somit gilt: $f(-x) = -f(x)$ und $g(-x) = -g(x)$. Zu zeigen: $(f \cdot g)(-x) = (f \cdot g)(x)$
\[(f \cdot g)(-x) = f(-x) \cdot g(-x) = -f(x) \cdot (-g(x)) = f(x) \cdot g(x) = (f \cdot g)(x)\]
Somit ist gezeigt, dass das Produkt zweier ungeraden Funktionen gerade ist.

\subsection{(ii)}
Sei $f$ eine gerade Funktion ($f(-x)=f(x)$) und $g$ eine ungerade Funktion ($g(-x)=-g(x)$).
Zu zeigen: $(f \cdot g)(-x) = -(f \cdot g)(x)$
\[(f \cdot g)(-x) = f(-x) \cdot g(-x) = f(x) \cdot (-g(x)) = -(f(x) \cdot g(x)) = -(f \cdot g)(x)\]
Somit ist gezeigt, dass das Produkt einer gerade und einer ungeraden Funktion ungerade ist.

\subsection{(iii)}
Seien $f$ und $g$ zwei gerade Funktionen. Zu zeigen: $(f + g)(-x) = (f + g)(x)$
\[(f + g)(-x) = f(-x) + g(-x) = f(x) + g(x) = (f+g)(x)\]
Somit ist gezeigt, dass die Summe zweier geraden Funktionen auch gerade ist.

\subsection{(iv)}
Sei $\lambda \in \mathbb{N}$ und $f$ eine gerade Funktion. Zu zeigen: $\lambda f(-x) = \lambda f(x)$
\[(\lambda f)(-x) = \lambda f (-x) = \lambda f (x)\]
Somit ist gezeigt, dass für $\lambda$ und die gerade Funktion $f$ das Produkt aus $\lambda f$ gerade ist.

\subsection{(v)}
Zu zeigen: \[f_n: \mathbb{R} \to \mathbb{R}, x \to x^n \begin{cases}
    \text{gerade wenn } n \text{ gerade} \\
    \text{ungerade wenn } n \text{ ungerade}
\end{cases}\]
I.A. \\
Sei $n = 0$. Da $f_0(-x) = (-x)^0 = f_0(x) = x^0 = 1$ ist $f$ gerade. \\
\newline
Sei $n = 1$. Da $f_1(-x) = (-x)^1 = -f_1(x) = -x^1 = -x$ ist $f$ ungerade.

\end{document}