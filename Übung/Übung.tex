\documentclass{article}
\usepackage[utf8]{inputenc}
\usepackage{amsmath, amssymb, amsthm}

\title{Übung 2}
\author{Pascal Diller, Timo Rieke}

\setcounter{secnumdepth}{0}

\begin{document}

\maketitle

\section{1}
\subsection{(i)}
\[R_1^{-1} = \{(z,x),(y,z)\}\]
$R_1 \bullet\ R_2 = \{(x,z),(x,x),(y,y),(y,z)\} $
\subsection{(ii)}
Überprüfen von $R_2$ auf \\
Reflexivität: 
$R_2$ ist Reflexiv, da $(x,x),(y,y),(z,z) \in R_2$ \\ 
Symmetrisch: 
$R_2$ ist symmetrisch, da bei allen $xRy$ ein $yRx$ folgt. \\
Antisymmetrie: 
$R_2$ ist nicht antisymmetrisch, da Paare wie $(x,y)$ und $(y,x)$ enthalten sind, ohne dass x=y gilt. \\
Asymmetrie: 
$R_2$ ist nicht asymmetrisch, da $(x,y) \in R_2$ und $(y,x) \in R_2$ \\ 
Transitivität:
$R_2$ ist nicht transitiv, da z.B. aus $(y,x)$ und $(x,z)$ NICHT $(y,z)$ folgt
\subsection{(iii)}
$R_1^{2} = R_1 \bullet R_1 = {(x,z),(z,y)} \bullet {(x,z),(z,y)} = {(x,y)}$ \\
$R_1^{3} = R_1^{2} \bullet R_1 = {(x,z),(z,y)} \bullet {(x,y)} = \emptyset$  \\
Reflexive Hülle von $R_1$: $R_1 \cup {(x,x),(y,y),(z,z)} =$ \\
Symmetrische Hülle von $R_1$: $R_1 \cup R_1^{-1} = {(x,z),(z,y),(z,x),(y,x)}$ \\
Transitive Hülle von $R_1$: ${(x,z),(z,y),(z,x)}$

\section{2}
\subsection{(i)}
Die Relation ist asymmetrisch, da aus $f(0) < g(0)$ folgt, dass $f(0) \ngtr g(0)$ \\
Die Relation ist transitiv, da bei einem $g<h$ ebenfalls folgen würde das $f<h$ ist. \\
$\implies$ strikte Ordnung
\subsection{(ii)}
Die Ordnung ist total, da zwischen den zwei Werten $f(0)$ und $g(0)$ immer gilt: $f(0) < g(0)$

\section{3}
\subsection{(i)}
R ist reflexiv, da wenn $x = y$ gilt: $f(x) = f(y)$ \\
R ist symmetrisch, da wenn $x = y$ auch gilt: $y = x$ \\
R ist transitiv, dan wenn $x = y$ und $y = z$, auch $x = z$ \\
$\implies$ Äquivalenzrelation
\subsection{(ii)}

\section{4}
\subsection{(i)}
\subsection{(ii)}
\subsection{(iii)}

\section{5}
Seien X und Y Mengen \\
Zu zeigen: Wenn $f:X \rightarrow Y$ injektiv ist, ist $f':X \rightarrow f(X), x\rightarrow f(x)$ bijektiv. \\ 
Eine Funktion ist injektiv, wenn für alle $x_1, x_2$ gilt: $f'(x_1)=f'(x_2)$. \\
Sei also $f'(x_1) = f'(x_2) \implies f(x_1) = f(x_2) \implies x_1 = x_2$  \\
Daher ist $f'$ injektiv \\
Sei $y \in f(X)$. Da $x \in X$ gilt $f(x)=y$ \\
Daraus folgt: $f'(x) = f(x) = y$ \\
Daher ist $f'$ surjektiv \\
Da $f'$ injektiv und surjektiv ist, ist $f'$ ebenfalls bijektiv. \\

\end{document}
