\documentclass{article}
\usepackage[utf8]{inputenc}
\usepackage{amsmath, amssymb, amsthm}

\title{Übung 9}
\author{Pascal Diller, Timo Rieke}

\setcounter{secnumdepth}{0}

\begin{document}

\maketitle

\section{Aufgabe 1}
\subsection{(i)}
\[A = \begin{pmatrix}
    1 & -2 & 0 \\
    2 & 1 & 1 \\
    1 & 2 & 1 \\
\end{pmatrix}\]
\[\mathcal{M}_A = \{1 \cdot 1 \cdot 1, 1 \cdot 2 \cdot 1, 2 \cdot (-2) \cdot 1, 2 \cdot 2 \cdot 0, 1 \cdot (-2) \cdot 1, 1 \cdot 1 \cdot 0\} \]
\[\text{det}(A) = \sum_{P \in \mathcal{M}_A}^{} (-1)^{\text{ Misstände in } P} \cdot P \]
\[= (-1)^0 \cdot 1 + (-1)^1 \cdot 2 + (-1)^1 \cdot (-4) + (-1)^2 \cdot 0 + (-1)^2 \cdot (-2) + (-1)^3 \cdot 0\]
\[= 1 - 2 + 4 - 2 = 1\]
Da det$(A) = 1 \neq 0$ ist $A$ invertierbar.

\[\left(\begin{array}{ccc|ccc}
    1 & -2 & 0 & 1 & 0 & 0 \\
    2 & 1 & 1 & 0 & 1 & 0 \\
    1 & 2 & 1 & 0 & 0 & 1 \\
\end{array}\right)\]
\[Z_2 - 2 \cdot Z_1 \to Z_2 \left(\begin{array}{ccc|ccc}
    1 & -2 & 0 & 1 & 0 & 0 \\
    0 & 5 & 1 & -2 & 1 & 0 \\
    1 & 2 & 1 & 0 & 0 & 1 \\
\end{array}\right)\]
\[Z_3 - Z_1 \to Z_3 \left(\begin{array}{ccc|ccc}
    1 & -2 & 0 & 1 & 0 & 0 \\
    0 & 5 & 1 & -2 & 1 & 0 \\
    0 & 4 & 1 & -1 & 0 & 1 \\
\end{array}\right)\]
\[Z_2 - Z_3 \to Z_2 \left(\begin{array}{ccc|ccc}
    1 & -2 & 0 & 1 & 0 & 0 \\
    0 & 1 & 0 & -1 & 1 & -1 \\
    0 & 4 & 1 & -1 & 0 & 1 \\
\end{array}\right)\]
\[Z_1 + 2 \cdot Z_2 \to Z_1 \left(\begin{array}{ccc|ccc}
    1 & 0 & 0 & -1 & 2 & -2 \\
    0 & 1 & 0 & -1 & 1 & -1 \\
    0 & 4 & 1 & -1 & 0 & 1 \\
\end{array}\right)\]
\[Z_3 - 4 \cdot Z_2 \to Z_3 \left(\begin{array}{ccc|ccc}
    1 & 0 & 0 & -1 & 2 & -2 \\
    0 & 1 & 0 & -1 & 1 & -1 \\
    0 & 0 & 1 & 3 & -4 & 5 \\
\end{array}\right)\]
\[A^{-1} = \begin{pmatrix}
    -1 & 2 & -2 \\
    -1 & 1 & -1 \\
    3 & -4 & 5 \\
\end{pmatrix}\]
\subsection{(ii)}
\[B_t = \begin{pmatrix}
    t & 1 & 1 \\
    1 & t & 1 \\
    1 & 1 & t \\
\end{pmatrix}\]
Wenn det$(B_t) \neq 0$, dann ist $B_t$ invertierbar.
\[\mathcal{M}_{B_t} = \{ t \cdot t \cdot t, t \cdot 1 \cdot 1, 1 \cdot 1 \cdot t, 1 \cdot 1 \cdot 1, 1 \cdot 1 \cdot 1, 1 \cdot t \cdot 1 \}\]
\[\text{det}(B_t) = (-1)^0 \cdot t^3 + (-1)^1 \cdot t + (-1)^1 \cdot t + (-1)^2 \cdot 1 + (-1)^2 \cdot 1 + (-1)^3 \cdot t\]
\[= t^3 - t - t + 1 + 1 - t\]
\[= t^3 - 3t + 2\]
Wenn $t^3 - 3t + 2 \neq 0$, dann ist $B_t$ invertierbar.
Für $t = 1$ und $t = -2$ ist $t^3 - 3t + 2 = 0$, also:
\[\text{Für alle } t \in \mathbb{R} \setminus \{-2, 1\} \text{ ist $B_t$ invertierbar.}\]
\subsection{(iii)}
\[C = \begin{pmatrix}
    1 & 0 & 1 & 0 \\
    -3 & 2 & 3 & 2 \\
    1 & 3 & 5 & 0 \\
    0 & 0 & 4 & 0 \\
\end{pmatrix}\]
\subsubsection{(a)}
Alle Muster, die eine 0 beinhalten, können ignoriert werden.
\[\mathcal{M}_C = \{1 \cdot 3 \cdot 4 \cdot 2\}\]
\[\text{det}(C) = (-1)^2 \cdot 1 \cdot 3 \cdot 4 \cdot 2 = 24\]
\subsubsection{(b)}
\[Z_2 + 3 \cdot Z_1 \to Z_2: \begin{pmatrix}
    1 & 0 & 1 & 0 \\
    0 & 2 & 6 & 2 \\
    1 & 3 & 5 & 0 \\
    0 & 0 & 4 & 0 \\
\end{pmatrix}\]
\[Z_3 - Z_1 \to Z_3: \begin{pmatrix}
    1 & 0 & 1 & 0 \\
    0 & 2 & 6 & 2 \\
    0 & 3 & 4 & 0 \\
    0 & 0 & 4 & 0 \\
\end{pmatrix}\]
\[Z_3 - 1.5 \cdot Z_2 \to Z_3: \begin{pmatrix}
    1 & 0 & 1 & 0 \\
    0 & 2 & 6 & 2 \\
    0 & 0 & -5 & -3 \\
    0 & 0 & 4 & 0 \\
\end{pmatrix}\]
\[Z_4 + \frac{4}{5} \cdot Z_3 \to Z_4: \begin{pmatrix}
    1 & 0 & 1 & 0 \\
    0 & 2 & 6 & 2 \\
    0 & 0 & -5 & -3 \\
    0 & 0 & 0 & -\frac{12}{5} \\
\end{pmatrix}\]
\[\text{det}(C) = 1 \cdot 2 \cdot (-5) \cdot (-\frac{12}{5}) = \frac{120}{5} = 24\]
\subsubsection{(c)}
\[\text{det}\begin{pmatrix}
    1 & 0 & 1 & 0 \\
    -3 & 2 & 3 & 2 \\
    1 & 3 & 5 & 0 \\
    0^- & 0^+ & 4^- & 0^+ \\
\end{pmatrix}\]
Aus der definition des Laplaceschen Entwicklungssatzes folgt, dass für einen Nulleintrag in der ausgewählten Zeile die Determinante der dazugehörigen Teilmatrix wegfällt.
\[\implies \text{det}(C) = -4 \cdot \text{ det}\begin{pmatrix}
    1^+ & 0^- & 0^+ \\
    -3 & 2 & 2 \\
    1 &3  & 0 \\
\end{pmatrix}\]
\[= -4 \cdot (\text{det}\begin{pmatrix}
    2 & 2 \\
    3 & 0 \\
\end{pmatrix}) = -4 \cdot (2 \cdot 0 - 3 \cdot 2) = -4 \cdot (-6) = 24\]


\section{Aufgabe 2}

\subsection{(i) Beziehung zwischen det(D) und det(kD)}
Es gilt:
\[det(kD) = k^n \cdot det(D)\]
Begründung:
\begin{center}
    Da jede Zeile der Matrix $D$ mit $k$ multiplizert wird und es $n$ Zeilen gibt, wird die Determinate ingesamt mit $k^n$ multipliziert. 
\end{center}

\subsection{(ii) Beziehung zwischen det(D) und det(kD)}
Für eine invertierbare Matrix $E$ gilt: $E \cdot E^{-1}=I$ ($I$ die Einheitsmatrix). Somit:
\[det(E \cdot E^{-1})=det(I)\]
Die Determinante des Produkts zweier Matrizen ist gleich dem Produkt der Determinanten. $det(I)=1$, somit folgt:
\[det(E) \cdot det(E^{-1})=1\]
Geteilt durch $det(E)$:
\[det(E^{-1})=\frac{1}{det(E)}\]
Die Beziehung zwischen der Determinante einer Matrix $E$ und ihrer Inversen $E^{-1}$ ist:
\[det(E^{-1})=\frac{1}{det(E)}\]

\subsection{(iii) Nachweis das det(F)=0} 
\begin{center}
    Da die letze Zeile von $F$ ist eine Mischung aus den vorherigen Zeilen ist, ist die Determinate 0. Das liegt an der Eigenschaft, dass für eine Matrix $A \in M_n(\mathbb{R})$ mit zwei gleichen Zeilen immer det(A)=0 gilt (Proposition 3.5.9 (ii)).
\end{center}

\end{document}
