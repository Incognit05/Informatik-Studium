\documentclass{article}
\usepackage[utf8]{inputenc}
\usepackage{amsmath, amssymb, amsthm}

\title{Übung 9}
\author{Pascal Diller, Timo Rieke}

\setcounter{secnumdepth}{0}

\begin{document}

\maketitle

\section{Aufgabe 1}
\subsection{(i)}
\[\left(\begin{array}{ccc|ccc}
    1 & -2 & 0 & 1 & 0 & 0 \\
    2 & 1 & 1 & 0 & 1 & 0 \\
    1 & 2 & 1 & 0 & 0 & 1 \\
\end{array}\right)\]
\[Z_2 - 2 \cdot Z_1 \to Z_2 \left(\begin{array}{ccc|ccc}
    1 & -2 & 0 & 1 & 0 & 0 \\
    0 & 5 & 1 & -2 & 1 & 0 \\
    1 & 2 & 1 & 0 & 0 & 1 \\
\end{array}\right)\]
\[Z_3 - Z_1 \to Z_3 \left(\begin{array}{ccc|ccc}
    1 & -2 & 0 & 1 & 0 & 0 \\
    0 & 5 & 1 & -2 & 1 & 0 \\
    0 & 4 & 1 & -1 & 0 & 1 \\
\end{array}\right)\]
\[Z_2 - Z_3 \to Z_2 \left(\begin{array}{ccc|ccc}
    1 & -2 & 0 & 1 & 0 & 0 \\
    0 & 1 & 0 & -1 & 1 & -1 \\
    0 & 4 & 1 & -1 & 0 & 1 \\
\end{array}\right)\]
\[Z_1 + 2 \cdot Z_2 \to Z_1 \left(\begin{array}{ccc|ccc}
    1 & 0 & 0 & -1 & 2 & -2 \\
    0 & 1 & 0 & -1 & 1 & -1 \\
    0 & 4 & 1 & -1 & 0 & 1 \\
\end{array}\right)\]
\[Z_3 - 4 \cdot Z_2 \to Z_3 \left(\begin{array}{ccc|ccc}
    1 & 0 & 0 & -1 & 2 & -2 \\
    0 & 1 & 0 & -1 & 1 & -1 \\
    0 & 0 & 1 & 3 & -4 & 5 \\
\end{array}\right)\]
\[A^{-1} = \begin{pmatrix}
    -1 & 2 & -2 \\
    -1 & 1 & -1 \\
    3 & -4 & 5 \\
\end{pmatrix}\]

\section{Aufgabe 2}

\subsection{(i) Beziehung zwischen det(D) und det(kD)}
Es gilt:
\[det(kD) = k^n \cdot det(D)\]
Begründung:
\begin{center}
    Da jede Zeile der Matrix $D$ mit $k$ multiplizert wird und es $n$ Zeilen gibt, wird die Determinate ingesamt mit $k^n$ multipliziert. 
\end{center}

\subsection{(ii) Beziehung zwischen det(D) und det(kD)}
Für eine invertierbare Matrix $E$ gilt: $E \cdot E^{-1}=I$ ($I$ die Einheitsmatrix). Somit:
\[det(E \cdot E^{-1})=det(I)\]
Die Determinante des Produkts zweier Matrizen ist gleich dem Produkt der Determinanten. $det(I)=1$, somit folgt:
\[det(E) \cdot det(E^{-1})=1\]
Geteilt durch $det(E)$:
\[det(E^{-1})=\frac{1}{det(E)}\]
Die Beziehung zwischen der Determinante einer Matrix $E$ und ihrer Inversen $E^{-1}$ ist:
\[det(E^{-1})=\frac{1}{det(E)}\]

\subsection{(iii) Nachweis das det(F)=0} 
\begin{center}
    Da die letze Zeile von $F$ ist eine Mischung aus den vorherigen Zeilen ist, ist die Determinate 0. Das liegt an der Eigenschaft, dass für eine Matrix $A \in M_n(\mathbb{R})$ mit zwei gleichen Zeilen immer det(A)=0 gilt (Proposition 3.5.9 (ii)).
\end{center}

\end{document}
