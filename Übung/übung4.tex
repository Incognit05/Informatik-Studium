\documentclass{article}
\usepackage{graphicx} % Required for inserting images

\title{Mathe Übung 4}
\author{Pascal Diller, Timo Rieke}
\date{11 November 2024}

\begin{document}

\maketitle
\section{Aufgabe 1}
\begin{center}
  (i) \\
$\sum_{k = 1}^{n}k^2 = \frac{n(n+1)(2n+1)}{6}$ \\ 
"\\
\textbf{Induktionsanfang:} für n=1 \\
$\sum_{k = 1}^{1}k^2 = 1^2 = 1 = \frac{1*2*3}{6} =\frac{1(1+1)(2*1+1)}{6}$\\
$\Longrightarrow{}$ Die Gleichung stimmt für n = 1 \\
.\\
\textbf{Induktionsvorraussetzung:} \\
Angenommen, die Formel gilt für ein beliebiges $n \in \mathbb{N}$ \\
$\sum_{k = 1}^{n}k^2 = \frac{n(n+1)(2n+1)}{6}$ \\ 
.\\
\textbf{Induktionsschritt:} Formel für n+1 \\
$\sum_{k = 1}^{n+1}k^2 = \frac{(n+1)(n+2)(2n+3)}{6}$ \\ 
Es gilt: 
$\sum_{k = 1}^{n+1} k^2 = \sum_{k = 1}^{n}k^2 + (n+1)^2$ \\
Einsetzen: 
$\sum_{k = 1}^{n+1}k^2 = \frac{n(n+1)(2n+1)}{6} + (n+1)^2$  \\ 
$ = \frac{n(n+1)(2n+1)+6(n+1)^2}{6}$  \\ 
$ = \frac{(n+1)(n(2n+1))+6(n+1))}{6}$  \\ 
$ = \frac{(n+1)(2n^2+n+6n+6)}{6} = \frac{(n+1)(2n^2+7n+6)}{6}$  \\ 
$ = \frac{(n+1)(n+2)(2n+3)}{6}$  \\ 
$\Longrightarrow{}$ Dies zeigt, dass die Induktionsannahme ebenfalls für n+1 gilt, somit ist die Formel bewiesen \\
. \\
(ii) \\ 
(a) \\
Berechne :$\sum_{k=1}^{n}(2k-1)^2$ \\
$= \sum_{k=1}^{n}(4k^2 - 4k +1)$ \\
$= 4\sum_{k=1}^{n}k^2 - 4\sum_{k=1}^{n}k + \sum_{k=1}^{n}1$\\
Berechnen der Teilsummen:\\
\textbf{1.} $\sum_{k=1}^{n} k^2 = \frac{n(n+1)(2n+1)}{6}$ \\
    $\Longrightarrow{}$$4*\sum_{k=1}^{n} k^2 = 4*\frac{n(n+1)(2n+1)}{6} = \frac{2n(n+1)(2n+1}{3}$ \\
\textbf{2.} $\sum_{k=1}^{n} k = \frac{n(n+1)}{2}$ \\
    $\Longrightarrow{} -4*\sum_{k=1}^{n}k = -4\frac{n(n+1)}{2} = -2n(n+1)$ \\
\textbf{3.} $\sum_{k=1}^{n} 1 = n$ \\
Zusammensetzen der Teilsummen:
$\sum_{k=1}^{n}(2k-1)^2 = \frac{2n(n+1)(2n+1}{3} -2n(n+1) + n$ \\
.\\
(b) \\
Berechne :$\sum_{k=2}^{n+2}2^{k-2}$ \\
Setze \( j = k - 2 \), dann wird die Summe:
$\sum_{k=2}^{n+2} 2^{k-2} = \sum_{j=0}^{n} 2^j$ \\
$\sum_{j=0}^{n} 2^j = 2^{n+1} - 1$ \\
Also ergibt sich: \\
$\sum_{k=2}^{n+2} 2^{k-2} = 2^{n+1} - 1 $\\

.\\
(c) \\
Berechne :$\sum_{k=1}^{n}\frac{1}{k(k+1)}$ \\
Zerlege den Bruch als Differenz: \\
$\frac{1}{k(k+1)} = \frac{1}{k} - \frac{1}{k+1}$ \\
Einsetzen in die Summe: \\
$\sum_{k=1}^n \left( \frac{1}{k} - \frac{1}{k+1} \right) = 1 - \frac{1}{n+1}$ \\
Somit erhalten wir: \\
$\sum_{k=1}^n \frac{1}{k(k+1)} = 1 - \frac{1}{n+1}$ \\


\end{center}

\section{Aufgabe 2}


\section{Aufgabe 3}

\end{document}
