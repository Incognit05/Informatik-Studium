\documentclass{article}
\usepackage{amsmath}
\usepackage{amssymb}
\title{Mathe Übung 12}
\author{Pascal Diller, Timo Rieke}
\begin{document}
\maketitle

\section*{Aufgabe 1}

\subsection*{(i)}
\[\lim_{n \to \infty} \frac{7n^{3/2} - n^{-2} \sqrt{n}}{(\sqrt{n})^3 + n^{1/2} + n}.\]
Im Zähler dominiert \(7n^{3/2}\), da \(n^{-2} \sqrt{n}\) für \(n \to \infty\) vernachlässigbar ist. \\
Im Nenner dominiert \(n\), da \(n > n^{3/2} > n^{1/2}\) für große \(n\). \\
Es ergibt sich:
\[\lim_{n \to \infty} \frac{7n^{3/2}}{n} = 7\sqrt{n} \to \infty.\]

\subsection*{(ii)}
\[\lim_{n \to \infty} \sqrt{n^4 + n^2 + 1} - \sqrt{n^2 - 1}.\]
\[\sqrt{n^4 + n^2 + 1} - \sqrt{n^2 - 1} = \frac{(n^4 + n^2 + 1) - (n^2 - 1)}{\sqrt{n^4 + n^2 + 1} + \sqrt{n^2 - 1}}.\]
\[n^4 + n^2 + 1 - n^2 + 1 = n^4 + 2.\]
Im Nenner dominiert \(\sqrt{n^4} = n^2\), daher:
\[\frac{n^4 + 2}{\sqrt{n^4 + n^2 + 1} + \sqrt{n^2 - 1}} \sim \frac{n^4}{2n^2} =\frac{n^2}{2}.\]

\subsection*{(iii)} 
\[\lim_{n \to \infty} \frac{n}{\sqrt{n^5 + 4n}}.\]
Im Nenner dominiert \(\sqrt{n^5} = n^{5/2}\). Somit:
\[\frac{n}{\sqrt{n^5 + 4n}} \sim \frac{n}{n^{5/2}} = n^{-3/2}.\]
Daraus folgt:
\[\lim_{n \to \infty} \frac{n}{\sqrt{n^5 + 4n}} = 0.\]

\subsection*{(iv)} 
\[\lim_{n \to \infty} \left(\frac{\sqrt{n} - 1}{n}\right)^n.\]
Sei \(L = \left(\frac{\sqrt{n} - 1}{n}\right)^n\), Logarithmieren:
\[\ln L = n \ln\left(\frac{\sqrt{n} - 1}{n}\right).\]
Annäherung:
\[\ln\left(\frac{\sqrt{n} - 1}{n}\right) = \ln(\sqrt{n} - 1) - \ln(n) \sim \ln(\sqrt{n}) - \ln(n) = -\frac{\ln(n)}{2}.\]
Somit:
\[\ln L = n \cdot \left(-\frac{\ln(n)}{2}\right) = -\frac{n \ln(n)}{2} \to -\infty.\]
Daraus folgt:
\[L \to 0.\]

\section*{Aufgabe 2}
\[a_1 = 0, \quad a_{n+1} = \frac{1}{2}a_n + \frac{1}{2}.\]

\subsection*{(i)}  
Zu zeigen: \(a_n \leq 1\) für alle \(n \in \mathbb{N}\).  \\
Induktionsanfang: Für \(n = 1\) gilt \(a_1 = 0 \leq 1\). \\
Induktionsschritt: Angenommen, \(a_n \leq 1\). Dann folgt:
    \[a_{n+1} = \frac{1}{2}a_n + \frac{1}{2} \leq \frac{1}{2}(1) + \frac{1}{2} =1.\]
Damit ist \(a_n \leq 1\) für alle \(n\).

\subsection*{(ii)}  
Zu zeigen: \(a_{n+1} \geq a_n\).  
\[a_{n+1} - a_n = \frac{1}{2}a_n + \frac{1}{2} - a_n = \frac{1}{2}(1 - a_n).\]
Da \(a_n \leq 1\), ist \(1 - a_n \geq 0\) und somit \(a_{n+1} \geq a_n\). Somit ist \((a_n)_{n\in \mathbb{N}}\) weiter monton wachsend.

\subsection*{(iii)}  
Da die Folge beschränkt und monoton wachsend ist folgt, dass \(a_n\) konvergiert. Sei:
\[\lim_{n \to \infty} a_n = L.\]
Im Limes gilt:
\[L = \frac{1}{2}L + \frac{1}{2}.\]
Umstellen ergibt:
\[L = 1.\]
Die Folge \(a_n\) konvergiert gegen den Grenzwert 1. 


\end{document}
