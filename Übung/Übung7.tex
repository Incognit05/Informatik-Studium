\documentclass{article}
\usepackage[utf8]{inputenc}
\usepackage{amsmath, amssymb, amsthm}

\title{Übung 7}
\author{Pascal Diller, Timo Rieke}

\setcounter{secnumdepth}{0}

\begin{document}

\maketitle

\section{Aufgabe 1}
\subsection{(i)}
$G_1$:
\[\left(\begin{matrix}
    0 & 0 & 1 & 1 & 0 \\
    0 & 1 & 1 & 0 & 0 \\
    1 & 1 & 0 & 0 & 0 \\
    1 & 0 & 0 & 1 & 0 \\
\end{matrix}\right)\]
$Z_1$ und $Z_3$ vertauschen
\[\left(\begin{matrix}
    1 & 1 & 0 & 0 & 0 \\
    0 & 1 & 1 & 0 & 0 \\
    0 & 0 & 1 & 1 & 0 \\
    1 & 0 & 0 & 1 & 0 \\
\end{matrix}\right)\]
$Z_4 - Z_1 \to Z_4$
\[\left(\begin{matrix}
    1 & 1 & 0 & 0 & 0 \\
    0 & 1 & 1 & 0 & 0 \\
    0 & 0 & 1 & 1 & 0 \\
    0 & -1 & 0 & 1 & 0 \\
\end{matrix}\right)\]
$Z_4 + Z_2 \to Z_4$
\[\left(\begin{matrix}
    1 & 1 & 0 & 0 & 0 \\
    0 & 1 & 1 & 0 & 0 \\
    0 & 0 & 1 & 1 & 0 \\
    0 & 0 & 1 & 1 & 0 \\
\end{matrix}\right)\]
$Z_4 - Z_3 \to Z_4$
\[\left(\begin{matrix}
    1 & 1 & 0 & 0 & 0 \\
    0 & 1 & 1 & 0 & 0 \\
    0 & 0 & 1 & 1 & 0 \\
    0 & 0 & 0 & 0 & 0 \\
\end{matrix}\right)\]
$G_1$ hat einen Rang von 3 und ist singulär, da $3 < 4$ (Anzahl Zeilen). \\
\newline
$G_2$:
\[\left(\begin{matrix}
    3 & 11 & 19 & -2 \\ 
    7 & 23 & 39 & 10 \\
    -4 & -3 & -2 & 6 \\
\end{matrix}\right)\]
$Z_2 - \frac{7}{3}\cdot Z_1 \to Z_2$
\[\left(\begin{matrix}
    3 & 11 & 19 & -2 \\ 
    0 & -\frac{8}{3} & -\frac{16}{3} & \frac{44}{3} \\
    -4 & -3 & -2 & 6 \\
\end{matrix}\right)\]
$Z_3 + \frac{4}{3} \cdot Z_1 \to Z_3$
\[\left(\begin{matrix}
    3 & 11 & 19 & -2 \\ 
    0 & -\frac{8}{3} & -\frac{16}{3} & \frac{44}{3} \\
    0 & \frac{35}{3} & \frac{70}{3} & \frac{10}{3} \\
\end{matrix}\right)\]
$Z_3 + \frac{35}{8} \cdot Z_2 \to Z_3$
\[\left(\begin{matrix}
    3 & 11 & 19 & -2 \\ 
    0 & -\frac{8}{3} & -\frac{16}{3} & \frac{44}{3} \\
    0 & 0 & 0 & \frac{135}{2} \\
\end{matrix}\right)\]
$G_2$ hat einen Grad von 3 und ist regulär, da $3 = 3$ (Anzahl Zeilen). \\
\newline
$G_3$:
\[\left(\begin{matrix}
    3 & 5 & 3 & 25 \\ 
    7 & 9 & 19 & 65 \\
    -4 & 5 & 11 & 5 \\
\end{matrix}\right)\]
$Z_2 - \frac{7}{3} \cdot Z_1 \to Z_2$
\[\left(\begin{matrix}
    3 & 5 & 3 & 25 \\ 
    0 & -\frac{8}{3} & 12 & \frac{20}{3} \\
    -4 & 5 & 11 & 5 \\
\end{matrix}\right)\]
$Z_3 + \frac{4}{3} \cdot Z_1 \to Z_3$
\[\left(\begin{matrix}
    3 & 5 & 3 & 25 \\ 
    0 & -\frac{8}{3} & 12 & \frac{20}{3} \\
    0 & \frac{35}{3} & 15 & \frac{115}{3} \\
\end{matrix}\right)\]
$Z_3 + \frac{35}{8} \cdot Z_2 \to Z_3$
\[\left(\begin{matrix}
    3 & 5 & 3 & 25 \\ 
    0 & -\frac{8}{3} & 12 & \frac{20}{3} \\
    0 & 0 & \frac{135}{2} & \frac{135}{2} \\
\end{matrix}\right)\]
$G_3$ hat einen Grad von 3 und ist regulär, da $3 = 3$(Anzahl Zeilen).
\subsection{(ii)}
$\mathbb{L}_{G_1} = \{t, -t, t, -t | t \in \mathbb{R}\}$ \\
$\mathbb{L}_{G_2} = \emptyset$ (Es sind keine Lösungen vorhanden) \\
$\mathbb{L}_{G_3} = \{4, 2, 1\}$
\section{Aufgabe 2}
 \begin{center}
    $\vec{v_1} = \left(\begin{array}{c} 2\\ 3 \\-2 \end{array}\right)$
    $\vec{v_2} = \left(\begin{array}{c} 1\\ 1 \\0 \end{array}\right)$
    $\vec{v_3} = \left(\begin{array}{c} 0\\ 2 \\3 \end{array}\right)$
 \end{center}
\subsection{(i)} 
\subsubsection{(a)}
    $\vec{v_1} \cdot \vec{v_1} = 2\cdot2+3\cdot3+(-2)\cdot(-2) =17\Longrightarrow$ nicht orthogonal  \\
    $\vec{v_2} \cdot \vec{v_2} = 1\cdot1+1\cdot1+0\cdot0= 2 \Longrightarrow$ nicht orthogonal\\
    $\vec{v_3} \cdot \vec{v_3} = 0\cdot0+2\cdot2+3\cdot3=13\Longrightarrow$ nicht orthogonal \\
    $\vec{v_1} \cdot \vec{v_2}= 2 \cdot 1 + 3 \cdot 1 + (-2) \cdot 0 = 5 \Longrightarrow$ nicht orthogonal \\
    $\vec{v_1} \cdot \vec{v_3}=1\cdot0+1\cdot2+0\cdot3=2\Longrightarrow$  nicht orthogonal \\
    $\vec{v_2} \cdot \vec{v_3}=2\cdot0+3\cdot2-2\cdot3 = 0\Longrightarrow$ orthogonal \\
\subsubsection{(b)}
Gesucht: $\vec{x}= \left(\begin{array}{c} x_1\\ x_2 \\x_3 \end{array}\right)$ \\ \newline
Da $\vec{x}$ orthogonal zu $\vec{v_1}$ und $\vec{v_2}$ sein soll ergeben sich folgende Gleichungen: 
\begin{center}
    1. $2x_1+3x_2-2x_3=0$ \\
    2. $x_1+x_2=0$
\end{center}
Aus $x_1+x_2=0$ ergibt sich $x_1= -x_2$. Einsetzen in 1 Formel: 
\begin{center}
    $-2x_2 + 3x_2-2x_3=0 \Leftrightarrow x_2-2x_3=0$ \\
\end{center}
Daraus folgt: $x_2= 2x_3$ und $x_1=-2x_3$ \\
Somit folgt: $\Longrightarrow \vec{x} = x_3\left(\begin{array}{c} -2\\ 2 \\1 \end{array}\right)$

\subsection{(ii)}
\subsubsection{(a)}
Zu zeigen ist:
\[2\|v\|^2 + 2\|w\|^2 = \|v + w\|^2 + \|v - w\|^2.\] 
Erweitern der Terme:\\
Für \(\|v + w\|^2\) gilt:
\[\|v + w\|^2 = (v + w) \cdot (v + w) = \|v\|^2 + 2(v \cdot w) + \|w\|^2.\]
Für \(\|v - w\|^2\) gilt:
\[\|v - w\|^2 = (v - w) \cdot (v - w) = \|v\|^2 - 2(v \cdot w) + \|w\|^2.\]
\[\|v + w\|^2 + \|v - w\|^2 = (\|v\|^2 + 2(v \cdot w) + \|w\|^2) + (\|v\|^2 - 2(v \cdot w) + \|w\|^2).\]
Die Terme \(+2(v \cdot w)\) und \(-2(v \cdot w)\) heben sich auf. Es bleibt:
\[\|v + w\|^2 + \|v - w\|^2 = 2\|v\|^2 + 2\|w\|^2.\]
Die Gleichung ist bewiesen:
\[2\|v\|^2 + 2\|w\|^2 = \|v + w\|^2 + \|v - w\|^2.\]



\subsubsection{(b)}
Setze $u=v-w$ ein: \\
$u\cdot v= u\cdot w \Longrightarrow (v-w)\cdot v = (v-w)\cdot w$
$\Longrightarrow v\cdot v - v \cdot w = w \cdot v - w \cdot w$ \\
$v\cdot w$ rauskürzen: \\
$v \cdot v - w \cdot w = 0 \Longrightarrow v\cdot v = w\cdot w$ \\
Somit folgt $v=w$

\subsection{(iii)}
Zu zeigen ist:
\[||\sum_{i=1}^{m}a_i|| \leq|\sum_{i=1}^{m}||a_i||\]
Induktionsanfang: \\
Für m=1:  \[||a_1||\leq ||a_1||\]
Induktionsschritt: \\
Für m+1: 
\[ ||\sum_{i=1}^{m+1}a_i|| = ||\sum_{i=1}^m a_i|| + ||a_{m+1}|| \]
Nach der Dreiecksungleichung gilt:
\[||\sum_{i=1}^ma_i|| \leq ||\sum_{i=1}^ma_i||+||a_{m+1}||\]
Nach Induktionsannahme gilt:
\[||\sum_{i=1}^ma_i|| \leq \sum_{i=1}^m||a_i||\]
Zusammen folgt:
\[||\sum_{i=1}^{m+1}a_i|| \leq \sum_{i=1}^m||a_i|| + ||a_{m+1}|| \Longrightarrow ||\sum_{i=1}^{m+1}a_i|| \leq \sum_{i=1}^{m+1}||a_i|| \]
Bewiesen durch Induktion



\subsection{(iv)}
Nach der Cauchy-Schwarz-Unlgeichung folgt: 
\[(\sum_{i=1}^n{\sqrt{x_i}\cdot \sqrt{x_1^{-1}}})^2 \leq (\sum_{i=1}^{n}x_i) (\sum_{i=1}^{n}x_i^{-1})\]
Da $\sqrt{x_i} \cdot \sqrt{x_i^{-1}} = 1 $, folgt:
\[(\sum_{i=1}^n 1)^2 =n^2 \leq (\sum_{i=1}^n x_i)(\sum_{i=1}^n x_i^{-1})\]
$\Longrightarrow$ Ungleichung bewiesen

\section{Aufgabe 3}
$f+g$ ist eine lineare Abbildung, wenn gilt:
\begin{align*}
    (f+g)(a+b) &= (f+g)(a)+(f+g)(b) & (1) \\
    (f+g)(\lambda a) &= \lambda(f+g)(a) & (2)
\end{align*}
\newline
\begin{align*}
    (f+g)(a+b) &= f(a+b)+g(a+b) &\text{Definition von $f+g$}\\
    &= f(a)+f(b) + g(a)+g(b) &\text{$f$ und $g$ sind linear}\\
    &= f(a)+g(a) + f(b) + g(b) \\
    &= (f+g)(a) + (f+g)(b) &\text{Definition von $f+g$}
\end{align*}
Somit ist $(1)$ gezeigt. \\
\newline
\begin{align*}
    (f+g)(\lambda a) &= f(\lambda a) + g(\lambda a) &\text{Definition von $f+g$} \\
    &= \lambda f(a) + \lambda g(a) & \text{$f$ und $g$ sind linear} \\
    &= \lambda(f(a)+g(a)) \\
    &= \lambda(f+g)(a) & \text{Definition von $f+g$}
\end{align*}
Somit ist $(2)$ gezeigt. \\

\end{document}
