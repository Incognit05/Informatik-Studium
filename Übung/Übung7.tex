\documentclass{article}
\usepackage[utf8]{inputenc}
\usepackage{amsmath, amssymb, amsthm}

\title{Übung 7}
\author{Pascal Diller, Timo Rieke}

\setcounter{secnumdepth}{0}

\begin{document}

\maketitle

\section{Aufgabe 3}
$f+g$ ist eine lineare Abbildung, wenn gilt:
\begin{align*}
    (f+g)(a+b) &= (f+g)(a)+(f+g)(b) & (1) \\
    (f+g)(\lambda a) &= \lambda(f+g)(a) & (2)
\end{align*}
\newline
\begin{align*}
    (f+g)(a+b) &= f(a+b)+g(a+b) &\text{Definition von $f+g$}\\
    &= f(a)+f(b) + g(a)+g(b) &\text{$f$ und $g$ sind linear}\\
    &= f(a)+g(a) + f(b) + g(b) \\
    &= (f+g)(a) + (f+g)(b) &\text{Definition von $f+g$}
\end{align*}
Somit ist $(1)$ gezeigt. \\
\newline
\begin{align*}
    (f+g)(\lambda a) &= f(\lambda a) + g(\lambda a) &\text{Definition von $f+g$} \\
    &= \lambda f(a) + \lambda g(a) & \text{$f$ und $g$ sind linear} \\
    &= \lambda(f(a)+g(a)) \\
    &= \lambda(f+g)(a) & \text{Definition von $f+g$}
\end{align*}
Somit ist $(2)$ gezeigt. \\

\end{document}