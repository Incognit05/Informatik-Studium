\documentclass{article}
\usepackage[utf8]{inputenc}
\usepackage{amsmath, amssymb, amsthm}

\title{Übung 7}
\author{Pascal Diller, Timo Rieke}

\setcounter{secnumdepth}{0}

\begin{document}

\maketitle

\section{Aufgabe 1}
\subsection{(i)}
$G_1$:
\[\left(\begin{matrix}
    0 & 0 & 1 & 1 & 0 \\
    0 & 1 & 1 & 0 & 0 \\
    1 & 1 & 0 & 0 & 0 \\
    1 & 0 & 0 & 1 & 0 \\
\end{matrix}\right)\]
$Z_1$ und $Z_3$ vertauschen
\[\left(\begin{matrix}
    1 & 1 & 0 & 0 & 0 \\
    0 & 1 & 1 & 0 & 0 \\
    0 & 0 & 1 & 1 & 0 \\
    1 & 0 & 0 & 1 & 0 \\
\end{matrix}\right)\]
$Z_4 - Z_1 \to Z_4$
\[\left(\begin{matrix}
    1 & 1 & 0 & 0 & 0 \\
    0 & 1 & 1 & 0 & 0 \\
    0 & 0 & 1 & 1 & 0 \\
    0 & -1 & 0 & 1 & 0 \\
\end{matrix}\right)\]
$Z_4 + Z_2 \to Z_4$
\[\left(\begin{matrix}
    1 & 1 & 0 & 0 & 0 \\
    0 & 1 & 1 & 0 & 0 \\
    0 & 0 & 1 & 1 & 0 \\
    0 & 0 & 1 & 1 & 0 \\
\end{matrix}\right)\]
$Z_4 - Z_3 \to Z_4$
\[\left(\begin{matrix}
    1 & 1 & 0 & 0 & 0 \\
    0 & 1 & 1 & 0 & 0 \\
    0 & 0 & 1 & 1 & 0 \\
    0 & 0 & 0 & 0 & 0 \\
\end{matrix}\right)\]
$G_1$ hat einen Rang von 3 und ist singulär, da $3 < 4$ (Anzahl Zeilen). \\
\newline
$G_2$:
\[\left(\begin{matrix}
    3 & 11 & 19 & -2 \\ 
    7 & 23 & 39 & 10 \\
    -4 & -3 & -2 & 6 \\
\end{matrix}\right)\]
$Z_2 - \frac{7}{3}\cdot Z_1 \to Z_2$
\[\left(\begin{matrix}
    3 & 11 & 19 & -2 \\ 
    0 & -\frac{8}{3} & -\frac{16}{3} & \frac{44}{3} \\
    -4 & -3 & -2 & 6 \\
\end{matrix}\right)\]
$Z_3 + \frac{4}{3} \cdot Z_1 \to Z_3$
\[\left(\begin{matrix}
    3 & 11 & 19 & -2 \\ 
    0 & -\frac{8}{3} & -\frac{16}{3} & \frac{44}{3} \\
    0 & \frac{35}{3} & \frac{70}{3} & \frac{10}{3} \\
\end{matrix}\right)\]
$Z_3 + \frac{35}{8} \cdot Z_2 \to Z_3$
\[\left(\begin{matrix}
    3 & 11 & 19 & -2 \\ 
    0 & -\frac{8}{3} & -\frac{16}{3} & \frac{44}{3} \\
    0 & 0 & 0 & \frac{135}{2} \\
\end{matrix}\right)\]
$G_2$ hat einen Grad von 3 und ist regulär, da $3 = 3$ (Anzahl Zeilen). \\
\newline
$G_3$:
\[\left(\begin{matrix}
    3 & 5 & 3 & 25 \\ 
    7 & 9 & 19 & 65 \\
    -4 & 5 & 11 & 5 \\
\end{matrix}\right)\]
$Z_2 - \frac{7}{3} \cdot Z_1 \to Z_2$
\[\left(\begin{matrix}
    3 & 5 & 3 & 25 \\ 
    0 & -\frac{8}{3} & 12 & \frac{20}{3} \\
    -4 & 5 & 11 & 5 \\
\end{matrix}\right)\]
$Z_3 + \frac{4}{3} \cdot Z_1 \to Z_3$
\[\left(\begin{matrix}
    3 & 5 & 3 & 25 \\ 
    0 & -\frac{8}{3} & 12 & \frac{20}{3} \\
    0 & \frac{35}{3} & 15 & \frac{115}{3} \\
\end{matrix}\right)\]
$Z_3 + \frac{35}{8} \cdot Z_2 \to Z_3$
\[\left(\begin{matrix}
    3 & 5 & 3 & 25 \\ 
    0 & -\frac{8}{3} & 12 & \frac{20}{3} \\
    0 & 0 & \frac{135}{2} & \frac{135}{2} \\
\end{matrix}\right)\]
$G_3$ hat einen Grad von 3 und ist regulär, da $3 = 3$(Anzahl Zeilen).
\subsection{(ii)}
$\mathbb{L}_{G_1} = \{t, -t, t, -t | t \in \mathbb{R}\}$ \\
$\mathbb{L}_{G_2} = \emptyset$ (Es sind keine Lösungen vorhanden) \\
$\mathbb{L}_{G_3} = \{4, 2, 1\}$
\section{Aufgabe 2}

\section{Aufgabe 3}
$f+g$ ist eine lineare Abbildung, wenn gilt:
\begin{align*}
    (f+g)(a+b) &= (f+g)(a)+(f+g)(b) & (1) \\
    (f+g)(\lambda a) &= \lambda(f+g)(a) & (2)
\end{align*}
\newline
\begin{align*}
    (f+g)(a+b) &= f(a+b)+g(a+b) &\text{Definition von $f+g$}\\
    &= f(a)+f(b) + g(a)+g(b) &\text{$f$ und $g$ sind linear}\\
    &= f(a)+g(a) + f(b) + g(b) \\
    &= (f+g)(a) + (f+g)(b) &\text{Definition von $f+g$}
\end{align*}
Somit ist $(1)$ gezeigt. \\
\newline
\begin{align*}
    (f+g)(\lambda a) &= f(\lambda a) + g(\lambda a) &\text{Definition von $f+g$} \\
    &= \lambda f(a) + \lambda g(a) & \text{$f$ und $g$ sind linear} \\
    &= \lambda(f(a)+g(a)) \\
    &= \lambda(f+g)(a) & \text{Definition von $f+g$}
\end{align*}
Somit ist $(2)$ gezeigt. \\

\end{document}