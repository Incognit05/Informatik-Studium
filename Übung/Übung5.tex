\documentclass{article}
\usepackage[utf8]{inputenc}
\usepackage{amsmath, amssymb, amsthm}

\title{Übung 5}
\author{Pascal Diller, Timo Rieke}

\setcounter{secnumdepth}{0}

\begin{document}

\maketitle
\section{Aufgabe 1}
\subsection{(i)}
\textbf{I.A.}
\[\left|\sum_{k=1}^{1} a_k\right| \leq \sum_{k=1}^{1} |a_k| = 1 \leq 1\]
\textbf{I.V.}
\[\left|\sum_{k=1}^{n} a_k\right| \leq \sum_{k=1}^{n}|a_k|\]
\textbf{I.S}
\begin{align*}
    |\sum_{k=1}^{n+1} a_k| &\leq \sum_{k=1}^{n+1} |a_k| \\
    = \left|\left(\sum_{k=1}^{n}a_k\right) + a_{n+1}\right| &\leq \left(\sum_{k=1}^{n}|a_k|\right) + |a_{n+1}| \\
    \overset{I.V.}{=} \left|\sum_{k=1}^{n+1} a_k\right| &\leq \sum_{k=1}^{n}|a_k| + |a_{n+1}| \\
    = \left|\sum_{k=1}^{n+1}a_k\right| &\leq \sum_{k=1}^{n+1}|a_k| 
\end{align*}
Somit ist gezeigt, dass $A(n) \implies A(n+1)$.
\newpage
\subsection{(ii)}
Zu zeigen: \[\sum_{k=0}^{n}(-1)^k \begin{pmatrix} n\\k \end{pmatrix} = 0\]
Binomischer Lehrsatz: \[(x+y)^n = \sum_{k=0}^{n} \begin{pmatrix}n\\k \end{pmatrix} x^k y^{n-k}\]
Seien $x = -1$ und $y = 1$:
\begin{align*}
    (-1 + 1)^n &= \sum_{k=0}^{n} \begin{pmatrix}n\\k\end{pmatrix}(-1)^k 1^{n-k} \\
    = 0^n &= \sum_{k=0}^{n} \begin{pmatrix}n\\k\end{pmatrix} (-1)^k \\
    = 0 &= \sum_{k=0}^{n}(-1)^k \begin{pmatrix}n\\k\end{pmatrix}
\end{align*}
\end{document}