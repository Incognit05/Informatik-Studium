\documentclass{article}
\usepackage{amsmath}
\usepackage{amssymb}
\usepackage[utf8]{inputenc}
\usepackage[ngerman]{babel}

\title{Übungsblatt 2}
\author{Pascal Diller, Timo Rieke}

\begin{document}
\maketitle

\section*{Aufgabe 1}
\subsection*{(i)}
\begin{align*}
    \lambda_1 (x^2 + 5x + 1) + \lambda_2 (9x + 1) + \lambda_3 (3x^2 - 3x + 1) &= 2x^2 + x = 1 \\
     (\lambda_1 + 3 \lambda_3)x^2 + (5 \lambda_1 + 9 \lambda_2 - 3 \lambda_3)x + (\lambda_1 + \lambda_2 + \lambda_3) &= 2x^2 + x = 1 \\
\end{align*}
Gleichsetzen der Koeffezienten
\begin{align}
    \lambda_1 + 3 \lambda_3 &= 2 \\ 
    \lambda_1 + 9 \lambda_2 - 3 \lambda_3 &= 1 \\
    \lambda_1 + \lambda_2 + \lambda_3 &= 1
\end{align}
Aus (1):
\[\lambda_1 + 3 \lambda_3 = 2 \Leftrightarrow \lambda_1 = 2 - 3 \lambda_3\]
Einsetzen in (3)
\[(2 - 3 \lambda_3) + \lambda_2 + \lambda_3 = 1\]
\[\Leftrightarrow \lambda_2 = 1 - (2 - 3\lambda_3) - \lambda_3 = 2 \lambda_3 - 1\]
Einsetzen in (2)
\[5(2 - 3 \lambda_3) + 9(2 \lambda_3 - 1) = 3 \lambda_3 = 1\]
\[10-15 \lambda_3 + 18 \lambda_3 - 9 - 3 \lambda_3 = 1\]
\[1 = 1 \rightarrow \text{ unbestimmtes System }\] 
Sei $\lambda_3 = t \in \mathbb{R}$ \\
dann:
\begin{align*}
    \lambda_1 &= 2 - 3t \\
    \lambda_2 &= 2t - 1 \\
    \lambda_3 &= t
\end{align*}
Sei $t = 0$, dann:
\begin{align*}
    \lambda_1 &= 2  \\
    \lambda_2 &= 1 \\
    \lambda_3 &= 0
\end{align*}
\[2x^2 + x + 1 = 2 (x^2 + 5x + 1) + 1 (9x + 1)\]

\subsection*{(ii)}
\subsubsection*{(a)}
\begin{align*}
    \lambda_1 (x^3 + x^2 + x + 1) + \lambda_2 (2x^2 - 4x + 6) + \lambda_3 (-x^2 - 2x + 5) &= 2x^3 - x^2 - 2x + 13 \\
    \lambda_1 x^3 + (\lambda_1 + 2 \lambda_2 - \lambda_3)x^2 + (\lambda_1 - 4 \lambda_2 - 2 \lambda_3)x + (\lambda_1 + 6 \lambda_2 + 5 \lambda_3) &= 2x^3 - x^2 - 2x + 13
\end{align*}
Gleichstellen der Koeffizienten
\begin{align}
    \lambda_1 &= 2 \\
    \lambda_1 + 2 \lambda_2 - \lambda_3 &= -1 \\
    \lambda_1 - 4\lambda_2 - 2\lambda_3 &= -2 \\
    \lambda_1 - 6\lambda_2 + 5\lambda_3 &= 13 
\end{align}
Einsetzen in (2)
\[2 + 2 \lambda_2 - \lambda_3 = -1 \Leftrightarrow \lambda_3 = 2 \lambda_2 + 3\]
Einsetzen in (3)
\[2 - 4 \lambda_2 - 2 (2 \lambda_2 + 3) = -2\]
\[\Leftrightarrow -8 \lambda_2 - 4 = - 2\]
\[\Leftrightarrow \lambda_2 = - \frac{1}{4}\]
Einsetzen in $\lambda_3$
\[\lambda_3 = 2 (- \frac{1}{4}) + 3 = 2.5 = \frac{5}{7}\]
Alle Koeffizienten wurden gefunden, also ist (a) ein Element von der Span.

\subsubsection*{b}

\subsection*{(iii)}
\subsubsection*{a}
\subsubsection*{b}

\end{document}
