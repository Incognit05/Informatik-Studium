\documentclass{article}
\usepackage{amsmath}
\usepackage{amssymb}
\usepackage[utf8]{inputenc}
\usepackage[ngerman]{babel}

\title{Übungsblatt 2}
\author{Pascal Diller, Timo Rieke}

\begin{document}
\maketitle

\section*{Aufgabe 1}
\subsection*{(i)}
\begin{align*}
    \lambda_1 (x^2 + 5x + 1) + \lambda_2 (9x + 1) + \lambda_3 (3x^2 - 3x + 1) &= 2x^2 + x = 1 \\
     (\lambda_1 + 3 \lambda_3)x^2 + (5 \lambda_1 + 9 \lambda_2 - 3 \lambda_3)x + (\lambda_1 + \lambda_2 + \lambda_3) &= 2x^2 + x = 1 \\
\end{align*}
Gleichsetzen der Koeffezienten
\begin{align}
    \lambda_1 + 3 \lambda_3 &= 2 \\ 
    \lambda_1 + 9 \lambda_2 - 3 \lambda_3 &= 1 \\
    \lambda_1 + \lambda_2 + \lambda_3 &= 1
\end{align}
Aus (1):
\[\lambda_1 + 3 \lambda_3 = 2 \Leftrightarrow \lambda_1 = 2 - 3 \lambda_3\]
Einsetzen in (3)
\[(2 - 3 \lambda_3) + \lambda_2 + \lambda_3 = 1\]
\[\Leftrightarrow \lambda_2 = 1 - (2 - 3\lambda_3) - \lambda_3 = 2 \lambda_3 - 1\]
Einsetzen in (2)
\[5(2 - 3 \lambda_3) + 9(2 \lambda_3 - 1) = 3 \lambda_3 = 1\]
\[10-15 \lambda_3 + 18 \lambda_3 - 9 - 3 \lambda_3 = 1\]
\[1 = 1 \rightarrow \text{ unbestimmtes System }\] 
Sei $\lambda_3 = t \in \mathbb{R}$ \\
dann:
\begin{align*}
    \lambda_1 &= 2 - 3t \\
    \lambda_2 &= 2t - 1 \\
    \lambda_3 &= t
\end{align*}
Sei $t = 0$, dann:
\begin{align*}
    \lambda_1 &= 2  \\
    \lambda_2 &= 1 \\
    \lambda_3 &= 0
\end{align*}
\[2x^2 + x + 1 = 2 (x^2 + 5x + 1) + 1 (9x + 1)\]

\subsection*{(ii)}
\subsubsection*{(a)}
\begin{align*}
    \lambda_1 (x^3 + x^2 + x + 1) + \lambda_2 (2x^2 - 4x + 6) + \lambda_3 (-x^2 - 2x + 5) &= 2x^3 - x^2 - 2x + 13 \\
    \lambda_1 x^3 + (\lambda_1 + 2 \lambda_2 - \lambda_3)x^2 + (\lambda_1 - 4 \lambda_2 - 2 \lambda_3)x + (\lambda_1 + 6 \lambda_2 + 5 \lambda_3) &= 2x^3 - x^2 - 2x + 13
\end{align*}
Gleichstellen der Koeffizienten
\begin{align}
    \lambda_1 &= 2 \\
    \lambda_1 + 2 \lambda_2 - \lambda_3 &= -1 \\
    \lambda_1 - 4\lambda_2 - 2\lambda_3 &= -2 \\
    \lambda_1 - 6\lambda_2 + 5\lambda_3 &= 13 
\end{align}
Einsetzen in (2)
\[2 + 2 \lambda_2 - \lambda_3 = -1 \Leftrightarrow \lambda_3 = 2 \lambda_2 + 3\]
Einsetzen in (3)
\[2 - 4 \lambda_2 - 2 (2 \lambda_2 + 3) = -2\]
\[\Leftrightarrow -8 \lambda_2 - 4 = - 2\]
\[\Leftrightarrow \lambda_2 = - \frac{1}{4}\]
Einsetzen in $\lambda_3$
\[\lambda_3 = 2 (- \frac{1}{4}) + 3 = 2.5 = \frac{5}{7}\]
Alle Koeffizienten wurden gefunden, also ist (a) ein Element von der Span.

\subsubsection*{(b)}
\[\lambda_1(x^3 + x^2 + x + 1) + \lambda_2(2x^2 - 4x + 6) + \lambda_3(-x^2 - 2x + 5) = x^3 + 12x^2 -4x + 6\]
\[\lambda_1 x^3 + (\lambda_1 + 2 \lambda_2 - \lambda_3)x^2 + (\lambda_1 - 4 \lambda_2 - 2 \lambda_3)x (\lambda_1 + 6 \lambda_2 + 5 \lambda_3) = x^3 + 12x^2 - 4x + 6\]
Gleichsetzen der Koeffizienten
\begin{align}
    \lambda_1 &= 1 \\
    \lambda_1 + 2 \lambda_2 - \lambda_3 &= 12 \\
    \lambda_1 - 4 \lambda_2 - 2 \lambda_3 &= -4 \\
    \lambda_1 - 6 \lambda_2 - 5 \lambda_3 &= 4 
\end{align}
Einsetzen in (2)
\[1 + 2 \lambda_2 - \lambda_3 = 12 \Leftrightarrow \lambda_3 = 2 \lambda_2 - 11\]
Einsetzen in (3)
\[1 - 4 \lambda_2 - 2 (2 \lambda_2 - 11) = -4\]
\[= -8 \lambda_2 + 23 = -4\]
\[= -8 \lambda_2 = -27\]
\[= \lambda_2 = \frac{27}{8}\]
Einsetzen in (4)
\[1 + 6 \cdot \frac{27}{8} + 5 \cdot (2 \cdot \frac{27}{8} - 11) = 4\]
\[1 + 6 \cdot \frac{27}{8} + 10 \cdot \cdot \frac{27}{8} - 55 = 4\]
\[16 \cdot \frac{27}{8} \neq 5\]
Es gibt keine Lösung, also ist ist (b) kein Element vom Span

\subsection*{(iii)}
\subsubsection*{a}
Prüfe auf lineare Unabhängigkeiten
\[\lambda_1(x^2 + x + 1) + \lambda_2(2x^2 + 2x) + \lambda_3(-4x^2  4x + 2) = 0\]
\[(\lambda_1 + 2 \lambda_2 - 4 \lambda_3)x^2 + (\lambda_1 + 2 \lambda_2 - 4\lambda_3)x + (\lambda_1 + 2 \lambda_3) = 0\]
Gleichsetzen der Koeffizienten
\begin{align}
    \lambda_1 + 2 \lambda_2 - 4 \lambda_3 &= 0 \\
    \lambda_1 + 2 \lambda_2 - 4 \lambda_3 &= 0 \\
    \lambda_1 + 2 \lambda_3 &= 0 
\end{align}
Aus (3):
\[\lambda_1 = -2 \lambda_3\]
In (1) einsetzen
\[-2 \lambda_3 + 2 \lambda_2 - 4 \lambda_3 = 0\]
\[\Leftrightarrow \lambda_2 = 3 \lambda_3\]
In (2) einsetzen
\[-2 \lambda_3 + 2 \cdot (3 \lambda_3) - 4 \lambda_3 = 0\]
\[\Leftrightarrow 0 = 0\]
Kein eindeutige Lösung, also kein Erzeugendensystem.

\subsubsection*{(b)}
\[\lambda_1(x^2 + x + 1) + \lambda_2 (x^2 + 2x + 2) + \lambda_3(-3x + 4) = 0\]
\[(\lambda_1 + \lambda_2)x^2 + (\lambda_1 + 2 \lambda_2 - 3 \lambda_3)x + (\lambda_1 + \lambda_2 - 4 \lambda_3) = 0\]
\begin{align}
    \lambda_1 + \lambda_2 &= 0 \\
    \lambda_1 + 2\lambda_2 - 3 \lambda_3 &= 0 \\
    \lambda_1 + 2\lambda_2 - 4 \lambda_3 &= 0 
\end{align}
\[(1) \Leftrightarrow \lambda_1 = - \lambda_2\]
(2)
\[-\lambda_2 + 2 \lambda_2 - 2 \lambda_3 = 0 \Leftrightarrow \lambda_2 = 3 \lambda_3 \Leftrightarrow \lambda_3 = \frac{1}{3} \lambda_2\]
(3)
\[\lambda_2 + \lambda_2 - 4 (\frac{1}{3} \lambda_2) = 0\]
\[-\frac{4}{3} = 0 \Leftrightarrow \lambda_2 = 0\]
\[\lambda_1 + 0 = 0 \Leftrightarrow \lambda_1 = 0\]
\[0 + 0 - 4 \lambda_3 = 0 \Leftrightarrow \lambda_3 = 0\]
Es gibt eine Lösung, also ist es ein Erzeugendensystem.

\section*{Aufgabe 2}

\subsection*{(i)}
Beweis per Induktion nach n: 
\subsubsection*{IA (n=1):} Sei $w_1 \in W$. Eine Linearkombination ist $\lambda_1 w_1$. Da W ein Unterraum ist, ist W abgeschlossen unter Skalarmultiplikation, also $\lambda_1 w_1 \in W$. 
\subsubsection*{IV:} Die Behauptung gelte für ein $n \in \mathbb{N}$. 
\subsubsection*{IS (n $\to$ n+1):} Seien $(w_1, ..., w_n, w_{n+1}) \in W^{n+1}$ und $\lambda_1, ..., \lambda_{n+1} \in \mathbb{R}$. \\ Betrachte $w = \sum_{i=1}^{n+1} \lambda_i w_i = (\sum_{i=1}^n \lambda_i w_i) + \lambda_{n+1} w_{n+1}$. \\
Sei $w' = \sum_{i=1}^n \lambda_i w_i$. Nach IV ist $w' \in W$. \\
Da W ein Unterraum ist, ist $\lambda_{n+1} w_{n+1} \in W$ (abgeschlossen unter Skalarmultiplikation). \\
Da W ein Unterraum ist, ist $w = w' + \lambda_{n+1} w_{n+1} \in W$ (abgeschlossen unter Addition).
    

\subsection*{(ii)}
Prüfen der Unterraumkriterien:
\subsubsection*{$0_V \in Span(M)$?}
Ja, da $M \ne \emptyset$, existiert ein $v \in M$. Dann ist $0_V = 0 \cdot v \in Span(M)$.
\subsubsection*{Abgeschlossenheit unter Addition und Skalarmultiplikation?}
Seien $w, w' \in Span(M)$ und $\lambda \in \mathbb{R}$.\\ 
Dann existieren $v_1, ..., v_n \in M$ und $u_1, ..., u_m \in M$ sowie Skalare $\alpha_i, \beta_j \in \mathbb{R}$ mit $w = \sum_{i=1}^n \alpha_i v_i$ und $w' = \sum_{j=1}^m \beta_j u_j$.\\
Es folgt $\lambda w + w' = \lambda (\sum_{i=1}^n \alpha_i v_i) + (\sum_{j=1}^m \beta_j u_j) = \sum_{i=1}^n (\lambda \alpha_i) v_i + \sum_{j=1}^m \beta_j u_j$. \\
Dies ist wieder eine Linearkombination von Elementen aus M. Also ist $\lambda w + w' \in Span(M)$. \\
Damit ist Span(M) ein Unterraum von V. 

\end{document}
