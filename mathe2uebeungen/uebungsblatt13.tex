\documentclass{article}
\usepackage{graphicx} % Required for inserting images
\usepackage{amsmath}
\usepackage{amssymb}

\title{Übungsblatt 11}
\author{Pascal Diller, Timo Rieke}

\begin{document}
\maketitle

\section*{Aufgabe 1}
\subsection*{(i)}
\[f(x) = x^{\frac{1}{2}} = e^{\frac{\ln(x)}{x}}\]
\[\text{Sei } g(x) = \frac{\ln(x)}{x}, \text{ dann ist } f(x) = e^{g(x)}\]
\[f'(x) = e^{g(x)} \cdot g'(x)\]
\[g'(x) = \frac{\frac{1}{x} \cdot x - \ln(x) \cdot 1}{x^2} = \frac{1 - \ln(x)}{x^2}\]
\[\text{Da } e^{g(x)} > 0 \text{ gilt:} \]
\[f'(x) = 0 \Leftrightarrow g'(x) = 0 \Leftrightarrow 1 - \ln(x) = 0 \Leftrightarrow \ln(x) = 1 \Leftrightarrow x = e\]
\[\]
Es existiert genau eine kritische Stelle an $x_0 = e$.

\subsection*{(ii)}
Untersuchen des Vorzeichens von $f'$
\[\text{ Für } x \in (0, e): \ln(x) < 1 \implies g'(x) > 0 \implies f'(x) > 0 \to \text{ streng monoton wachsend}\]
\[\text{ Für } x \in (e, \infty): \ln(x) > 1 \implies g'(x) < 0 \implies f'(x) < 0 \to \text{ streng monoton fallend}\]

\section*{Aufgabe 2}
\subsection*{(i)}
\[f'(x) = \frac{(2x)(x+1) - (x^2+1)(1)}{(x+1)^2} = \frac{x^2+2x-1}{(x+1)^2}.\]
\(f'(x) = 0\) setzen:
\[x^2+2x-1 = 0\]
Die Lösungen sind \(x = -1 \pm \sqrt{2}\). Nur \(x_0 = \sqrt{2}-1\) liegt im Intervall \((0, 2)\).
\[f''(x) = \frac{4}{(x+1)^3}.\]
Einsetzen:
\[f''(\sqrt{2}-1) = \frac{4}{(\sqrt{2}-1+1)^3} = \frac{4}{(\sqrt{2})^3} = \sqrt{2} > 0.\]
Daher liegt bei \(x_0 = \sqrt{2}-1\) ein lokales Minimum vor. Der Extremwert ist \(f(\sqrt{2}-1) = 2\sqrt{2}-2\).

\subsection*{(ii)}
Lokales Minimum: \[f(\sqrt{2}-1) = 2\sqrt{2}-2 \approx 0.828\]
Randpunkt \(x=0\): \[f(0) = 1\]
Randpunkt \(x=2\): \[f(2) = \frac{5}{3} \approx 1.667\]
Daraus folgt: \\
Das globale Minimum ist \[m = 2\sqrt{2}-2\]
Das globale Maximum ist \[M = \frac{5}{3}\]

\subsection*{(iii)}
Das Vorzeichen von \(f'(x) = \frac{x^2+2x-1}{(x+1)^2}\) hängt nur vom Zähler \(x^2+2x-1\) ab. \\
Dessen Nullstelle im Intervall ist \(x_0 = \sqrt{2}-1\). \\
\newline
Für \(x \in [0, \sqrt{2}-1)\) ist \(f'(x) < 0\), somit ist die Funktion auf \([0, \sqrt{2}-1]\) streng monoton fallend. \\
Für \(x \in (\sqrt{2}-1, 2]\) ist \(f'(x) > 0\), somit ist die Funktion auf \([\sqrt{2}-1, 2]\) streng monoton wachsend.

\section*{Aufgabe 3}
\subsection*{(i)}
Wenn $x \to \pi$, geht der Term $(\pi - x)$ gegen 0. Gleichzeitig geht $\tan(\frac{\pi}{2})$ gegen $\tan(\frac{\pi}{2})$, was gegen unendlich strebt. Also: $0 \cdot \infty$ \\
Da $\tan(\alpha) = \frac{1}{\cot{\alpha}}$:
\[(\pi - x)\tan(\frac{x}{2} = \frac{\pi - x}{\cot(\frac{x}{2})}\]
Bei diesem Bruch streben sowohl nenner, als auch Zaehler gegen 0.
\[\frac{d}{dx}(\pi - x) = -1\]
\[\frac{d}{dx}(\cot(\frac{x}{2})) = -\csc^2(\frac{x}{2}) \cdot \frac{1}{2} = -\frac{1}{2 \sin^2(\frac{x}{2})}\]
\[\lim_{x\to\pi} \frac{-1}{-\frac{1}{2} \csc^2(\frac{x}{2})} = \lim_{x\to\pi} 2 \sin^2(\frac{x}{2})\]
\[\pi\text{ einsetzen: } 2\sin^2(\frac{\pi}{2}) = 2(1)^2 = 2\]
Der Grenzwert ist 2.

\section*{Aufgabe 4}
Sei 
\[g(x) := f(x)e^{-x}\]
Ableitung:
\[g'(x) = f'(x)e^{-x} + f(x)(-e^{-x}) = (f'(x) - f(x))e^{-x}.\]
Nach Voraussetzung ist \(f'(x) = f(x)\), also ist \(f'(x) - f(x) = 0\).
\[g'(x) = 0 \cdot e^{-x} = 0.\]
Da die Ableitung von \(g(x)\) auf dem Intervall \((a,b)\) null ist, muss \(g(x)\) konstant sein. Es gibt also ein \(c \in \mathbb{R}\) mit:
\[g(x) = c.\]
Durch Einsetzen der Definition von \(g(x)\) folgt:
\[f(x)e^{-x} = c.\]
Somit ist \(f(x) = ce^x\), was zu beweisen war.

\end{document}
