\documentclass{article}
\usepackage{amsmath}
\usepackage{amssymb}
\usepackage[utf8]{inputenc}
\usepackage[ngerman]{babel}

\title{Übungsblatt 4}
\author{Pascal Diller, Timo Rieke}

\begin{document}
\maketitle

\section*{Aufgabe 1}
\subsection*{(i)}
\subsubsection*{(a)}
Seien $f(x) = a_1x^3 + b_1x^2 + c_1x + d_1$ und $g(x) = a_2x^3 + b_2x^2 + c_2x + d_2$ \\
\newline
Dann ist $f(x) + g(x) = a_1x^3 + b_1x^2 + c_1x + d_1 + a_2x^3 + b_2x^2 + c_2x + d_2 = (a_1 + a_2)x^3 + (b_1 + b_2)x^2 + (c_1 + c_2)x + (d_1 + d_2)$ \\
\newline
Also gilt: \[T(f + g) = (2(a_1 + a_2) + (b_1 + b_2) + 3(d_1 + d_2) , (a_1 + a_2) + (c_1 + c_2) - (d_1 + d_2))\]
\[= (2a_1 + b_1 + 3d_1, a_1 + c_1 - d_1) + (2a_2 + b_2 + 3d_2, a_2 + c_2 - d_2 = T(f) + T(g))\] \\
\newline
Sei $\lambda \in \mathbb{R}$, dann gilt:
\[T(\lambda f) = T(\lambda a_1 x^3 + \lambda b_1 x^2 + \lambda c_1 x + \lambda d_1) = (2 \lambda a_1 + \lambda b_1 + 3 \lambda d_1, \lambda a_1 + \lambda c_1 - \lambda d_1)\] 
\[ = \lambda(2a_1 + b_1 + 3d_1, a_1 + c_1  d_1 = \lambda T(f) )\]
Also ist $T$ linear.

\subsubsection*{(b)}
$T(a_1 x^3 + b_1 x^2 + c_1 x + d_1) = (0, 0)$
\begin{align}
    2a + b + 3d &= 0 \\
    a + c - d &= 0
\end{align}
\[\Longleftrightarrow\]

\begin{align*}
    (1) \: b &= -2a - 3d \\
    (2) \: c &= -a + d
\end{align*}
Parameterfrei ist die Lösung: $(a,b,c,d) = a(1, -2, -1, 0) + d(0, -3, 1, 1)$ \\
\newline
Also:
\[\ker(T) = \text{Span} \left\{ x^3 - 2x^2 - x,\ -3x^2 + x + 1 \right\}\]

\subsubsection*{(c)}
Da \( T: \mathbb{R}^4 \rightarrow \mathbb{R}^2 \) linear ist und 
\[
\dim(\ker(T)) = 2 \quad \Rightarrow \quad \dim(\text{Im}(T)) = 4 - 2 = 2
\]

Da der Zielraum \( \mathbb{R}^2 \) ebenfalls Dimension 2 hat, folgt:

\textbf{Folgerung:} \( T \) ist surjektiv.


\subsection*{(ii)}
\subsubsection*{(a)}
Berechne die Bilder der Basisvektoren von \( \mathbb{R}^2 \):

\[
S(1, 0) = 
\begin{pmatrix}
1 & 2 \\
-1 & 0
\end{pmatrix}, \quad
S(0, 1) = 
\begin{pmatrix}
1 & -1 \\
2 & 1
\end{pmatrix}
\]

Diese Matrizen sind linear unabhängig.

\textbf{Folgerung:}
\[
\text{Im}(S) = \text{Span} \left\{
\begin{pmatrix}
1 & 2 \\
-1 & 0
\end{pmatrix},
\begin{pmatrix}
1 & -1 \\
2 & 1
\end{pmatrix}
\right\}
\]

\subsubsection*{(b)}

Gesucht: \( (a, b) \in \mathbb{R}^2 \), sodass \( S(a, b) = 0 \). Also:

\[
\begin{cases}
a + b = 0 \\
2a - b = 0 \\
-a + 2b = 0 \\
b = 0
\end{cases}
\Rightarrow b = 0 \Rightarrow a = 0
\]

\textbf{Ergebnis:} \( \ker(S) = \{(0, 0)\} \)

\subsubsection*{(c)}
Da \( \ker(S) = \{(0, 0)\} \), folgt direkt:

\( S \) ist injektiv.

\end{document}
