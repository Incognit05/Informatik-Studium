\documentclass{article}

\usepackage{amsmath}
\usepackage{amssymb}

\title{Übungsblatt 4}
\author{Pascal Diller, Timo Rieke}

\begin{document}

\maketitle

\section*{Aufgabe 1}
\subsection*{(i)}
\subsubsection*{(a)}
Seien $f(x) = a_1x^3 + b_1x^2 + c_1x + d_1$ und $g(x) = a_2x^3 + b_2x^2 + c_2x + d_2$ \\
\newline
Dann ist $f(x) + g(x) = a_1x^3 + b_1x^2 + c_1x + d_1 + a_2x^3 + b_2x^2 + c_2x + d_2 = (a_1 + a_2)x^3 + (b_1 + b_2)x^2 + (c_1 + c_2)x + (d_1 + d_2)$ \\
\newline
Also gilt: \[T(f + g) = (2(a_1 + a_2) + (b_1 + b_2) + 3(d_1 + d_2) , (a_1 + a_2) + (c_1 + c_2) - (d_1 + d_2))\]
\[= (2a_1 + b_1 + 3d_1, a_1 + c_1 - d_1) + (2a_2 + b_2 + 3d_2, a_2 + c_2 - d_2 = T(f) + T(g))\] \\
\newline
Sei $\lambda \in \mathbb{R}$, dann gilt:
\[T(\lambda f) = T(\lambda a_1 x^3 + \lambda b_1 x^2 + \lambda c_1 x + \lambda d_1) = (2 \lambda a_1 + \lambda b_1 + 3 \lambda d_1, \lambda a_1 + \lambda c_1 - \lambda d_1)\] 
\[ = \lambda(2a_1 + b_1 + 3d_1, a_1 + c_1  d_1 = \lambda T(f) )\]
Also ist $T$ linear.

\subsubsection*{(b)}
$T(a_1 x^3 + b_1 x^2 + c_1 x + d_1) = (0, 0)$
\begin{align}
    2a + b + 3d &= 0 \\
    a + c - d &= 0
\end{align}
\[\Longleftrightarrow\]

\begin{align*}
    (1) \: b &= -2a - 3d \\
    (2) \: c &= -a + d
\end{align*}
Parameterfrei ist die Lösung: $(a,b,c,d) = a(1, -2, -1, 0) + d(0, -3, 1, 1)$ \\
\newline
Also:
\[\ker(T) = \text{Span} \left\{ x^3 - 2x^2 - x,\ -3x^2 + x + 1 \right\}\]

\subsubsection*{(c)}
Da \( T: \mathbb{R}^4 \rightarrow \mathbb{R}^2 \) linear ist und 
\[
\dim(\ker(T)) = 2 \quad \Rightarrow \quad \dim(\text{Im}(T)) = 4 - 2 = 2
\]

Da der Zielraum \( \mathbb{R}^2 \) ebenfalls Dimension 2 hat, folgt:

\textbf{Folgerung:} \( T \) ist surjektiv.


\subsection*{(ii)}
\subsubsection*{(a)}
Berechne die Bilder der Basisvektoren von \( \mathbb{R}^2 \):

\[
S(1, 0) = 
\begin{pmatrix}
1 & 2 \\
-1 & 0
\end{pmatrix}, \quad
S(0, 1) = 
\begin{pmatrix}
1 & -1 \\
2 & 1
\end{pmatrix}
\]

Diese Matrizen sind linear unabhängig.

\textbf{Folgerung:}
\[
\text{Im}(S) = \text{Span} \left\{
\begin{pmatrix}
1 & 2 \\
-1 & 0
\end{pmatrix},
\begin{pmatrix}
1 & -1 \\
2 & 1
\end{pmatrix}
\right\}
\]

\subsubsection*{(b)}

Gesucht: \( (a, b) \in \mathbb{R}^2 \), sodass \( S(a, b) = 0 \). Also:

\[
\begin{cases}
a + b = 0 \\
2a - b = 0 \\
-a + 2b = 0 \\
b = 0
\end{cases}
\Rightarrow b = 0 \Rightarrow a = 0
\]

\textbf{Ergebnis:} \( \ker(S) = \{(0, 0)\} \)

\subsubsection*{(c)}
Da \( \ker(S) = \{(0, 0)\} \), folgt direkt:

\( S \) ist injektiv.

\section*{Aufgabe 2}
\subsection*{$B_1 = (x^2+2x-3, 3x^2+2x-5, -3x^2+2x+1)$}
$\lambda_1(x^2+2x-3) + \lambda_2(3x^2+2x-5) + \lambda_3(-3x^2+2x+1) = 0$. \\ 
Koeffizientenvergleich:
\[ \lambda_1 + 3\lambda_2 - 3\lambda_3 = 0 \]
\[ 2\lambda_1 + 2\lambda_2 + 2\lambda_3 = 0 \implies \lambda_1 + \lambda_2 + \lambda_3 = 0 \]
\[ -3\lambda_1 - 5\lambda_2 + \lambda_3 = 0 \]
Aus (2) folgt $\lambda_1 = -\lambda_2 - \lambda_3$. \\
Einsetzen in (1): $(-\lambda_2 - \lambda_3) + 3\lambda_2 - 3\lambda_3 = 0 \implies 2\lambda_2 = 4\lambda_3 \implies \lambda_2 = 2\lambda_3$. \\
Damit $\lambda_1 = -2\lambda_3 - \lambda_3 = -3\lambda_3$. \\
Einsetzen in (3): $-3(-3\lambda_3) - 5(2\lambda_3) + \lambda_3 = 9\lambda_3 - 10\lambda_3 + \lambda_3 = 0$. \\
Das System hat nichttriviale Lösungen (z.B. $\lambda_3=1, \lambda_2=2, \lambda_1=-3$). \\
Somit ist $B_1$ linear abhängig und somit keine Basis.

\subsection*{$B_2 = (x^2+2x-3, 3x^2+2x-5, -3x^2+2x+4)$}
$\lambda_1(x^2+2x-3) + \lambda_2(3x^2+2x-5) + \lambda_3(-3x^2+2x+4) = 0$. \\
Koeffizientenvergleich:
\[ \lambda_1 + 3\lambda_2 - 3\lambda_3 = 0 \]
\[ 2\lambda_1 + 2\lambda_2 + 2\lambda_3 = 0 \implies \lambda_1 + \lambda_2 + \lambda_3 = 0 \]
\[ -3\lambda_1 - 5\lambda_2 + 4\lambda_3 = 0 \]
Wie oben folgt $\lambda_1 = -3\lambda_3$ und $\lambda_2 = 2\lambda_3$. \\
Einsetzen in (3): $-3(-3\lambda_3) - 5(2\lambda_3) + 4\lambda_3 = 9\lambda_3 - 10\lambda_3 + 4\lambda_3 = 3\lambda_3 = 0$. \\
Daraus folgt $\lambda_3 = 0$, und somit auch $\lambda_1 = 0$ und $\lambda_2 = 0$. \\
Die einzige Lösung ist die triviale Lösung. \\
Somit ist $B_2$  linear unabhängig und somit eine Basis.

\section*{Aufgabe 3}

\subsection*{IA (n=1):} 
$T(\lambda_1 v_1) = \lambda_1 T(v_1)$

\subsection*{IV:} 
Es gelte $T\left(\sum_{i=1}^{n} \lambda_i v_i\right) = \sum_{i=1}^{n} \lambda_i T(v_i)$ für ein $n \in \mathbb{N}$.

\subsection*{IS (n $\to$ n+1):}
\begin{align*} T\left(\sum_{i=1}^{n+1} \lambda_i v_i\right) &= T\left(\sum_{i=1}^{n} \lambda_i v_i + \lambda_{n+1} v_{n+1}\right) \\ &\overset{(i)}{=} T\left(\sum_{i=1}^{n} \lambda_i v_i\right) + T(\lambda_{n+1} v_{n+1})  \\ &\overset{IV}{=} \left(\sum_{i=1}^{n} \lambda_i T(v_i)\right) + T(\lambda_{n+1} v_{n+1}) \\ &\overset{(ii)}{=} \left(\sum_{i=1}^{n} \lambda_i T(v_i)\right) + \lambda_{n+1} T(v_{n+1})  \\ &= \sum_{i=1}^{n+1} \lambda_i T(v_i) \end{align*}

\section*{Aufgabe 4}
Das Tupel $(w_1, w_2)$ ist linear unabhängig $\iff$ ($\lambda_1 w_1 + \lambda_2 w_2 = 0_V \implies \lambda_1 = \lambda_2 = 0$).
\begin{align*} \lambda_1 w_1 + \lambda_2 w_2 = 0_V &\iff \lambda_1 (a v_1 + b v_2) + \lambda_2 (c v_1 + d v_2) = 0_V \\ &\iff (\lambda_1 a + \lambda_2 c) v_1 + (\lambda_1 b + \lambda_2 d) v_2 = 0_V\end{align*}
Da $(v_1, v_2)$ linear unabhängig ist, folgt:
\begin{align*} \lambda_1 a + \lambda_2 c &= 0 \\ \lambda_1 b + \lambda_2 d &= 0 \end{align*}
Gleichungssystem für $(\lambda_1, \lambda_2)$:
$$ \begin{pmatrix} a & c \\ b & d \end{pmatrix} \begin{pmatrix} \lambda_1 \\ \lambda_2 \end{pmatrix} = \begin{pmatrix} 0 \\ 0 \end{pmatrix} \quad \text{oder} \quad A^T \begin{pmatrix} \lambda_1 \\ \lambda_2 \end{pmatrix} = \begin{pmatrix} 0 \\ 0 \end{pmatrix} $$
Das System hat genau dann nur die triviale Lösung $(\lambda_1, \lambda_2) = (0, 0)$, wenn $A^T$ invertierbar ist. \\
$A^T$ ist invertierbar, deswegen gilt $\det(A^T) \neq 0$. \\
Da $\det(A^T) = \det(A)$, ist $A$ invertierbar und somit $\det(A) \neq 0$. \\
Also ist $(w_1, w_2)$ linear unabhängig und $\det A \neq 0$. 

\end{document}
