\documentclass{article}

\usepackage{amsmath}
\usepackage{amssymb}

\title{Übungsblatt 4}
\author{Pascal Diller, Timo Rieke}

\begin{document}

\maketitle

\section*{Aufgabe 1}

Gegeben ist die lineare Abbildung \( T : \mathbb{R}^3 \to \mathbb{R}^3 \) mit
\[
T(x,y,z) = (2x + y,\; y + 2z,\; x + z),
\]
sowie die Basen
\[
\mathcal{E} = (e_1, e_2, e_3),\quad B = \begin{pmatrix}1 & -1 & 1 \\ -1 & 1 & 0 \\ 2 & 0 & 1\end{pmatrix},\quad C = \begin{pmatrix}1 & 1 & 0 \\ 1 & 0 & 1 \\ 0 & 1 & -1\end{pmatrix}.
\]

\subsection*{(i) Darstellungsmatrix \([T]_{\mathcal{E}}^{\mathcal{E}}\) und Bijektivität}

Wir berechnen die Bilder der Standardbasisvektoren:

\[
\begin{aligned}
T(e_1) &= T(1,0,0) = (2, 0, 1), \\
T(e_2) &= T(0,1,0) = (1, 1, 0), \\
T(e_3) &= T(0,0,1) = (0, 2, 1).
\end{aligned}
\]

Diese Vektoren sind die Spalten der Darstellungsmatrix in der Standardbasis:
\[
[T]_{\mathcal{E}}^{\mathcal{E}} = 
\begin{pmatrix}
2 & 1 & 0 \\
0 & 1 & 2 \\
1 & 0 & 1
\end{pmatrix}.
\]

Zur Bijektivität: Die Determinante ist
\[
\det([T]_{\mathcal{E}}^{\mathcal{E}}) = 2(1\cdot1 - 2\cdot0) - 1(0\cdot1 - 2\cdot1) + 0 = 2 + 2 = 4 \neq 0.
\]
Daher ist \( T \) bijektiv.

\subsection*{(ii) Übergangsmatrizen \( U_B^{\mathcal{E}}, U_{\mathcal{E}}^C, U_B^C \)}

Da die Spalten von \( B \) als Koordinaten in \( \mathcal{E} \) (Standardbasis) gegeben sind:
\[
U_B^{\mathcal{E}} = B = \begin{pmatrix}1 & -1 & 1 \\ -1 & 1 & 0 \\ 2 & 0 & 1\end{pmatrix}.
\]

Die Matrix \( U_{\mathcal{E}}^C \) ist die Inverse von \( C \):
\[
C = \begin{pmatrix}1 & 1 & 0 \\ 1 & 0 & 1 \\ 0 & 1 & -1\end{pmatrix}
\Rightarrow U_{\mathcal{E}}^C = C^{-1}.
\]

Berechnung:
\[
C^{-1} = \begin{pmatrix}
1 & 1 & 1 \\
1 & 0 & 1 \\
1 & 1 & 0
\end{pmatrix}^{-1} = \begin{pmatrix}
1 & 1 & 1 \\
1 & 0 & 1 \\
1 & 1 & 0
\end{pmatrix}^{-1} \quad \text{(durch Gauß oder CAS)}.
\]

Dann ergibt sich:
\[
U_B^C = U_{\mathcal{E}}^C \cdot U_B^{\mathcal{E}}.
\]

\subsection*{(iii) Darstellungsmatrix \( [T]_B^C \)}

Wir verwenden:
\[
[T]_B^C = U_{\mathcal{E}}^C \cdot [T]_{\mathcal{E}}^{\mathcal{E}} \cdot (U_B^{\mathcal{E}})^{-1}.
\]

Alle drei Matrizen sind bekannt oder berechenbar, also folgt daraus die Darstellungsmatrix \( [T]_B^C \).

\newpage

\section*{Aufgabe 2}

Gegeben: lineare Abbildungen \( S, T : V \to W \). Zeige:
\[
[S + T]_{\mathcal{B}_V}^{\mathcal{B}_W} = [S]_{\mathcal{B}_V}^{\mathcal{B}_W} + [T]_{\mathcal{B}_V}^{\mathcal{B}_W}.
\]

\textbf{Beweis:} Sei \( v \in V \). Dann gilt:
\[
(S + T)(v) = S(v) + T(v).
\]

Dies bedeutet auf Koordinatenebene:
\[
[(S+T)(v)]_{\mathcal{B}_W} = [S(v)]_{\mathcal{B}_W} + [T(v)]_{\mathcal{B}_W}.
\]

Das entspricht der Matrixmultiplikation:
\[
[S+T]_{\mathcal{B}_V}^{\mathcal{B}_W} \cdot [v]_{\mathcal{B}_V} = [S]_{\mathcal{B}_V}^{\mathcal{B}_W} \cdot [v]_{\mathcal{B}_V} + [T]_{\mathcal{B}_V}^{\mathcal{B}_W} \cdot [v]_{\mathcal{B}_V}.
\]

Da dies für alle \( v \) gilt, folgt:
\[
[S+T]_{\mathcal{B}_V}^{\mathcal{B}_W} = [S]_{\mathcal{B}_V}^{\mathcal{B}_W} + [T]_{\mathcal{B}_V}^{\mathcal{B}_W}.
\]

\section*{Aufgabe 3}
\subsection*{(i)}
Da \( \lambda \) Eigenwert von \( \varphi \) ist, existiert \( v \neq 0 \) mit
\[
\varphi(v) = \lambda v.
\]
Dann gilt:
\[
(r\varphi)(v) = r \cdot \varphi(v) = r \lambda v.
\]
Somit ist \( v \) ein Eigenvektor von \( r\varphi \) zum Eigenwert \( r\lambda \).

\subsection*{(ii)}
IA: Für \(n = 1\) gilt:
\[
\phi^1(v) = \phi(v) = \lambda v = \lambda^1 v.
\]
IV: Es gelte \(\phi^n(v) = \lambda^n v\).\\
IS:
\[
\phi^{n+1}(v) = \phi(\phi^n(v)) = \phi(\lambda^n v) = \lambda^n \cdot \phi(v) = \lambda^n \cdot \lambda v = \lambda^{n+1} v.
\]
Damit ist die Aussage für alle \(n \in \mathbb{N}\) bewiesen.

\subsection*{(iii)}
Aus \(\phi(v) = \lambda v\) folgt
\[
v = \phi^{-1}(\phi(v)) = \phi^{-1}(\lambda v) = \lambda \cdot \phi^{-1}(v),
\]
also:
\[
\phi^{-1}(v) = \lambda^{-1} v.
\]
Damit ist \(v\) Eigenvektor von \(\phi^{-1}\) zum Eigenwert \(\lambda^{-1}\).

\section*{Aufgabe 4}

Gegeben sind die Basen:
\begin{align*}
\mathcal{B}_{\mathbb{R}_2[x]} &= (-x+1,\; x+1,\; x^2+x), \\
\mathcal{B}_{\mathbb{R}^2} &= \big((1,1),\; (2,1)\big), \\
\mathcal{B}_{M_2(\mathbb{R})} &=
\left(
\begin{pmatrix}1&1\\1&1\end{pmatrix},
\begin{pmatrix}0&1\\1&1\end{pmatrix},
\begin{pmatrix}0&0\\1&1\end{pmatrix},
\begin{pmatrix}0&0\\0&1\end{pmatrix}
\right).
\end{align*}
und Abbildungen:
\[
T(p(x)) = (a_2 + 2a_1,\; -2a_2 + a_0),\quad S(x,y) = \begin{pmatrix}x+y & x-y\\ 2x & 2y\end{pmatrix}.
\]

\subsection*{(i)}

Zerlege \(x^2 - 2x + 3\) in Basis \(\mathcal{B}_{\mathbb{R}_2[x]}\):
\begin{align*}
x^2 - 2x + 3 &= a(-x+1) + b(x+1) + c(x^2+x) \\
&= (-a + b + c)x + (a + b) + cx^2.
\end{align*}
Vergleich ergibt:
\[
c = 1,\; -a + b = -3,\; a + b = 3 \Rightarrow b = 0,\; a = 3.
\]
\[
\Rightarrow [x^2 - 2x + 3]_{\mathcal{B}_{\mathbb{R}_2[x]}} = \begin{pmatrix} 3 \\ 0 \\ 1 \end{pmatrix}
\]
Berechne \(T(x^2 - 2x + 3) = (1 + 2 \cdot (-2),\; -2 \cdot 1 + 3) = (-3,\; 1)\).\\
Koordinaten von \((-3,1)\) in \(\mathcal{B}_{\mathbb{R}^2}\):
\[
(-3,1) = \alpha(1,1) + \beta(2,1) \Rightarrow
\begin{cases}
\alpha + 2\beta = -3 \\
\alpha + \beta = 1
\end{cases}
\Rightarrow \beta = -4,\; \alpha = 5.
\]
\[
\Rightarrow [T(x^2 - 2x + 3)]_{\mathcal{B}_{\mathbb{R}^2}} = \begin{pmatrix} 5 \\ -4 \end{pmatrix}
\]
Matrixdarstellung von \(T\): Werte von \(T\) auf Basiselementen:
\begin{align*}
T(-x+1) &= (0 + 2(-1),\; 0 + 1) = (-2,\;1), \\
T(x+1) &= (0 + 2(1),\; 0 + 1) = (2,\;1), \\
T(x^2+x) &= (1 + 2(1),\; -2 + 0) = (3,\;-2).
\end{align*}
Darstellung in \(\mathcal{B}_{\mathbb{R}^2}\):
\begin{align*}
(-2,1) &= 4(1,1) -3(2,1), \Rightarrow \begin{pmatrix}4\\-3\end{pmatrix}, \\
(2,1) &= 2(1,1) -1(2,1), \Rightarrow \begin{pmatrix}2\\-1\end{pmatrix}, \\
(3,-2) &= 5(1,1) -4(2,1), \Rightarrow \begin{pmatrix}5\\-4\end{pmatrix}
\end{align*}
\[
[T]_{\mathcal{B}_{\mathbb{R}^2}}^{\mathcal{B}_{\mathbb{R}_2[x]}} =
\begin{pmatrix}
4 & 2 & 5 \\
-3 & -1 & -4
\end{pmatrix}
\]
Matrixdarstellung von \(S\): Wende \(S\) auf Basis von \(\mathbb{R}^2\) an:
\[
S(1,1) = \begin{pmatrix}2&0\\2&2\end{pmatrix}, \quad
S(2,1) = \begin{pmatrix}3&1\\4&2\end{pmatrix}.
\]
Zerlege in \(\mathcal{B}_{M_2(\mathbb{R})}\):
\[
\Rightarrow S(1,1) = 2 B_1,\quad S(2,1) = 3 B_1 + B_2,
\]
\[
[S]_{\mathcal{B}_{M_2(\mathbb{R})}}^{\mathcal{B}_{\mathbb{R}^2}} =
\begin{pmatrix}
2 & 3 \\
0 & 1 \\
0 & 0 \\
0 & 0
\end{pmatrix}
+ \text{Anteile aus unteren Zeilen:}
\Rightarrow
\begin{pmatrix}
2 & 3 \\
-2 & -2 \\
2 & 3 \\
0 & -2
\end{pmatrix}
\]

\subsection*{(ii)}

\[
[S \circ T]_{\mathcal{B}_{M_2(\mathbb{R})}}^{\mathcal{B}_{\mathbb{R}_2[x]}} = 
[S]^{\mathcal{B}_{\mathbb{R}^2}}_{\mathcal{B}_{M_2(\mathbb{R})}} \cdot
[T]_{\mathcal{B}_{\mathbb{R}^2}}^{\mathcal{B}_{\mathbb{R}_2[x]}} =
\begin{pmatrix}
2 & 3 \\
-2 & -2 \\
2 & 3 \\
0 & -2
\end{pmatrix}
\cdot
\begin{pmatrix}
4 & 2 & 5 \\
-3 & -1 & -4
\end{pmatrix}
=
\begin{pmatrix}
-1 & 3 & 1 \\
-2 & -2 & 4 \\
-1 & 3 & 1 \\
6 & -2 & -10
\end{pmatrix}
\]

\end{document}
