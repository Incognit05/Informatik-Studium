\documentclass{article}

\usepackage{amsmath}
\usepackage{amssymb}

\title{Übungsblatt 4}
\author{Pascal Diller, Timo Rieke}

\begin{document}

\maketitle

\section*{Aufgabe 1}

Gegeben ist die lineare Abbildung \( T : \mathbb{R}^3 \to \mathbb{R}^3 \) mit
\[
T(x,y,z) = (2x + y,\; y + 2z,\; x + z),
\]
sowie die Basen
\[
\mathcal{E} = (e_1, e_2, e_3),\quad B = \begin{pmatrix}1 & -1 & 1 \\ -1 & 1 & 0 \\ 2 & 0 & 1\end{pmatrix},\quad C = \begin{pmatrix}1 & 1 & 0 \\ 1 & 0 & 1 \\ 0 & 1 & -1\end{pmatrix}.
\]

\subsection*{(i) Darstellungsmatrix \([T]_{\mathcal{E}}^{\mathcal{E}}\) und Bijektivität}

Wir berechnen die Bilder der Standardbasisvektoren:

\[
\begin{aligned}
T(e_1) &= T(1,0,0) = (2, 0, 1), \\
T(e_2) &= T(0,1,0) = (1, 1, 0), \\
T(e_3) &= T(0,0,1) = (0, 2, 1).
\end{aligned}
\]

Diese Vektoren sind die Spalten der Darstellungsmatrix in der Standardbasis:
\[
[T]_{\mathcal{E}}^{\mathcal{E}} = 
\begin{pmatrix}
2 & 1 & 0 \\
0 & 1 & 2 \\
1 & 0 & 1
\end{pmatrix}.
\]

Zur Bijektivität: Die Determinante ist
\[
\det([T]_{\mathcal{E}}^{\mathcal{E}}) = 2(1\cdot1 - 2\cdot0) - 1(0\cdot1 - 2\cdot1) + 0 = 2 + 2 = 4 \neq 0.
\]
Daher ist \( T \) bijektiv.

\subsection*{(ii) Übergangsmatrizen \( U_B^{\mathcal{E}}, U_{\mathcal{E}}^C, U_B^C \)}

Da die Spalten von \( B \) als Koordinaten in \( \mathcal{E} \) (Standardbasis) gegeben sind:
\[
U_B^{\mathcal{E}} = B = \begin{pmatrix}1 & -1 & 1 \\ -1 & 1 & 0 \\ 2 & 0 & 1\end{pmatrix}.
\]

Die Matrix \( U_{\mathcal{E}}^C \) ist die Inverse von \( C \):
\[
C = \begin{pmatrix}1 & 1 & 0 \\ 1 & 0 & 1 \\ 0 & 1 & -1\end{pmatrix}
\Rightarrow U_{\mathcal{E}}^C = C^{-1}.
\]

Berechnung:
\[
C^{-1} = \begin{pmatrix}
1 & 1 & 1 \\
1 & 0 & 1 \\
1 & 1 & 0
\end{pmatrix}^{-1} = \begin{pmatrix}
1 & 1 & 1 \\
1 & 0 & 1 \\
1 & 1 & 0
\end{pmatrix}^{-1} \quad \text{(durch Gauß oder CAS)}.
\]

Dann ergibt sich:
\[
U_B^C = U_{\mathcal{E}}^C \cdot U_B^{\mathcal{E}}.
\]

\subsection*{(iii) Darstellungsmatrix \( [T]_B^C \)}

Wir verwenden:
\[
[T]_B^C = U_{\mathcal{E}}^C \cdot [T]_{\mathcal{E}}^{\mathcal{E}} \cdot (U_B^{\mathcal{E}})^{-1}.
\]

Alle drei Matrizen sind bekannt oder berechenbar, also folgt daraus die Darstellungsmatrix \( [T]_B^C \).

\newpage

\section*{Aufgabe 2}

Gegeben: lineare Abbildungen \( S, T : V \to W \). Zeige:
\[
[S + T]_{\mathcal{B}_V}^{\mathcal{B}_W} = [S]_{\mathcal{B}_V}^{\mathcal{B}_W} + [T]_{\mathcal{B}_V}^{\mathcal{B}_W}.
\]

\textbf{Beweis:} Sei \( v \in V \). Dann gilt:
\[
(S + T)(v) = S(v) + T(v).
\]

Dies bedeutet auf Koordinatenebene:
\[
[(S+T)(v)]_{\mathcal{B}_W} = [S(v)]_{\mathcal{B}_W} + [T(v)]_{\mathcal{B}_W}.
\]

Das entspricht der Matrixmultiplikation:
\[
[S+T]_{\mathcal{B}_V}^{\mathcal{B}_W} \cdot [v]_{\mathcal{B}_V} = [S]_{\mathcal{B}_V}^{\mathcal{B}_W} \cdot [v]_{\mathcal{B}_V} + [T]_{\mathcal{B}_V}^{\mathcal{B}_W} \cdot [v]_{\mathcal{B}_V}.
\]

Da dies für alle \( v \) gilt, folgt:
\[
[S+T]_{\mathcal{B}_V}^{\mathcal{B}_W} = [S]_{\mathcal{B}_V}^{\mathcal{B}_W} + [T]_{\mathcal{B}_V}^{\mathcal{B}_W}.
\]

\section*{Aufgabe 3}

Sei \( \varphi : V \to V \) ein Endomorphismus und \( \lambda \in \mathbb{R} \) ein Eigenwert von \( \varphi \). Zeige:

\subsection*{(i) \( r\lambda \) ist Eigenwert von \( r\varphi \) für \( r \in \mathbb{R} \)}

Da \( \lambda \) Eigenwert von \( \varphi \) ist, existiert \( v \neq 0 \) mit
\[
\varphi(v) = \lambda v.
\]
Dann gilt:
\[
(r\varphi)(v) = r \cdot \varphi(v) = r \lambda v.
\]
Somit ist \( v \) ein Eigenvektor von \( r\varphi \) zum Eigenwert \( r\lambda \).


\end{document}
