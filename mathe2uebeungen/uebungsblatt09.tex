\documentclass{article}
\usepackage{graphicx} % Required for inserting images
\usepackage{amsmath}
\usepackage{amssymb}

\title{Übungsblatt 09}
\author{Pascal Diller, Timo Rieke}

\begin{document}
\maketitle
\section*{Aufgabe 1}
\subsection*{(i)}
\subsubsection*{(a)}
Wir klammern die höchste Potenz (\(x^3\)) aus:
\[\lim_{x\to\infty} \frac{x^3(6 - \frac{11}{x} - \frac{7}{x^2} - \frac{8}{x^3})}{x^3(2 + \frac{1}{x} + \frac{3}{x^2} + \frac{7}{x^3})} = \lim_{x\to\infty} \frac{6 - \frac{11}{x} - \frac{7}{x^2} - \frac{8}{x^3}}{2 + \frac{1}{x} + \frac{3}{x^2} + \frac{7}{x^3}}\]
Mit den Rechenregeln für Grenzwerte und da \(\lim_{x\to\infty} \frac{c}{x^n} = 0\) für \(n>0\), folgt:
\[\frac{6 - 0 - 0 - 0}{2 + 0 + 0 + 0} = \frac{6}{2} = 3\]

\subsubsection*{(b)}
Umformen der Terme, dann ausklammern mit (\(\sqrt{x} = x^{\frac{1}{2}}\)):
\[\lim_{x\searrow0} \frac{x^{\frac{3}{2}} - 15x - 2x^{\frac{1}{2}}}{x^{\frac{1}{2}} + 7x + x^4} = \lim_{x\searrow0} \frac{x^{\frac{1}{2}}(x - 15x^{\frac{1}{2}} - 2)}{x^{\frac{1}{2}}(1 + 7x^{\frac{1}{2}} + x^{\frac{7}{2}})} = \lim_{x\searrow0} \frac{x - 15\sqrt{x} - 2}{1 + 7\sqrt{x} + x^3\sqrt{x}}\]
Durch Einsetzen von \(x=0\) erhalten wir:
\[\frac{0 - 15\cdot0 - 2}{1 + 7\cdot0 + 0} = -2\]

\subsubsection*{(c)}
Da die Sinusfunktion beschränkt ist, gilt:
\[-1 \le \sin(x^2) \le 1\]
Für \(x \ne 0\) teilen wir durch \( x^2 \):
\[-\frac{1}{x^2} \le \frac{\sin(x^2)}{x^2} \le \frac{1}{x^2}\]
Da \(\lim_{x\to\infty} -\frac{1}{x^2} = 0\) und \(\lim_{x\to\infty} \frac{1}{x^2} = 0\), folgt:
\[\lim_{x\to\infty} \frac{\sin(x^2)}{x^2} = 0\]

\subsubsection*{(d)}
Wir formen den Ausdruck um:
\[\lim_{x\to0} \left(\frac{\sin x}{x}\right)^2 = \left(\lim_{x\to0} \frac{\sin x}{x}\right) \cdot \left(\lim_{x\to0} \frac{\sin x}{x}\right)\]
\(\lim_{x\to0} \frac{\sin x}{x} = 1\), deswegen folgt:
\[1 \cdot 1 = 1\]

\subsubsection*{(e)}
Wir erweitern mit der konjugierten Form:
\[\lim_{x\to\infty} \frac{(\sqrt{x} - \sqrt{x+10})(\sqrt{x} + \sqrt{x+10})}{\sqrt{x} + \sqrt{x+10}} = \lim_{x\to\infty} \frac{x - (x+10)}{\sqrt{x} + \sqrt{x+10}} = \lim_{x\to\infty} \frac{-10}{\sqrt{x} + \sqrt{x+10}}\]
Da der Nenner gegen \(\infty\) geht, geht der gesamte Bruch gegen 0.
\[\lim_{x\to\infty} (\sqrt{x} - \sqrt{x+10}) = 0\]

\subsection*{(ii)}
\subsubsection*{(a)}
Wir formen den Ausdruck in der Klammer durch Erweitern mit der Konjugierten um:
\[\sqrt{x}(\sqrt{x+1} - \sqrt{x}) = \sqrt{x} \frac{(\sqrt{x+1} - \sqrt{x})(\sqrt{x+1} + \sqrt{x})}{\sqrt{x+1} + \sqrt{x}} = \sqrt{x} \frac{x+1-x}{\sqrt{x+1} + \sqrt{x}} = \frac{\sqrt{x}}{\sqrt{x+1} + \sqrt{x}}\]
Nun klammern wir im Nenner \(\sqrt{x}\) aus:
\[\frac{\sqrt{x}}{\sqrt{x}(\sqrt{1+\frac{1}{x}} + 1)} = \frac{1}{\sqrt{1+\frac{1}{x}} + 1}\]
Damit ergibt sich der Grenzwert:
\[\lim_{x\to\infty} \frac{1}{\sqrt{1+\frac{1}{x}} + 1} = \frac{1}{\sqrt{1+0} + 1} = \frac{1}{2}\]

\subsubsection*{(b)}
Für den linksseitigen Grenzwert gilt: Wenn \(x \nearrow 0\), dann \(x<0\) und \(\frac{1}{x} \to -\infty\).
Da \(\lim_{y\to-\infty} \exp(y) = 0\), folgt:
\[\lim_{x\nearrow0} \exp\left(\frac{1}{x}\right) = 0\]

\subsubsection*{(c)}
Für den rechtsseitigen Grenzwert gilt: Wenn \(x \searrow 0\), dann \(x>0\) und \(\frac{1}{x} \to \infty\).
Da \(\lim_{y\to\infty} \exp(y) = \infty\), folgt:
\[\lim_{x\searrow0} \exp\left(\frac{1}{x}\right) = \infty\]

\section*{Augabe 2}
\subsection*{Stetigkeit für \(x \notin \{0, 1\}\)}
Für \(x < 0\): \(f(x) = 2\sqrt{1-x} - \cos(x)\) ist als Komposition und Differenz stetiger Funktionen stetig. \\
Für \(0 < x < 1\): \(f(x) = x \cdot \sin(\frac{\pi}{x})\) ist als Produkt und Komposition stetiger Funktionen stetig. \\
Für \(x > 1\): \(f(x) = 1 - \frac{\sin(\pi(x-1))}{x-1}\) ist als Komposition, Quotient und Differenz stetiger Funktionen stetig.

\subsection*{Untersuchung bei \(x = 0\)}
Eine Funktion ist in \(x_0\) stetig, wenn \(\lim_{x \to x_0} f(x) = f(x_0)\) gilt, was erfordert, dass links- und rechtsseitiger Grenzwert existieren und mit \(f(x_0)\) übereinstimmen.
Funktionswert:
\[f(0) = 2\sqrt{1-0} - \cos(0) = 2 - 1 = 1\]
Linksseitiger Grenzwert:
\[\lim_{x\nearrow0} f(x) = \lim_{x\nearrow0} (2\sqrt{1-x} - \cos(x)) = 2\sqrt{1} - \cos(0) = 2 - 1 = 1\]
Rechtsseitiger Grenzwert:
\[\lim_{x\searrow0} f(x) = \lim_{x\searrow0} x \cdot \sin\left(\frac{\pi}{x}\right)\]
Da \(-1 \le \sin(\frac{\pi}{x}) \le 1\), gilt für \(x > 0\):
\[-x \le x \sin\left(\frac{\pi}{x}\right) \le x\]
Mit \(\lim_{x\searrow0} (-x) = 0\) und \(\lim_{x\searrow0} x = 0\) folgt aus dem Sandwichsatz:
\[\lim_{x\searrow0} x \sin\left(\frac{\pi}{x}\right) = 0\]
Da \(\lim_{x\nearrow0} f(x) = 1 \ne 0 = \lim_{x\searrow0} f(x)\), existiert der Grenzwert \(\lim_{x\to0} f(x)\) nicht. Die Funktion \(f\) ist an der Stelle \(x=0\) nicht stetig.

\subsection*{Untersuchung bei \(x = 1\)}
Funktionswert:
\[f(1) = 1 \cdot \sin\left(\frac{\pi}{1}\right) = \sin(\pi) = 0\]
Linksseitiger Grenzwert:
\[\lim_{x\nearrow1} f(x) = \lim_{x\nearrow1} x \sin\left(\frac{\pi}{x}\right) = 1 \cdot \sin(\pi) = 0\]
Rechtsseitiger Grenzwert:
\[\lim_{x\searrow1} f(x) = \lim_{x\searrow1} \left(1 - \frac{\sin(\pi(x-1))}{x-1}\right)\]
Wir substituieren \(h = x-1\). Wenn \(x \searrow 1\), dann \(h \searrow 0\).
\[\lim_{h\searrow0} \left(1 - \frac{\sin(\pi h)}{h}\right) = 1 - \lim_{h\searrow0} \frac{\pi \sin(\pi h)}{\pi h} = 1 - \pi \cdot \lim_{h\searrow0} \frac{\sin(\pi h)}{\pi h}\]
Mit der Substitution \(y = \pi h\) und dem bekannten Grenzwert \(\lim_{y\to0} \frac{\sin y}{y} = 1\) folgt:
\[1 - \pi \cdot 1 = 1 - \pi\]
Da \(\lim_{x\nearrow1} f(x) = 0 \ne 1 - \pi = \lim_{x\searrow1} f(x)\), existiert der Grenzwert \(\lim_{x\to1} f(x)\) nicht. Die Funktion \(f\) ist an der Stelle \(x=1\) nicht stetig \\
\(\longrightarrow\)Die Funktion \(f\) ist stetig für alle \(x \in \mathbb{R} \setminus \{0, 1\}\).

\end{document}
