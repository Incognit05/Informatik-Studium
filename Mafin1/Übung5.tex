\documentclass{article}
\usepackage[utf8]{inputenc}
\usepackage{amsmath, amssymb, amsthm}

\title{Übung 5}
\author{Pascal Diller, Timo Rieke}

\setcounter{secnumdepth}{0}

\begin{document}

\maketitle
\section{Aufgabe 1}
\subsection{(i)}
\textbf{I.A.}
\[\left|\sum_{k=1}^{1} a_k\right| \leq \sum_{k=1}^{1} |a_k| = 1 \leq 1\]
\textbf{I.V.}
\[\left|\sum_{k=1}^{n} a_k\right| \leq \sum_{k=1}^{n}|a_k|\]
\textbf{I.S}
\begin{align*}
    |\sum_{k=1}^{n+1} a_k| &\leq \sum_{k=1}^{n+1} |a_k| \\
    = \left|\left(\sum_{k=1}^{n}a_k\right) + a_{n+1}\right| &\leq \left(\sum_{k=1}^{n}|a_k|\right) + |a_{n+1}| \\
    \overset{I.V.}{=} \left|\sum_{k=1}^{n+1} a_k\right| &\leq \sum_{k=1}^{n}|a_k| + |a_{n+1}| \\
    = \left|\sum_{k=1}^{n+1}a_k\right| &\leq \sum_{k=1}^{n+1}|a_k| 
\end{align*}
Somit ist gezeigt, dass $A(n) \implies A(n+1)$.
\newpage
\subsection{(ii)}
Zu zeigen: \[\sum_{k=0}^{n}(-1)^k \begin{pmatrix} n\\k \end{pmatrix} = 0\]
Binomischer Lehrsatz: \[(x+y)^n = \sum_{k=0}^{n} \begin{pmatrix}n\\k \end{pmatrix} x^k y^{n-k}\]
Seien $x = -1$ und $y = 1$:
\begin{align*}
    (-1 + 1)^n &= \sum_{k=0}^{n} \begin{pmatrix}n\\k\end{pmatrix}(-1)^k 1^{n-k} \\
    = 0^n &= \sum_{k=0}^{n} \begin{pmatrix}n\\k\end{pmatrix} (-1)^k \\
    = 0 &= \sum_{k=0}^{n}(-1)^k \begin{pmatrix}n\\k\end{pmatrix}
\end{align*}

\section{Aufgabe 2}
\subsection{(i) $f,g$ beschränkt $\Longrightarrow f*g$ beschränkt}
\begin{center}
    Wenn f ung g beschränkt sind, gib es Konstanten $M_f, M_g > 0$, sodass \\
    $|f(x)| \leq M_f$ und $|g(x)| \leq M_g$ für alle $x \in \mathbb{R}$ \\
    Für das Produkt $f*g$ gilt dann: \\
    $|f(x)*g(x)| \leq |f(x)|*|g(x)| \leq M_f*M_g$ \\
    $\Longrightarrow f*g$ ist ebenfalls beschränkt.
\end{center}

\subsection{(ii) $f,g$ monoton wachsend $\Longrightarrow f*g$ monoton wachsend}
\begin{center}
    \textbf{Gegenbeispiel:} \\
    Seien $f(x)=x$ und $g(x)=x-1$ \\
    Beide sind monoton wachsend, da $f'(x)=1 > 0$ und $g'(x)=1 > 0$. \\
    \textbf{$f*g$:} $(f*g)(x)=x*(x-1)=x^2-x$ \\
    Ableitung: $(f*g)'(x)=2x-1$\\
    $\Longrightarrow$ Da für $x<\frac{1}{2}$ $(f*g)'(x)<0$ ist und für  $x>\frac{1}{2}$ $(f*g)'(x)>0$ ist, ist $f*g$ nicht monoton wachsend.
\end{center}

\section{Aufgabe 3}
\subsection{(i) Untersuchen auf (strenge) Monotonie}
\begin{center}
$f(x)=x^3$ \\
Ableitung: $f'(x)=3x^2$ \\
$f'(x) = 0$ bei $x=0$ \\
$f'(x) > 0$ bei $x \neq 0$ \\
$\Longrightarrow$ Da $f'(x)>0$ überall außer an der Stelle x=0 ist, ist f(x) streng monton wachsend.
\end{center}

\begin{center}
$g(x)=x^4$ \\
Ableitung: $f'(x)=4x^3$ \\
$g'(x) = 0$ bei $x=0$ \\
$g'(x) > 0$ bei $x > 0$ \\
$g'(x) < 0$ bei $x < 0$ \\
$\Longrightarrow$ Da $g'(x)$ sein Vorzeichen wechselt, ist $g(x)$ nicht streng monoton
\end{center}

\subsection{(ii) Betrachten der Funktion $f:(0,\infty)\xrightarrow{}(0,\infty), x \mapsto \frac{1}{x}$}
\subsubsection{a)}
Es gilt: $\forall x_1, x_2 \in (0, \infty): f(x_1) = f(x_2) \to x_1 = x_2$. Somit ist $f$ injektiv. \\ 
Da der Definitionsbereich gleich dem Wertebereich ist, gilt: $\forall y \in (0,\infty) \exists x = y^{-1}: f(x) = y$. Somit ist $f$ surjektiv. \\
Da $f$ bijektiv ist, besitzt $f$ auch eine Umkehrfunktion.

\subsubsection{b)}
\[y = f(x) = \frac{1}{x} \Longleftrightarrow x = f^{-1}(y) = \frac{1}{y}\]
Der Definitionsbereich von $f^{-1}$ ist der Wertebereich von $f$. \\
Der Wertebereich von $f^{-1}$ ist der Definitionsbereich von $f$. \\
\[\implies f^{-1}: (0,\infty) \to (0, \infty), y \mapsto \frac{1}{y}\]

\section{Aufgabe 4}
\subsection{(i) nach unten durch 1 beschränkt ist}
\subsubsection{Induktionsanfang}
Für $n = 1$:
\begin{center}
    $f(1) = 2 \longrightarrow f(1) \geq 1$
\end{center}
\subsubsection{Induktionsvoraussetzung}
\begin{center}
$f(k) \geq 1$ gilt für ein $k \in \mathbb{N}$ \\
\end{center}
\subsubsection{Induktionsschluss}
Wir zeigen, dass $f(k+1) \geq 1$ \\
Die Rekursionsgleichung lautet:\\
\begin{center}
$f(k+1) = 2- \frac{2}{f(k)+2}$ \\
\end{center}
Da $f(k) \geq 1$, folgt: \\
\begin{center}
$f(k)+2 \geq 3 \Longrightarrow \frac{2}{f(k)+2} \geq \frac{2}{3}$ \\
\end{center}
Somit gilt: \\
\begin{center}
$f(k+1)=2-\frac{2}{f(k)+2} \geq 2-\frac{2}{3} = \frac{4}{3}$ \\
\end{center}
Da $\frac{4}{3}>1$, folgt $f(k+1) \geq 1$ \\

$\Longrightarrow$ Nach vollständiger Induktion ist $f(n) \geq 1$ für alle $n \in \mathbb{N}$


\subsection{(ii)}
$f(1) = 2$, $f(n+1) = 2 - \frac{2}{f(n)+2}$ \\
\newline
Zu zeigen: Es gibt $x \leq y$ so dass $f(x) \geq f(y)$
\subsubsection{Induktionsanfang}
Sei $n = 1$.
\begin{align*}
    f(2) &\geq f(3) \\
    = 2 - \frac{2}{f(1) + 2} &\geq 2 - \frac{2}{f(2) + 2} \\
    = 2 - \frac{2}{4} &\geq 2 - \frac{2}{2- \frac{2}{4} + 2} \\
    = \frac{3}{2} &\geq 2 - \frac{2}{\frac{3}{2} + 2} \\
    = \frac{3}{2} &\geq \frac{10}{7}
\end{align*}

\subsubsection{Induktionsvoraussetzung}
$f(n + 1) \geq f(n + 2)$ für $n \in \mathbb{N}$ \\
$\implies f(n) \geq f(n + 1)$ für $n \in \mathbb{N}$

\subsubsection{Induktionsschluss}
Man untersucht die Differenz $f(n + 1) - f(n + 2)$. Wenn $f(n + 1) - f(n + 2) \geq 0$, dann gilt $f(n + 1) \geq f(n + 2)$
\begin{align*}
    f(n + 1) &- f(n + 2) \\
    = \left(2 - \frac{2}{f(n) + 2}\right) &- \left(2 - \frac{2}{f(n + 1) + 2}\right) \\
    = 2 - \frac{2}{f(n)+2} &- 2 + \frac{2}{f(n+1) + 2} \\
    = \frac{2}{f(n+1) + 2} &- \frac{2}{f(n) + 2}
\end{align*}
Aus der I.V. gilt: $f(n) \geq f(n + 1)$. Demnach ist $f(n) + 2 \geq f(n + 1) + 2$. \\
Daraus folgt, dass $\frac{2}{f(n+1) + 2} \geq \frac{2}{f(n) + 2}$ \\
Also ist $\frac{2}{f(n+1)+2} - \frac{2}{f(n)+2} = f(n + 1) - f(n + 2) \geq 0$ \\
Damit ist gezeigt, dass $f(n + 1) \geq f(n + 2)$.

\end{document}
