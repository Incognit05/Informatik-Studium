\documentclass{article}
\usepackage[utf8]{inputenc}
\usepackage{amsmath, amssymb, amsthm}

\title{Übung 3}
\author{Pascal Diller, Timo Rieke}

\setcounter{secnumdepth}{0}

\begin{document}

\maketitle

\section{1}
\subsection{(i)}
\subsubsection{(a)}
Für $m$ gerade: \\
\[m = 2k \text{ für } k \in \mathbb{N}\]
\[f(m) = \frac{2k}{2} = k\]
$\implies$ $k$ kann alle natürlichen Zahlen annehmen. Somit sind die Werte bei $f(m)$ für $m$ gerade alle in $\mathbb{N}$. \\
\newline
Für $m$ gerade: \\
\[m = 2k + 1 \text{ für } k \in \mathbb{N}\]
\[f(m) = \frac{(2k + 1) - 1}{2} = \frac{2k}{2} = k\]
$\implies k$ kann wieder $\mathbb{N}$ annehmen. \\
\newline
$\implies$ Da $f(m)$ alle natürlichen Zahlen abbildet ist $f$ surjektiv auf $\mathbb{N}$, jedoch nicht auf $\mathbb{Z}$, da keine negativen Zahlen erreicht werden.
\subsubsection{(b)}
Für $z \in \mathbb{N}$: \\
gerade: $m = 2z$ \\
ungerade: $m = 2z+1$ \\
jedes $z \in \mathbb{N}$ hat 2 Werte von $m$(gerade, ungerade), die auf $z$ abgebildet werden. \\
Kardinalität $|f^{-1}(\{z\})| = 2$ für $z \in \mathbb{N}$ \\
\newline
Für $z < 0$ \\
Da $f(m)$ nur Werte aus $\mathbb{N}$ annimmt, gibt es kein $m$, das auf ein negatives $z$ abgebildet wird. \\
Daher: $|f^{-1}(\{z\})| = 0$ für $z < 0$ \\
\newline
Für $z=0$ \\
Da $f(m)$ immer einen positiven Wert ergibt, kann 0 von $f(m)$ ebenfalls nicht erreicht werden. \\
Somit: $|f^{-1}(\{z\})| = 0$\\
\newline
Für $z \in \mathbb{N}: |f^{-1}(\{z\})| = 2$ \\
Für $z \leq 0: |f^{-1}(\{z\})| = 0$ \\
\newline
Injektivität; Da jedes $z \in \mathbb{N}$ zwei Urbilder hat, ist $f$ nicht injektiv.
\subsubsection{(c)}
$M = \{2n | n \in \mathbb{N}\}$ (Menge der geraden Zahlen) \\
$m$ ist eine gerade Zahl ($m \in M$) \\
$m = 2k, k \in \mathbb{N}$ \\
$\implies f(m) = \frac{m}{2} = k$ \\
Jeder Wert von $k$ wird auf genau einen Wert $m = 2k$ abgebildet, sodass für jedes Bild $f(m)$ genau ein Urbild existiert. \\
Da $f|_M$ alle natürlichen Zahlen erreicht, ist $f$ surjektiv auf $\mathbb{N}$. \\
Somit ist $M(2n | n \in \mathbb{N})$ eine Teilmenge, bei der $f|_M: M \to f(\mathbb{N})$ bijektiv ist.
\subsection{(ii)}
\subsubsection{(a)}
$g: \mathbb{N} \times \mathbb{N} \to \mathbb{N}, (n,m) \mapsto nm$ \\
Bild von $(3, 11)$: $g(3,11) = 3 \cdot 11 = 33$ \\
\newline
Urbild von $\{10\}$: $g(n,m)=10 \implies n \cdot m = 10$ \\
mögliche Paare für $(n,m): (1,10),(2,5),(10,1),(5,2)$ \\
Urbild: $\{(1,10), (2,5), (5,2), (10,1)\}$
\subsubsection{(b)}
Injektivität: \\
$g(n_1, m_2) = g(n_2, m_2)$ für $(n_1,m_1) \neq (n_2, m_2)$ \\
Bsp: $g(1,10) = 10 = g(2,5)$, somit ist $g$ nicht injektiv. \\
Surjektiv: \\
Für jedes $k \in \mathbb{N}$ kann $g(1,k) = k$ verwendet werden um alle Werte in $\mathbb{N}$ zu erreichen. Somit ist $g$ surjektiv.

\section{2}
\subsection{(i)}
\[A: \forall x \in \mathbb{N}, x > 1 \land (d|x \to d = 1 \lor d = x)\]
\subsection{(ii)}
$\neg A$: Es gibt mindestens eine Natürliche Zahl, die keine Primzahl ist.
\[\neg A: \exists x \in \mathbb{N}, d|x \to d \neq 1 \land d \neq x\]
Die Aussage $A$ ist falsch, da es auch Natürliche Zahlen gibt, die nicht nur 1 und sich selber als Teiler haben.

\section{3}
\subsection{(i)}
\[\neg((Q \lor R) \to P) \sim ((Q \land \neg P) \lor (R \land \neg P))\]
Implikation:
\[(Q \lor R) \to P \sim \neg(Q \lor R) \lor P\]
\[\neg((Q \lor R) \to P) \sim \neg(\neg(Q \lor R) \lor P)\]
De Morgan Gesetz:
\[\neg(\neg(Q \lor R) \lor P) \sim (Q \lor R) \lor \neg P\]
Distributivgesetz:
\[(Q \lor R) \land \neg P \sim (Q \land \neg P) \lor (R \land \neg P)\]
Somit ist bewiesen, dass die Formeln logisch äquivalent sind.

\subsection{(ii)}
\[\left( ((P \to R) \to (R \to Q)) \to (P \to Q) \right) \mapsto 0\]

\end{document}