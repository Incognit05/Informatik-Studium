\documentclass{article}
\usepackage{amsmath}
\usepackage{amssymb}
\title{Mathe Übung 12}
\author{Pascal Diller, Timo Rieke}
\begin{document}
\maketitle

\section*{Aufgabe 1}

\subsection*{(i)}
\[\lim_{n \to \infty} \frac{7n^{3/2} - n^{-2} \sqrt{n}}{(\sqrt{n})^3 + n^{1/2} + n}.\]
Im Zähler dominiert \(7n^{3/2}\), da \(n^{-2} \sqrt{n}\) für \(n \to \infty\) vernachlässigbar ist. \\
Im Nenner dominiert \(n\), da \(n > n^{3/2} > n^{1/2}\) für große \(n\). \\
Es ergibt sich:
\[\lim_{n \to \infty} \frac{7n^{3/2}}{n} = 7\sqrt{n} \to \infty.\]

\subsection*{(ii)}
\[\lim_{n \to \infty} \sqrt{n^4 + n^2 + 1} - \sqrt{n^2 - 1}.\]
\[\sqrt{n^4 + n^2 + 1} - \sqrt{n^2 - 1} = \frac{(n^4 + n^2 + 1) - (n^2 - 1)}{\sqrt{n^4 + n^2 + 1} + \sqrt{n^2 - 1}}.\]
\[n^4 + n^2 + 1 - n^2 + 1 = n^4 + 2.\]
Im Nenner dominiert \(\sqrt{n^4} = n^2\), daher:
\[\frac{n^4 + 2}{\sqrt{n^4 + n^2 + 1} + \sqrt{n^2 - 1}} \sim \frac{n^4}{2n^2} =\frac{n^2}{2}.\]

\subsection*{(iii)} 
\[\lim_{n \to \infty} \frac{n}{\sqrt{n^5 + 4n}}.\]
Im Nenner dominiert \(\sqrt{n^5} = n^{5/2}\). Somit:
\[\frac{n}{\sqrt{n^5 + 4n}} \sim \frac{n}{n^{5/2}} = n^{-3/2}.\]
Daraus folgt:
\[\lim_{n \to \infty} \frac{n}{\sqrt{n^5 + 4n}} = 0.\]

\subsection*{(iv)} 
\[\lim_{n \to \infty} \left(\frac{\sqrt{n} - 1}{n}\right)^n.\]
Sei \(L = \left(\frac{\sqrt{n} - 1}{n}\right)^n\), Logarithmieren:
\[\ln L = n \ln\left(\frac{\sqrt{n} - 1}{n}\right).\]
Annäherung:
\[\ln\left(\frac{\sqrt{n} - 1}{n}\right) = \ln(\sqrt{n} - 1) - \ln(n) \sim \ln(\sqrt{n}) - \ln(n) = -\frac{\ln(n)}{2}.\]
Somit:
\[\ln L = n \cdot \left(-\frac{\ln(n)}{2}\right) = -\frac{n \ln(n)}{2} \to -\infty.\]
Daraus folgt:
\[L \to 0.\]

\section*{Aufgabe 2}
\[a_1 = 0, \quad a_{n+1} = \frac{1}{2}a_n + \frac{1}{2}.\]

\subsection*{(i)}  
Zu zeigen: \(a_n \leq 1\) für alle \(n \in \mathbb{N}\).  \\
Induktionsanfang: Für \(n = 1\) gilt \(a_1 = 0 \leq 1\). \\
Induktionsschritt: Angenommen, \(a_n \leq 1\). Dann folgt:
    \[a_{n+1} = \frac{1}{2}a_n + \frac{1}{2} \leq \frac{1}{2}(1) + \frac{1}{2} =1.\]
Damit ist \(a_n \leq 1\) für alle \(n\).

\subsection*{(ii)}  
Zu zeigen: \(a_{n+1} \geq a_n\).  
\[a_{n+1} - a_n = \frac{1}{2}a_n + \frac{1}{2} - a_n = \frac{1}{2}(1 - a_n).\]
Da \(a_n \leq 1\), ist \(1 - a_n \geq 0\) und somit \(a_{n+1} \geq a_n\). Somit ist \((a_n)_{n\in \mathbb{N}}\) weiter monton wachsend.

\subsection*{(iii)}  
Da die Folge beschränkt und monoton wachsend ist folgt, dass \(a_n\) konvergiert. Sei:
\[\lim_{n \to \infty} a_n = L.\]
Im Limes gilt:
\[L = \frac{1}{2}L + \frac{1}{2}.\]
Umstellen ergibt:
\[L = 1.\]
Die Folge \(a_n\) konvergiert gegen den Grenzwert 1. 

\section*{Aufgabe 3}
\subsection*{(i)}
Zu zeigen: $\forall a,b \in \mathbb{R} \exists q \in \mathbb{Q}: a < \sqrt{2}+q < b$ \\
Da $\mathbb{Q}$ dicht in $\mathbb{R}$ ist, gilt: $a < q < b$ und da $\sqrt{2} \in \mathbb{R}$: $a + \sqrt{2} < q + \sqrt{2} < b + \sqrt{2}$. \\
Da alle Summanden reelle Zahlen sind gilt die Aussage und somit auch $a < \sqrt{2} + q < b$
\subsection*{(ii)}
Angenommen, es gilt: $\exists q \in \mathbb{Q}: \sqrt{2}+q \in \mathbb{Q}$. \\ 
Sei $r = \sqrt{2} + q$, dann würde gelten: $\sqrt{2} = r - q$, mit $r, q \in \mathbb{Q}$. \\
Das widerspricht jedoch der Tatsache, dass $\sqrt{2}$ irrational ist. Da Summe aus einer irrationalen und einer rationalen Zahl irrational ist, muss auch $\sqrt{2}+q$ mit $q \in \mathbb{Q}$ irrational sein.
\subsection*{(iii)}
Die Menge $M$ besteht aus Zahlen der Form $\sqrt{2} + q$ mit $q \in \mathbb{Q}$. \\
Da $\mathbb{Q}$ abzählbar ist, gibt es eine bijektive Abbildung von $\mathbb{N}$ auf $\mathbb{Q}$. Daher gibt es auch eine bijektive Abbildung von $\mathbb{N}$ auf $M$. \\
Deshalb muss $M$ ebenfalls abzählbar sein.

\section*{Aufgabe 4}
Zu zeigen: $|a_m - a_n| < \epsilon$ mit $\epsilon > 0$.
Für $m > n$:
\[|a_m - a_n| = |(a_m - a_{m - 1}) + (a_{m-1} - a_{m-2}) + \dots + (a_{n+1} - a_n)|\]
Mit der Dreiecksgleichung:
\[|a_m - a_n| \leq |a_m - a_{m - 1}| + |a_{m-1} - a_{m-2}| + \dots + |a_{n+1} - a_n|\]
Verwendung der Schranke: $|a_{k+1} - a_k| \leq 2^{-k}$:
\[|a_m - a_n| \leq 2^{-n} + 2^{-(n+1)} + 2^{-(n+2)}+ \dots + 2^{-(m-1)}\]
Diese Summe ist eine geometrische Reihe mit dem ersten Term $2^{-n}$ und dem Quotienten $\frac{1}{2}$.
\[|a_m - a_n| \leq 2^{-n}(1 + \frac{1}{2} + \frac{1}{4} + \dots)\]
$\sum_{k=0}^{\infty} (\frac{1}{2})^k$ konvergiert gegen $\frac{1}{1 - \frac{1}{2}} = 2$
\[|a_m - a_n| \leq 2^{-n} \cdot 2 = 2^{-(n-1)}\]
Für ein $\epsilon > 0$ wählen wir $n$ so, dass: $2^{-(n-1)} < \epsilon$. Für hinreichend großes $n$ ist dies erfüllt. \\
Also:
\[|a_m - a_n| < 2^{-(n-1)} < \epsilon \implies |a_m - a_n| < \epsilon\]
\end{document}
