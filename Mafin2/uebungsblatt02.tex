\documentclass{article}
\usepackage{amsmath}
\usepackage{amssymb}
\usepackage[utf8]{inputenc}
\usepackage[ngerman]{babel}

\title{Übungsblatt 2}
\author{Pascal Diller, Timo Rieke}

\begin{document}
\maketitle

\section*{Aufgabe 1}
\subsection*{(i)}
1. Das Nullpolynom $p(x) = 0$ hat $a_0 = 0$, also $0 \in M_0$. \\
2. Seien $p(x) = \sum_{i=1}^n a_i x^i$ und $q(x) = \sum_{j=1}^m b_j x^j$ in $M_0$. Der konstante Term von $p(x)+q(x)$ ist $0+0=0$. Also $p(x)+q(x) \in M_0$. \\
3. Sei $p(x) = \sum_{i=1}^n a_i x^i \in M_0$ und $\lambda \in \mathbb{R}$. Der konstante Term von $\lambda p(x) = \sum_{i=1}^n (\lambda a_i) x^i$ ist $\lambda \cdot 0 = 0$. Also $\lambda p(x) \in M_0$. \\
Somit ist $M_0$ ein Unterraum.

\subsection*{(ii)} 
$M_1 = \{ p(x) \in \mathbb{R}[x] \mid a_0 = 1 \}$: Das Nullpolynom $p(x)=0$ hat $a_0=0$, also $0 \notin M_1$. \\ $M_1$ ist kein Unterraum. \\ 
$M_{\ge 0} = \{ p(x) \in \mathbb{R}[x] \mid a_0 \ge 0 \}$: \\
Sei $p(x) = 1 \in M_{\ge 0}$ (da $a_0=1 \ge 0$) und $\lambda = -1$. \\
Dann ist $\lambda p(x) = -1$. Der konstante Term ist $-1$, was nicht $\ge 0$ ist. \\
Also $\lambda p(x) \notin M_{\ge 0}$. $M_{\ge 0}$ ist nicht abgeschlossen bzgl. Skalarmultiplikation und somit kein Unterraum.

\section*{Aufgabe 2}
\subsection*{(i)} 
Seien $(a_n), (b_n), (c_n) \in F$ und $\lambda, \mu \in \mathbb{R}$. \\
Addition ist assoziativ $((a_n+b_n)+c_n = a_n+(b_n+c_n))$ und kommutativ $(a_n+b_n = b_n+a_n)$, da $+$ in $\mathbb{R}$ diese Eigenschaften hat. \\
Das neutrale Element ist die Nullfolge $(0)_{n\in\mathbb{N}}$, da $(a_n) + (0) = (a_n+0) = (a_n)$. \\
Das inverse Element zu $(a_n)$ ist $(-a_n)$, da $(a_n) + (-a_n) = (a_n-a_n) = (0)$. \\
$(F,+)$ ist eine abelsche Gruppe. \\
Für Skalarmultiplikation: \\
Ax 1: $1 \cdot (a_n) = (1 \cdot a_n) = (a_n)$. \\
Ax 2: $(\lambda \mu) \cdot (a_n) = ((\lambda \mu) a_n) = (\lambda (\mu a_n)) = \lambda \cdot (\mu (a_n))$ (Assoziativität von $\cdot$ in $\mathbb{R}$).\\
Ax 3: $\lambda \cdot ((a_n)+(b_n)) = \lambda \cdot (a_n+b_n) = (\lambda(a_n+b_n)) = (\lambda a_n + \lambda b_n) = \lambda(a_n) + \lambda(b_n)$ (Distributivität in $\mathbb{R}$).\\
Ax 4: $(\lambda + \mu) \cdot (a_n) = ((\lambda + \mu) a_n) = (\lambda a_n + \mu a_n) = \lambda (a_n) + \mu (a_n)$ (Distributivität in $\mathbb{R}$).\\
F ist ein $\mathbb{R}$-Vektorraum.

\subsection*{(ii)}
1. Die Nullfolge $(0)$ konvergiert gegen 0, also $0_F \in M_{\rightarrow 0}$. \\
2. Seien $(a_n), (b_n) \in M_{\rightarrow 0}$. Dann $\lim a_n = 0$ und $\lim b_n = 0$. Es gilt $\lim (a_n + b_n) = \lim a_n + \lim b_n = 0 + 0 = 0$. Also $(a_n) + (b_n) \in M_{\rightarrow 0}$. \\
3. Sei $(a_n) \in M_{\rightarrow 0}$ und $\lambda \in \mathbb{R}$. \\
Somit $\lim a_n = 0$. Es gilt $\lim (\lambda a_n) = \lambda \lim a_n = \lambda \cdot 0 = 0$. \\
Also $\lambda (a_n) \in M_{\rightarrow 0}$. \\
Somit ist $M_{\rightarrow 0}$ ein Unterraum.

\subsection*{(iii)}
Betrachte $(a_n)$ mit $a_n = n$ ($\lim a_n = \infty$, also $(a_n) \in M_{\rightarrow \infty}$) und $(b_n)$ mit $b_n = -n+1$ ($\lim b_n = -\infty$, also $(b_n) \in M_{\rightarrow \infty}$). \\
Die Summe ist $(a_n) + (b_n) = (n + (-n+1)) = (1)_{n\in\mathbb{N}}$. \\
Die konstante Folge (1) strebt nicht gegen $\pm\infty$ und ist nicht die Nullfolge. Also $(a_n)+(b_n) \notin M_{\rightarrow \infty}$. \\
$M_{\rightarrow \infty}$ ist nicht abgeschlossen bzgl. Addition und somit kein Unterraum. 

\section*{Aufgabe 3}
\subsection*{(i)}
zu zeigen:
\begin{enumerate}
    \item $0 \in M_{\text{ger}}$
    \item[] $\lambda f + g \in M_{\text{ger}}$
    \item $\implies f + g \in M_{\text{ger}}$
    \item $\implies \lambda f \in M_{\text{ger}}$
\end{enumerate}
Beweise: 
\begin{enumerate}
    \item $f(t) = 0 = f(-t) \in M_{\text{ger}}$
    \item $(f+g)(-t) = f(-t) + g(-t) = f(t) = g(t) = (f+g)(t)$
    \item $(\lambda f)(-t) = \lambda \cdot f(-t) = \lambda \cdot f(t) = (\lambda f)(t)$
\end{enumerate}
\subsection*{(ii)}
Für $M_{\mathbb{Q}}$ mit Gegenbeispiel: \\
\newline
Sei $f \in \mathbb{Q}$ und $f(x) = x$ für alle $x$. \\
$\implies f(\mathbb{Q}) = \mathbb{Q} \in \mathbb{Q}$ \\
\newline
Sei $\lambda = \sqrt 2$. Dann gilt: $(\lambda f)(x) = \sqrt 2 x$ \\
Für $x = 1$ gilt jedoch: $(\lambda f)(1) = \sqrt 2 \notin \mathbb{Q}$ \\
\newline
Somit gilt nicht: $\lambda f \in \mathbb{Q}$ \\
\newline
Für $M_{+1}$ mit Gegenbeispiel: \\
\newline
Seien $f(n) = n, g(n) = -n$. Dann gilt: \\
$f(n+1) = n + 1 = f(n) + 1$ \\
$g(n+1) = -(n + 1) = g(n) - 1$ \\
Allerdings: \\
$(f+g)(n+1) = f(n+1) + g(n+1) = (n + 1) - (n+1)$, \\
aber: $(f+g)(n) + 1 = n - n + 1 = 1 \neq 0 = (f+g)(n+1)$ \\
$\implies (f+g)(n+1) \neq (f+g)(n) \implies f+g \notin M_{+1} \implies$ Kriterium nicht erfüllt 

\section*{Aufgabe 4}
\begin{align*}
    &\lambda(-v) + \lambda(-u) \\
    =& (-\lambda)v + (-\lambda)(u) & -(\lambda v) = (-\lambda)v = \lambda(-v) \\
    =& (-\lambda) \cdot (v+u) & \text{Ax 3} \\
    =& (-1)\lambda \cdot (v+u) & (-1)v = -v \\
    =& -(\lambda(v+u)) & \text{Ax 2}
\end{align*}

\end{document}
