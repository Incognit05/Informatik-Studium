\documentclass{article}
\usepackage{graphicx} % Required for inserting images
\usepackage{amsmath}
\usepackage{amssymb}

\title{Übungsblatt 11}
\author{Pascal Diller, Timo Rieke}

\begin{document}
\maketitle
\section*{Aufgabe 1}
\subsection*{(i)}
$M = \{ x \in \mathbb{R}, x^3 > 8\}$ \\
$x^3 > 8 \Leftrightarrow x > 2 \implies M = (2, \infty)$ \\
\newline
$M$ ist nach oben unbeschränkt, da $M \to \infty$ geht. \\
$M$ ist nach unten durch $x > 2$ beschränkt. Demenstprechend ist das Infimum 2, welches jedoch nicht in $M$ enthalten ist, also gibt es kein Minimum. \\
\subsection*{(ii)}
$M = \{1 - \frac{2}{n}: n \in \mathbb{N}\}$ \\
Untersuchen der Werte: $M = \{-1, -\frac{1}{2}, \frac{1}{3}, \frac{1}{2}, \frac{3}{5}\}$ \\
Daraus folgt: $\lim_{n \to \infty} \left(1 - \frac{2}{n}\right) = 1$ \\
\newline
$M$ ist nach oben beschränkt, da alle Werte $< 1$, Somit ist das Supremum 1, jedoch nicht in $M$ enthalten, also kein Maximum. \\
\newline
$M$ ist nach unten beschränkt, da bei kleinstem $n = 1: \: 1- 2 = -1$. Somit ist das Infimum $-1$, was auch das Minimum ist, da $-1 \in M$.
\subsection*{(iii)}
$M = \{1 + \frac{1}{n} - \frac{1}{2m}: n,m \in \mathbb{N}\}$ \\
Größter Wert entsteht für $n = 1, m = 1\implies 1 + 1 - 0.5 = 1.5$ \\
Kleinster Wert ensteht für $n \to \infty, m \to \infty \implies 1 + 0 - 0 = 1$ \\
Also ist das Supremum 1.5, welches in $M$ ist und somit das Maximum. \\
Das Infimum ist 1, welches nicht in $M$ enthalten ist, also gibt es kein Minimum

\section*{Aufgabe 2}
\subsection*{(i)}
\[f(x) = x^3 \cos x - 3x^2 \sin x\]
\[g(x) = x^3 \cos x \implies g'(x) = (x^3)' \cos x + x^3(-\sin x) = 3x^2 \cos x - x^3 \sin x\]
\[h(x) = -3x^2 \sin x \implies h'(x) = -3(2x \sin x + x^2 \cos x) = -6x \sin x - 3x^2 \cos x\]
\[f'(x) = g'(x) + h'(x) = (3x^2 \cos x - x^3 \sin x) + (-6x \sin x - 3x^2 \cos x)\]
\[= x^3 \sin x - 6x \sin x = -\sin x (x^3 + 6x)\]


\subsection*{(ii)}
\[g(x) = \frac{1}{2}(x - \sin x \cos x)\]
\[g'(x) = \frac{1}{2} \left(1 - \frac{d}{dx}(\sin x \cos x) \right)\]
\[\frac{d}{dx}(\sin x \cos x) = (\sin x)' \cos x + \sin x (\cos x)' = \cos x \cos x - \sin x \sin x = \cos(2x)\]
\[\Rightarrow g'(x) = \frac{1}{2}(1 - \cos(2x))\]

\subsection*{(iii)}
\[h(x) = \frac{\sqrt{x} - 1}{\sqrt{x} + 1}, \quad x \in [0,\infty)\]
\[u(x) = \sqrt{x} - 1, \quad u'(x) = \frac{1}{2\sqrt{x}} \]
\[v(x) = \sqrt{x} + 1, \quad v'(x) = \frac{1}{2\sqrt{x}} \]
\[
h'(x) = \frac{u'(x) v(x) - u(x) v'(x)}{(v(x))^2}
= \frac{\frac{1}{2\sqrt{x}}(\sqrt{x} + 1) - \frac{1}{2\sqrt{x}}(\sqrt{x} - 1)}{(\sqrt{x} + 1)^2}
\]
\[
= \frac{\frac{1}{2\sqrt{x}} \left[(\sqrt{x} + 1) - (\sqrt{x} - 1)\right]}{(\sqrt{x} + 1)^2}
= \frac{\frac{1}{2\sqrt{x}}(2)}{(\sqrt{x} + 1)^2}
= \frac{1}{\sqrt{x}(\sqrt{x} + 1)^2}
\]

\subsection*{(iv)}
\[u(x) = \frac{1 - x^2}{x^2 + 1}\]
\[u(x) = \frac{f(x)}{g(x)}, \quad f(x) = 1 - x^2, \quad f'(x) = -2x\]
\[g(x) = x^2 + 1, \quad g'(x) = 2x\]
\[
u'(x) = \frac{f'(x)g(x) - f(x)g'(x)}{(g(x))^2}
= \frac{(-2x)(x^2 + 1) - (1 - x^2)(2x)}{(x^2 + 1)^2}
\]
\[
= \frac{-2x(x^2 + 1) - 2x(1 - x^2)}{(x^2 + 1)^2}
= \frac{-2x^3 - 2x - 2x + 2x^3}{(x^2 + 1)^2}
= \frac{-4x}{(x^2 + 1)^2}
\]

\section*{Aufgabe 3}
\subsection*{Differenzierbarkeit für \(x \ne 0\)}
\subsubsection*{Fall 1: \(x < 0\)}
Für \(x < 0\) ist \(f(x) = 1 + 2x + 3x^2\) ein Polynom. Polynome sind auf ganz \(\mathbb{R}\) unendlich oft differenzierbar. Also ist \(f\) für alle \(x < 0\) differenzierbar.
\subsubsection*{Für \(x > 0\):}
Für \(x > 0\) ist \(f(x) = 1 + 2\sin x + x^{\frac{3}{2}}\sin\frac{1}{x}\). Die Funktionen \(x \mapsto 1\), \(x \mapsto 2\sin x\), \(x \mapsto x^{\frac{3}{2}}\) und \(x \mapsto \sin\frac{1}{x}\) sind für \(x > 0\) alle differenzierbar. Somit ist auch deren Summe und Produkt differenzierbar, was bedeuted, dass \(f\) für alle \(x > 0\) differenzierbar ist.


\subsection*{Differenzierbarkeit bei \(x = 0\)}
\subsubsection*{Stetigkeit bei \(x=0\):}
Funktionswert: 
\[f(0) = 1 + 2(0) + 3(0)^2 = 1\]
Linksseitiger Grenzwert: 
\[\lim_{x\nearrow0} f(x) = \lim_{x\nearrow0} (1+2x+3x^2) = 1\]
Rechtsseitiger Grenzwert: 
\[\lim_{x\searrow0} f(x) = \lim_{x\searrow0} (1 + 2\sin x + x\sqrt{x}\sin\frac{1}{x})\]
\[\lim_{x\searrow0} (1 + 2\sin x) = 1 + 2\sin(0) = 1\]
Für den Term \(x\sqrt{x}\sin\frac{1}{x}\) gilt die Abschätzung 
\[-x\sqrt{x} \le x\sqrt{x}\sin\frac{1}{x} \le x\sqrt{x}\]
, da \(-1 \le \sin(\cdot) \le 1\). \\
Weil \(\lim_{x\searrow0} x\sqrt{x} = 0\), folgt aus dem Sandwichsatz  
\[\lim_{x\searrow0} x\sqrt{x}\sin\frac{1}{x} = 0\]
Somit ist \(\lim_{x\searrow0} f(x) = 1 + 0 = 1\). \\
Da der linksseitige und rechtsseitige Grenzwert mit dem Funktionswert übereinstimmen, ist \(f\) in \(x_0=0\) stetig.

\subsubsection*{Differenzierbarkeit bei \(x=0\):} 

Linksseitige Ableitung:
\[f'_-(0) = \lim_{h\nearrow0} \frac{f(h)-f(0)}{h} = \lim_{h\nearrow0} \frac{(1+2h+3h^2)-1}{h} = \lim_{h\nearrow0} \frac{2h+3h^2}{h} = \lim_{h\nearrow0} (2+3h) = 2.\]
Rechtsseitige Ableitung:

\[f'_+(0) = \lim_{h\searrow0} \frac{f(h)-f(0)}{h} = \lim_{h\searrow0} \frac{(1+2\sin h + h\sqrt{h}\sin\frac{1}{h})-1}{h} \]
\[= \lim_{h\searrow0} \left( \frac{2\sin h}{h} + \sqrt{h}\sin\frac{1}{h} \right)\]
Der erste Summand konvergiert gegen \(2 \cdot 1 = 2\). \\
Der zweite Summand konvergiert nach dem Sandwichsatz gegen 0. \\
Somit ist \(f'_+(0) = 2 + 0 = 2\). \\
Da \(f'_-(0) = f'_+(0) = 2\), ist die Funktion \(f\) an der Stelle \(x_0=0\) differenzierbar mit \(f'(0)=2\). \\
Somit ist die Funktion \(f\) ist auf ganz \(\mathbb{R}\) differenzierbar.

\section*{Aufgabe 4}
\subsection*{(i)}
\[\frac{d}{dx}\cos x = \lim_{h\to0} \frac{\cos(x+h) - \cos x}{h} \]
\[= \lim_{h\to0} \frac{-2\sin\left(\frac{x+h+x}{2}\right)\sin\left(\frac{x+h-x}{2}\right)}{h} \]
\[= \lim_{h\to0} \frac{-2\sin\left(x+\frac{h}{2}\right)\sin\left(\frac{h}{2}\right)}{h} \]
\[= \lim_{h\to0} \left(-\sin\left(x+\frac{h}{2}\right) \cdot \frac{\sin(h/2)}{h/2}\right) \]
Wir betrachten die Grenzwerte der beiden Faktoren einzeln. \\
Da die Sinusfunktion stetig ist, gilt \(\lim_{h\to0} -\sin(x+\frac{h}{2}) = -\sin(x)\). \\
Mit der Substitution \(u = h/2\) und dem bekannten Grenzwert \(\lim_{u\to0} \frac{\sin u}{u} = 1\) erhalten wir \(\lim_{h\to0} \frac{\sin(h/2)}{h/2} = 1\). \\
Zusammengesetzt ergibt sich:
\[\frac{d}{dx}\cos x = -\sin(x) \cdot 1 = -\sin x.\]

\subsection*{(ii)}
\[f'(0) = \lim_{x\to0} \frac{f(x)-f(0)}{x}.\]
Aus der Bedingung \(|f(x)| \le \mu x^2\) folgt durch Einsetzen von \(x=0\), dass \(|f(0)| \le \mu \cdot 0^2 = 0\), also muss \(f(0)=0\) sein.
Der Differenzenquotient ist somit \(\frac{f(x)}{x}\). Wir schätzen seinen Betrag ab:
\[\left|\frac{f(x)}{x}\right| = \frac{|f(x)|}{|x|} \le \frac{\mu x^2}{|x|} = \mu|x|.\]
Dies bedeutet \(-\mu|x| \le \frac{f(x)}{x} \le \mu|x|\). 
Da \(\lim_{x\to0} (-\mu|x|) = 0\) und \(\lim_{x\to0} (\mu|x|) = 0\), folgt:
\[\lim_{x\to0} \frac{f(x)}{x} = 0.\]
Somit ist \(f\) in 0 differenzierbar und es gilt \(f'(0)=0\).

\subsection*{(iii)}
Da \(b\) eine beschränkte Funktion ist, existiert eine Konstante \(M \ge 0\), sodass \(|b(x)| \le M\) für alle \(x \in (-1,1)\). \\
Wir betrachten den Betrag von \(g(x)\):
\[|g(x)| = |b(x)x^2| = |b(x)| \cdot |x^2| \le M \cdot x^2.\]
Die Bedingung \(|g(x)| \le M x^2\) entspricht genau der Voraussetzung aus Aufgabenteil (ii) mit \(\mu = M\).
Daher folgt direkt aus dem Ergebnis von (ii), dass \(g\) in 0 differenzierbar ist mit \(g'(0)=0\).


\end{document}
