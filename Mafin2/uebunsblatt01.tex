\documentclass{article}
\usepackage{amsmath}
\usepackage{amssymb}
\usepackage[utf8]{inputenc}
\usepackage[ngerman]{babel}

\title{Übungsblatt 1}
\author{Pascal Diller (rip62jik), Timo Rieke (bop59buz)}

\begin{document}
\maketitle

\section*{Aufgabe 1}
Sei $z=1+i$.

\subsection*{(i)}
\[ z^5 = (1+i)^5 = \sum_{k=0}^{5} \binom{5}{k} 1^{5-k} i^k \]
\[ = \binom{5}{0}i^0 + \binom{5}{1}i^1 + \binom{5}{2}i^2 + \binom{5}{3}i^3 + \binom{5}{4}i^4 + \binom{5}{5}i^5 \]
\[ = 1 \cdot 1 + 5 \cdot i + 10 \cdot (-1) + 10 \cdot (-i) + 5 \cdot 1 + 1 \cdot i \]
\[ = 1 + 5i - 10 - 10i + 5 + i = (1-10+5) + (5-10+1)i \]
\[ = -4 - 4i \]

\subsection*{(ii)}
Für $z = 1+i$: Betrag $r = |z| = \sqrt{1^2+1^2} = \sqrt{2}$.
Argument $\phi$: $\tan(\phi) = \frac{\text{Im}(z)}{\text{Re}(z)} = \frac{1}{1} = 1$. Da $z$ im 1. Quadrant liegt, ist $\phi = \frac{\pi}{4}$.
Polarform $z$: $r=\sqrt{2}$, $\phi=\pi/4$. Also $z = \sqrt{2}e^{i\pi/4}$. \\ \newline  
Für $z^5$:
\[ z^5 = (re^{i\phi})^5 = r^5 e^{i5\phi} \]
\[ r^5 = (\sqrt{2})^5 = 4\sqrt{2} \]
\[ 5\phi = 5 \cdot \frac{\pi}{4} = \frac{5\pi}{4} \]
Polarform $z^5$: $r_5=4\sqrt{2}$, $\phi_5=5\pi/4$. Also $z^5 = 4\sqrt{2}e^{i5\pi/4}$.

\subsection*{(iii)}
\[ z^5 = 4\sqrt{2} e^{i5\pi/4} = 4\sqrt{2} (\cos(5\pi/4) + i\sin(5\pi/4)) \]
Mit $\cos(5\pi/4) = -\frac{1}{\sqrt{2}}$ und $\sin(5\pi/4) = -\frac{1}{\sqrt{2}}$:
\[ z^5 = 4\sqrt{2} (-\frac{1}{\sqrt{2}} + i(-\frac{1}{\sqrt{2}})) \]
\[ z^5 = 4\sqrt{2} \cdot (-\frac{1}{\sqrt{2}}) + i \cdot 4\sqrt{2} \cdot (-\frac{1}{\sqrt{2}}) \]
\[ z^5 = -4 - 4i \]
$\text{Re}(z^5) = -4$, $\text{Im}(z^5) = -4$.

\subsection*{(iv)}
\[ z^{17} = r^{17} e^{i17\phi} = (\sqrt{2})^{17} e^{i17\pi/4} \]
\[ r^{17} = (\sqrt{2})^{17} = 2^{17/2} = 2^8 \sqrt{2} = 256\sqrt{2} \]
Der Winkel ist $17\pi/4 = 4\pi + \pi/4$, äquivalent zu $\pi/4$.
\[ z^{17} = 256\sqrt{2} e^{i\pi/4} = 256\sqrt{2} (\cos(\pi/4) + i\sin(\pi/4)) \]
Mit $\cos(\pi/4) = \frac{1}{\sqrt{2}}$ und $\sin(\pi/4) = \frac{1}{\sqrt{2}}$:
\[ z^{17} = 256\sqrt{2} (\frac{1}{\sqrt{2}} + i\frac{1}{\sqrt{2}}) \]
\[ z^{17} = 256\sqrt{2} \cdot \frac{1}{\sqrt{2}} + i \cdot 256\sqrt{2} \cdot \frac{1}{\sqrt{2}} \]
\[ z^{17} = 256 + 256i \]
$\text{Re}(z^{17}) = 256$, $\text{Im}(z^{17}) = 256$.

\section*{Aufgabe 2}

\subsection*{(i)}
\[ \sum_{n=0}^{\infty} \left| \frac{z^n}{n!} \right| = \sum_{n=0}^{\infty} \frac{|z^n|}{n!} = \sum_{n=0}^{\infty} \frac{|z|^n}{n!} \]
Sei $r = |z| \in \mathbb{R}_{\ge 0}$. Die Reihe $\sum_{n=0}^{\infty} \frac{r^n}{n!} = e^r$ konvergiert für alle $r \in \mathbb{R}$.
Daher konvergiert $\sum \frac{|z|^n}{n!}$ für alle $z \in \mathbb{C}$.
Somit konvergiert $\exp(z) = \sum \frac{z^n}{n!}$ absolut $\forall z \in \mathbb{C}$.

\subsection*{(ii)}
Setze $z=i\varphi$:
\[ \exp(i\varphi) = \sum_{n=0}^{\infty} \frac{(i\varphi)^n}{n!} \]
Da die Reihe absolut konvergiert:
\[ = \sum_{k=0}^{\infty} \frac{(i\varphi)^{2k}}{(2k)!} + \sum_{k=0}^{\infty} \frac{(i\varphi)^{2k+1}}{(2k+1)!} \]
\[ = \sum_{k=0}^{\infty} \frac{i^{2k}\varphi^{2k}}{(2k)!} + \sum_{k=0}^{\infty} \frac{i^{2k+1}\varphi^{2k+1}}{(2k+1)!} \]
Mit $i^{2k} = (-1)^k$ und $i^{2k+1} = i(-1)^k$:
\[ = \sum_{k=0}^{\infty} \frac{(-1)^k \varphi^{2k}}{(2k)!} + i \sum_{k=0}^{\infty} \frac{(-1)^k \varphi^{2k+1}}{(2k+1)!} \]
Reihendarstellungen:
\[ = \cos(\varphi) + i \sin(\varphi) \]

\section*{Aufgabe 3}
\subsection*{(i)}
Die Menge ist nicht leer (min. 3 Elemente) \\
\newline
Funktionenkombinationen sind assoziativ, also gilt $f \circ ( g \circ h) = (f \circ g) \circ h$. Somit ist die Operation assoziativ.\\
\newline
Die Identitätsabbildung $\text{Id}_X$ erfüllt $f \circ \text{Id}_X = \text{Id}_X \circ f = f$ für alle $f \in F$. Somit hat die Operation ein neutrales Element. \\
\newline
Zu jeder bijektiven Abbildung $f \in F$ gibt es eine bijektive Inverse $f^{-1}$. Somit ist jedes Element invertierbar. \\
\newline
$\implies (F, \circ, \text{Id}_X)$ ist eine Gruppe. 

\subsection*{(ii)}
Die Gruppe ist nicht kommutativ, da sich bereits für $X = {a, b, c}$ permutationen $f, g$ finden lassen, für die gilt: $f \circ g \neq g \circ f$. \\
Beispiel: \\
$f = (a \: b)$ \\
$g = (b \: c)$
\begin{align*}
    f \circ g \to & \quad f(g(a)) = f(a) = b \\
    & \quad f(g(b)) = f(c) = c \\
    & \quad f(g(c)) = f(b) = a \\ 
\end{align*}
\begin{align*}
    g \circ f \to & \quad g(f(a)) = g(a) = c \\
    & \quad g(f(b)) = g(c) = a \\
    & \quad g(f(c)) = g(b) = b \\ 
\end{align*}
$\implies f \circ g \neq g \circ f$

\section*{Aufgabe 4}
\subsection*{(i)}
\begin{tabular}{l l}
    $(k+l) \cdot (k + (-l))$ & \\
    $=kk+k(-l)+lk+l(-l)$ & Multiplikation steht distributiv über der Addition \\
    $=kk + (-(kl)) + kl + (-(ll))$ & $(-k)l = k(-l) = -(kl)$ \\
    $=kk + (-(ll))$ & inv.
\end{tabular}
\subsection*{(ii)}
Wenn $n$ nicht prim ist, dann gibt es $a, b \in \mathbb{Z}$ mit $0 < a, b < n$ mit $a \cdot b = 0 \mod n$. \\
Da $a, b \neq 0$, aber $a \cdot b = 0$ gibt es Nullteiler und $\mathbb{Z}/n\mathbb{Z}$ ist kein Körper. \\
Beispiel: \\
$\mathbb{Z}/6\mathbb{Z} = \{0, 1, 2, 3, 4, 5\}$ \\
$2 \cdot 3 = 6 \mod n = 0$ aber $a, b \neq 0 \implies$ Nullteiler

\end{document}
